
\section{Pappus' Theorem}

The next result beautifully combines two of our previous topics, center of mass and surface areas/volumes of figures of revolution.  

Let's revisit the torus and notice a too-good-to-be-true-but-it-is kind of fact.  

\begin{exercise}{Surface Area as Perimeter Times Length of Revolution! \Coffeecup}
Recall our construction of the torus as the revolution of the circle given by $$(x-R)^2+y^2=r^2 $$ about the $y$-axis. 

\begin{itemize}
\item Consider the center of the circle, $(R,0)$.  What is the length of the path this point takes as it completes one revolution about the $y$-axis?

\vspace*{1in}

\item What is the perimeter of the circle?

\vspace*{1in}

\item Multiply the two above quantities together.  How does this product compare to the surface area of the torus as computed in Exercise \ref{Ford}.\ref{torus}? 

\end{itemize}
\end{exercise}