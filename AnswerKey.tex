\par 
 {\noindent \protect \bf  Exercise ~0.1.1.3.} The sine gumdrop is just a translation $\pi /2$ units to the right of the cosine gumdrop. So, we would expect the center of mass to have the same $y$-coordinate but have an $x$-coordinate that is $\pi /2$ units larger. Indeed, when computed with the moment integrals, we get $\left (\protect \mathaccentV {bar}016{x},\protect \mathaccentV {bar}016{y}\right )=\left (\pi /2,\pi /8\right )$.  \protect \newline  \protect \newline  
 {\noindent \protect \bf  Exercise ~0.1.3.1.} The diagonals have the equations $$ y=\protect \frac  {b}{a}x \protect \text  { and } y=\protect \frac  {b-2c}{a}x+c$$ with intersection point $(a/2,b/2)$, which is also the center of mass of the region. \protect \newline  \protect \newline  
 {\noindent \protect \bf  Exercise ~0.1.4.2.} The coordinates of the vertices are (0,0), (0,$c$), and ($a,b$). Two of the medians are $$y=\protect \frac  {b+c}{a}x \protect \text  { and } y=\protect \frac  {2b-c}{2a}x+\protect \frac  {c}{2}$$ and their intersection point (and center of mass of the triangle) is $\left (\protect \frac  {a}{3},\protect \frac  {b+c}{3}\right )$. \protect \newline  \protect \newline  
 {\noindent \protect \bf  Exercise ~0.1.5.1.} The center of mass is $\left ( 0, \protect \frac  {4r}{3\pi }\right )$. \protect \newline  \protect \newline  
