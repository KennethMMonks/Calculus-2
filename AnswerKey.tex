\par 
 {\noindent \protect \bf  Exercise ~0.1.1.2.} The exact arc length is $\protect \frac  {2\protect \sqrt  {5}+\protect \qopname  \relax o{ln}\left | 2+\protect \sqrt  {5}\right |}{4}$. \protect \newline  \protect \newline  
 {\noindent \protect \bf  Exercise ~0.1.3.1.} The length of the graph of the natural logarithm from (1,0) to (e,1) is $$\protect \sqrt  {e^2+1}+\protect \frac  {1}{2}\protect \qopname  \relax o{ln}\left | \protect \frac  {\protect \sqrt  {e^2+1}-1}{\protect \sqrt  {e^2+1}+1}\right |-\protect \sqrt  {2}+\protect \frac  {1}{2}\protect \qopname  \relax o{ln}\left | \protect \frac  {\protect \sqrt  {2}+1}{\protect \sqrt  {2}-1}\right | $$ which is roughly 2.003497. Also, notice that the natural exponential function is just the inverse of the natural logarithm; think about what this means regarding arc length! \protect \newline  \protect \newline  
 {\noindent \protect \bf  Exercise ~0.1.4.1.} The two-frusta approximation is $$\protect \frac  {\pi }{8}\left (\protect \sqrt  {5}+3\protect \sqrt  {13} \right )\approx 5.126 $$ The exact value of the surface area is $$\protect \frac  {\pi }{6}\left (5\protect \sqrt  {5}-1 \right )\approx 5.3304 $$ which is just slightly larger, as one would expect.  \protect \newline  \protect \newline  
 {\noindent \protect \bf  Exercise ~0.1.5.3.} Notice that if you turn the pyramid sideways, you can get the 2D side view to be almost exactly the same as we had for the cone! The volume is $V=\protect \frac  {1}{3}r^2h$. \protect \newline  \protect \newline  
 {\noindent \protect \bf  Exercise ~0.1.5.4.} The volume is $V=\protect \frac  {1}{6}abc$. \protect \newline  \protect \newline  
 {\noindent \protect \bf  Exercise ~0.1.5.6.} The circular cross section has equation $x^2+y^2=1$. If you solve for the $y$ coordinate, you'll have a function for the radius of a circular cross section at position $x$. This formula can be integrated to produce the volume $V=\protect \frac  {4}{3}\pi r^3$. \protect \newline  \protect \newline  
 {\noindent \protect \bf  Exercise ~0.1.5.8.} The parabolic bowl has volume $\pi /2$ and occupies exactly fifty percent of the cylinder it sits in! \protect \newline  \protect \newline  
 {\noindent \protect \bf  Exercise ~0.1.5.11.} The volume estimate with a single cylinder is $2\pi $. To get the heights of the six cylindrical shells, you'll need to use the fact that $x^2+y^2=1$ for every point on the boundary of the circle. With six shells, the volume estimate is $\pi \cdot \protect \frac  {\protect \sqrt  {35} + 12 \protect \sqrt  {2} + 15 \protect \sqrt  {3} + 14 \protect \sqrt  {5} + 9 \protect \sqrt  {11}}{108}\approx 1.018\pi $. The first is an overestimate, whereas the second is an underestimate. \protect \newline  \protect \newline  
 {\noindent \protect \bf  Exercise ~0.1.5.12.} The function $g(x)=\protect \sqrt  {1-x^2}$ represents just the QI $y$-coordinate. It needs to be doubled to represent the height of the shell since the each shell extends the same vertical distance into QIII. Once the integral is evaluated, it will return the exact volume $\protect \frac  {4}{3}\pi $. \protect \newline  \protect \newline  
 {\noindent \protect \bf  Exercise ~0.1.6.1.} The volume of the torus is $$V=2\pi ^2Rr^2 $$ and the surface area is $$SA=4\pi ^2Rr. $$ \protect \newline  \protect \newline  
