\par 
 {\noindent \protect \bf  Exercise ~1.0.0.1.} Saying that $F$ is an antiderivative of $f$ is equivalent to saying the derivative of $F$ is $f$. That is, $F'(x)=f(x)$. The Fundamental Theorem of Calculus states that after antidifferentiating the integrand, one can plug the bounds into the antiderivative and take their difference in order to calculate the integral. Because $F'(x)=f(x)$ by the definition of an antiderivative, a good way to check that your antiderivative $F$ is correct is to take its derivative $F'$. You should get the original function, $f$. \protect \newline  \protect \newline  
 {\noindent \protect \bf  Exercise ~1.1.1.1.} $ \intop f'(g(x)) \cdot g'(x) \mathop {}\protect \tmspace  -\thinmuskip {.1667em}\protect \mathtt  {d}x = \intop \left (f\left (g(x)\right )\right )' \mathop {}\protect \tmspace  -\thinmuskip {.1667em}\protect \mathtt  {d}x= f\left (g(x)\right )+C$ \protect \newline  \protect \newline  
 {\noindent \protect \bf  Exercise ~1.1.1.6.} Use the substitutions $u=x^2+x+8, \protect \qopname  \relax o{ln}(x),$ and $-x^2$. In the last case, the $\mathop {}\protect \tmspace  -\thinmuskip {.1667em}\protect \mathtt  {d}u$ term has nothing to cancel the $x$ with! \protect \newline  \protect \newline  
 {\noindent \protect \bf  Exercise ~1.1.2.2.} To have four intervals in the Riemann sum, $\Delta x$ would be 1 while $\Delta u$ would be 2. Thus, the width of each rectangle is getting doubled, since to convert between $u$ and $x$ we use the formula $u=2x+1$. The ``plus one'' merely slides all the rectangles one unit to the right, but it does not stretch their width at all, so it does not affect their area. Thus, the slope of the graph of $u=2x+1$ is the only thing that mattered regarding our conversion between $x$ and $u$. That is to say, the quantity $\mathop {}\protect \tmspace  -\thinmuskip {.1667em}\protect \mathtt  {d}u/ \mathop {}\protect \tmspace  -\thinmuskip {.1667em}\protect \mathtt  {d}x$ gives us the scaling factor. \protect \newline  \protect \newline  
 {\noindent \protect \bf  Exercise ~1.1.2.3.} The definite integral evaluates to roughly 0.95. The horizontal scaling factor at each $x$-coordinate should correspond to the derivative $\mathop {}\protect \tmspace  -\thinmuskip {.1667em}\protect \mathtt  {d}u /\mathop {}\protect \tmspace  -\thinmuskip {.1667em}\protect \mathtt  {d}x$ at each point.  \protect \newline  \protect \newline  
 {\noindent \protect \bf  Exercise ~1.2.1.3.} By factoring out the quantity $(x+1)^{3/2}$, both answers can be brought into the form $(x+1)^{3/2}\left (\protect \frac  {2}{5}x-\protect \frac  {4}{15}\right )+C$. \protect \newline  \protect \newline  
 {\noindent \protect \bf  Exercise ~1.2.2.2.} Use the substitution $u=1-x^2$. \protect \newline  \protect \newline  
 {\noindent \protect \bf  Exercise ~1.2.2.3.} The antiderivative is $x\protect \qopname  \relax o{ln}(x)-x+C$. \protect \newline  \protect \newline  
 {\noindent \protect \bf  Exercise ~1.2.3.5.} The antiderivative is $\protect \frac  {1}{2}\left (\protect \qopname  \relax o{sec}(x)\protect \qopname  \relax o{tan}(x)+\protect \qopname  \relax o{ln}|\protect \qopname  \relax o{sec}(x)+\protect \qopname  \relax o{tan}(x)|\right )+C$. \protect \newline  \protect \newline  
 {\noindent \protect \bf  Exercise ~1.3.0.1.} Use the substitution $u=\protect \sqrt  {x}$ to transform the first integral into $\intop 2u\protect \qopname  \relax o{cos}(u) \mathop {}\protect \tmspace  -\thinmuskip {.1667em}\protect \mathtt  {d}u$. \protect \newline  \protect \newline  
 {\noindent \protect \bf  Exercise ~1.3.0.2.} Choosing $u=\protect \qopname  \relax o{ln}(x)$ will make the logarithm disappear upon differentiation. The opposite choice will not clean up the log. \protect \newline  \protect \newline  
 {\noindent \protect \bf  Exercise ~1.4.0.2.} The antiderivative is $\protect \frac  {1}{3}\protect \qopname  \relax o{sin}^3(x)+C$. \protect \newline  \protect \newline  
 {\noindent \protect \bf  Exercise ~1.4.1.2.} Since seven is odd, when we pulled out one factor of sine, we ended up with the sixth power of sine remaining. Since six is even, we were able to express it as a power of a perfect square of sine, which in turn let us rewrite as cosines using the Pythagorean identity. \protect \newline  \protect \newline  
 {\noindent \protect \bf  Exercise ~1.4.1.3.} The first antiderivative is $-\protect \frac  {1}{3}\protect \qopname  \relax o{cos}^3(x)+\protect \frac  {2}{5}\protect \qopname  \relax o{cos}^5(x)-\protect \frac  {1}{7}\protect \qopname  \relax o{cos}^7(x)+C$. For the second, rewrite as $(1-\protect \qopname  \relax o{sin}^2(x))^4\protect \qopname  \relax o{cos}(x)$ and proceed by letting $u=\protect \qopname  \relax o{sin}(x)$. \protect \newline  \protect \newline  
 {\noindent \protect \bf  Exercise ~1.4.1.4.} Often when trying to show that two antiderivatives are compatible, it is easiest to verify that their difference is a constant.  \protect \newline  \protect \newline  
 {\noindent \protect \bf  Exercise ~1.4.1.4.} The substitution $u=\protect \qopname  \relax o{sin}(x)$ is much cleaner since the other will involve having to expand a binomial to the fifth power. The antiderivative is $\protect \frac  {1}{12}\protect \qopname  \relax o{sin}^{12}(x)-\protect \frac  {1}{14}\protect \qopname  \relax o{sin}^{14}(x)+C$. \protect \newline  \protect \newline  
 {\noindent \protect \bf  Exercise ~1.4.2.1.} The exponent on sine is zero, which is indeed even. Thus both exponents are even in this case. \protect \newline  \protect \newline  
 {\noindent \protect \bf  Exercise ~1.4.2.3.} When all like terms are combined and the one-eighth is distributed, the result is $\protect \frac  {5}{16}x+\protect \frac  {1}{4}\protect \qopname  \relax o{sin}(2x)-\protect \frac  {1}{48}\protect \qopname  \relax o{sin}^3(2x)+\protect \frac  {3}{64}\protect \qopname  \relax o{sin}(4x)+C$. \protect \newline  \protect \newline  
 {\noindent \protect \bf  Exercise ~1.4.2.4.} The antiderivative to $\protect \qopname  \relax o{cos}^6(x)$ came out to \par $$\protect \frac  {5}{16}x+\protect \frac  {1}{4}\protect \qopname  \relax o{sin}(2x)-\protect \frac  {1}{48}\protect \qopname  \relax o{sin}^3(2x)+\protect \frac  {3}{64}\protect \qopname  \relax o{sin}(4x)+C$$ Before we differentiate, first bash everything back down to an ``$x$'' in the argument using double angle identities. This produces \par $$\protect \frac  {5}{16}x+\protect \frac  {1}{2}\protect \qopname  \relax o{sin}(x)\protect \qopname  \relax o{cos}(x)-\protect \frac  {1}{6}\protect \qopname  \relax o{sin}^3(x)\protect \qopname  \relax o{cos}^3(x)+\protect \frac  {3}{16}\protect \qopname  \relax o{sin}(x)\protect \qopname  \relax o{cos}^3(x)-\protect \frac  {3}{16}\protect \qopname  \relax o{sin}^3(x)\protect \qopname  \relax o{cos}(x)+C$$ Factor out a sine and use the Pythagorean Identity to get everything else in terms of cosine. This produces $$\protect \frac  {5}{16}x+\protect \qopname  \relax o{sin}(x)\left (\protect \frac  {5}{16}\protect \qopname  \relax o{cos}(x)+\protect \frac  {5}{24}\protect \qopname  \relax o{cos}^3(x)+\protect \frac  {1}{6}\protect \qopname  \relax o{cos}^5(x)\right )+C$$ Then we differentiate and obtain $$\protect \frac  {5}{16}+\protect \qopname  \relax o{cos}(x)\left (\protect \frac  {5}{16}\protect \qopname  \relax o{cos}(x)+\protect \frac  {5}{24}\protect \qopname  \relax o{cos}^3(x)+\protect \frac  {1}{6}\protect \qopname  \relax o{cos}^5(x)\right )-\protect \qopname  \relax o{sin}^2(x)\left (\protect \frac  {5}{16}+\protect \frac  {5}{8}\protect \qopname  \relax o{cos}^2(x)+\protect \frac  {5}{6}\protect \qopname  \relax o{cos}^4(x)\right )$$ to which we apply the Pythagorean Identity $\protect \qopname  \relax o{sin}^2(x)=1-\protect \qopname  \relax o{cos}^2(x)$ to produce $$\protect \frac  {5}{16}+\protect \qopname  \relax o{cos}(x)\left (\protect \frac  {5}{16}\protect \qopname  \relax o{cos}(x)+\protect \frac  {5}{24}\protect \qopname  \relax o{cos}^3(x)+\protect \frac  {1}{6}\protect \qopname  \relax o{cos}^5(x)\right )-\left (1-\protect \qopname  \relax o{cos}^2(x)\right )\left (\protect \frac  {5}{16}+\protect \frac  {5}{8}\protect \qopname  \relax o{cos}^2(x)+\protect \frac  {5}{6}\protect \qopname  \relax o{cos}^4(x)\right )$$ This will simplify to $\protect \qopname  \relax o{cos}^6(x)$ once you expand and combine like terms.  \protect \newline  \protect \newline  
 {\noindent \protect \bf  Exercise ~1.4.2.5.} For the first, apply the identity $\protect \qopname  \relax o{sin}^2(3x)=\protect \frac  {1-\protect \qopname  \relax o{cos}(6x)}{2}$ and proceed. For the second, notice that $\protect \qopname  \relax o{sin}^4(x)$ can be rewritten as $\left (\protect \qopname  \relax o{sin}^2(x)\right )^2$, after which the half-angle identity can be applied. \protect \newline  \protect \newline  
 {\noindent \protect \bf  Exercise ~1.5.1.3.} First apply all the product and chain rules to reach the expression $$\protect \frac  {3}{\protect \sqrt  {1-\protect \frac  {x^2}{4}}}+4\protect \sqrt  {1-\protect \frac  {x^2}{4}}+\protect \frac  {-x^2}{\protect \sqrt  {1-\protect \frac  {x^2}{4}}}+\protect \sqrt  {1-\protect \frac  {x^2}{4}}\left (1-\protect \frac  {3}{2}x^2\right )+\protect \frac  {-x}{4\protect \sqrt  {1-\protect \frac  {x^2}{4}}}\left (x-\protect \frac  {x^3}{2}\right ) $$ Put all terms over the common denominator $\protect \sqrt  {4-x^2}$ and combine like terms in the numerator. Notice the numerator becomes $\left (4-x^2\right )^2$ and then reduce for the win! \protect \newline  \protect \newline  
 {\noindent \protect \bf  Exercise ~1.5.1.4.} The antiderivative is $2^{18}\left (\protect \frac  {\left (1-x^2/16\right )^{9/2}}{9}-\protect \frac  {\left (1-x^2/16\right )^{7/2}}{7}\right )+C$ \protect \newline  \protect \newline  
 {\noindent \protect \bf  Exercise ~1.5.2.3.} Exercise \protect \ref  {reappear}.\protect \ref  {seccubed} will be helpful! The antiderivative is $\protect \frac  {x\protect \sqrt  {x^2-4}}{2}-2\protect \qopname  \relax o{ln}|x+\protect \sqrt  {x^2-4}|+C$. \protect \newline  \protect \newline  
 {\noindent \protect \bf  Exercise ~1.5.4.3.} The antiderivative is $-\protect \frac  {1}{5}\protect \frac  {2x+1}{x^2+x-1}+\protect \frac  {4\protect \sqrt  {5}}{25}\protect \qopname  \relax o{ln}\left (\protect \frac  {2x+1+\protect \sqrt  {5}}{2\protect \sqrt  {x^2+x-1}}\right )+C$. Note that one can expand using properties of logarithms and then rename $C$ as $C-\protect \frac  {4\protect \sqrt  {5}}{25}\protect \qopname  \relax o{ln}(2)$ since it is anyhow just an arbitrary constant. Thus, we can slightly clean up the answer to become $-\protect \frac  {1}{5}\protect \frac  {2x+1}{x^2+x-1}+\protect \frac  {4\protect \sqrt  {5}}{25}\protect \qopname  \relax o{ln}\left (2x+1+\protect \sqrt  {5}\right )-\protect \frac  {2\protect \sqrt  {5}}{25}\protect \qopname  \relax o{ln}\left (x^2+x-1\right )+C.$ \protect \newline  \protect \newline  
 {\noindent \protect \bf  Exercise ~1.6.1.5.} Using properties of logarithms, both answers should be able to be put in the form $\protect \qopname  \relax o{ln}\left | \protect \sqrt  {\protect \frac  {x-1}{x+1}}\right |+C$ \protect \newline  \protect \newline  
 {\noindent \protect \bf  Exercise ~1.6.3.1.} \textbullet The function $\protect \frac  {1}{x^2-9x+20}$ has $\protect \qopname  \relax o{ln}\left |\protect \frac  {x-5}{x-4}\right |+C$ as its antiderivative. \textbullet The factorization $x^4-9=\left (x^2+3\right )\left (x-\protect \sqrt  {3}\right )\left (x+\protect \sqrt  {3}\right )$ will produce the following setup: $$\protect \frac  {1}{x^4-9}=\protect \frac  {Ax+B}{x^2+3}+\protect \frac  {C}{x-\protect \sqrt  {3}}+\protect \frac  {D}{x+\protect \sqrt  {3}} $$ in which you can then solve for the coefficients and antidifferentiate. \textbullet The function $\protect \frac  {x^4}{x^2+1}$ has an irreducible quadratic for a denominator. However, the degree of the numerator is not smaller than the degree of the denominator. Thus, polynomial long division is the only step of PFD that is required in this case. \textbullet The antiderivative of $\protect \frac  {2}{x^5+2x3+x}$ is $$2\protect \qopname  \relax o{ln}|x|-\protect \qopname  \relax o{ln}\left | x^2+1\right |+\protect \frac  {1}{x^2+1}$$ \textbullet The PFD will produce $$ \protect \frac  {x-2}{x^3+x^2+3x-5}=\protect \frac  {-\protect \frac  {1}{8}}{x-1}+\protect \frac  {\protect \frac  {1}{8}x+\protect \frac  {11}{8}}{x^2+2x+5}$$ While the first term is easy to integrate, the second is quite tricky! To hack through it, split it as follows: $$\protect \frac  {\protect \frac  {1}{8}x+\protect \frac  {11}{8}}{x^2+2x+5}=\protect \frac  {\protect \frac  {1}{8}x+\protect \frac  {1}{8}}{x^2+2x+5}+\protect \frac  {\protect \frac  {10}{8}}{x^2+2x+5} $$ The first fraction can then be integrated via $u$-sub, while the second can be done via trig sub after completing the square on the denominator. \protect \newline  \protect \newline  
 {\noindent \protect \bf  Exercise ~1.6.3.2.} For $\protect \frac  {1}{x^4-9x^2}$, keep in mind that $x^2$ is not an irreducible quadratic factor but rather a repeated linear factor. The PFD and integration will produce $$\protect \frac  {1}{9x}+\protect \frac  {1}{54}\protect \qopname  \relax o{ln}\left |\protect \frac  {x-3}{x+3}\right |+C $$  \protect \newline  \protect \newline  
 {\noindent \protect \bf  Exercise ~1.8.1.1.} $\protect \qopname  \relax o{ln}\left (1+x\right )+C$ \protect \newline  \protect \newline  
 {\noindent \protect \bf  Exercise ~1.8.1.2.} $x+ -2\protect \sqrt  {x}+2\protect \qopname  \relax o{ln}{|\protect \sqrt  {x}+1|}+C$ \protect \newline  \protect \newline  
 {\noindent \protect \bf  Exercise ~1.8.1.3.} $\protect \qopname  \relax o{ln}\left | \protect \frac  { 2+\protect \sqrt  {3}}{\protect \sqrt  {2} +1} \right |$ \protect \newline  \protect \newline  
 {\noindent \protect \bf  Exercise ~1.8.1.4.} $\protect \frac  {1}{2}\protect \qopname  \relax o{ln}{|x-1|}-\protect \frac  {1}{2}\protect \qopname  \relax o{ln}{|x+1|} +\protect \frac  {1}{x} +C$ \protect \newline  \protect \newline  
 {\noindent \protect \bf  Exercise ~1.8.2.1.} $-\protect \frac  {\protect \qopname  \relax o{cos}^{18}{x}}{18}+ \protect \frac  {\protect \qopname  \relax o{cos}^{16}{x}}{8} - \protect \frac  {\protect \qopname  \relax o{cos}^{14}{x}}{14} + C$ \protect \newline  \protect \newline  
 {\noindent \protect \bf  Exercise ~1.8.2.2.} The antiderivative is $-\protect \frac  {1}{2}\left (\protect \qopname  \relax o{csc}(x)\protect \qopname  \relax o{cot}(x)+\protect \qopname  \relax o{ln}\left |\protect \qopname  \relax o{csc}(x)+\protect \qopname  \relax o{cot}(x)\right |\right )+C$ \protect \newline  \protect \newline  
 {\noindent \protect \bf  Exercise ~1.8.2.3.} $\protect \frac  {1}{8} \protect \qopname  \relax o{ln}\left | \protect \frac  {x-4}{x+4} \right | + C$ \protect \newline  \protect \newline  
 {\noindent \protect \bf  Exercise ~1.8.2.4.} \textbullet $ \protect \frac  {x^3}{x^3-3x^2+4} = 1 + \protect \frac  { -\protect \frac  {1}{9}}{x+1} + \protect \frac  {\protect \frac  {28}{9}}{x-2} + \protect \frac  {\protect \frac  {8}{3}}{(x-2)^2} $ \protect \newline  \textbullet $ \intop {1 + \protect \frac  { -\protect \frac  {1}{9}}{x+1} + \protect \frac  {\protect \frac  {28}{9}}{x-2} + \protect \frac  {\protect \frac  {8}{3}}{(x-2)^2} \mathop {}\protect \tmspace  -\thinmuskip {.1667em}\protect \mathtt  {d}x} = x -\protect \frac  {1}{9} \protect \qopname  \relax o{ln}{|x+1|} + \protect \frac  {28}{9} \protect \qopname  \relax o{ln}{|x-2|} - \protect \frac  {8}{3} \protect \frac  {1}{(x-2)} + C$  \protect \newline  \protect \newline  
 {\noindent \protect \bf  Exercise ~1.8.2.5.} $\protect \frac  {1}{4}\protect \qopname  \relax o{sec}^3{x}\protect \qopname  \relax o{tan}{x} +\protect \frac  {3}{8} \protect \qopname  \relax o{sec}{x}\protect \qopname  \relax o{tan}{x}+\protect \frac  {3}{8} \protect \qopname  \relax o{ln}{|\protect \qopname  \relax o{sec}{x}+\protect \qopname  \relax o{tan}{x}|}+C$ \protect \newline  \protect \newline  
 {\noindent \protect \bf  Exercise ~2.1.0.4.} The limits are 0, $1/\pi $, and -1.  \protect \newline  \protect \newline  
 {\noindent \protect \bf  Exercise ~2.1.1.3.} The limits are $1$, $-2$, and $e$.  \protect \newline  \protect \newline  
 {\noindent \protect \bf  Exercise ~2.1.1.4.} The results are 1, 0, and 1. \protect \newline  \protect \newline  
 {\noindent \protect \bf  Exercise ~2.1.2.2.} Their ratio converges to 3 (both numerically in the table, and analytically as evaluated by LHR). Since this is a nonzero constant, the two functions have the same growth order. \protect \newline  \protect \newline  
 {\noindent \protect \bf  Exercise ~2.1.2.3.} In the first and third, the ratio between $f$ and $g$ seems to grow without bound, so $f$ has larger growth order. In the second, the ratio of $f$ to $g$ seems to always be right around 2. Thus, they have the same growth order. \protect \newline  \protect \newline  
 {\noindent \protect \bf  Exercise ~2.2.1.8.} The integrals evaluate to $2\protect \sqrt  {2},\infty ,\infty ,$ and $\infty $. \protect \newline  \protect \newline  
 {\noindent \protect \bf  Exercise ~2.2.2.4.} \textbullet The area under $xe^{-x^2}$ from zero to $\infty $ is $\protect \frac  {1}{2}$. \textbullet Splitting into two integrals at $x=0$ produces one of area one-half and one of area negative one-half, so the total integral is zero. \textbullet After applying IBP with $u=x$ and $\mathop {}\protect \tmspace  -\thinmuskip {.1667em}\protect \mathtt  {d}v = xe^{-x^2}\mathop {}\protect \tmspace  -\thinmuskip {.1667em}\protect \mathtt  {d}x$, one obtains $\protect \frac  {\protect \sqrt  {\pi }}{2}$ as the area under the curve. \textbullet The area under $\protect \frac  {1}{x\protect \qopname  \relax o{ln}(x)}$ from 2 to $\infty $ is infinite. \textbullet The area under $\protect \frac  {1}{x\left (\protect \qopname  \relax o{ln}(x)\right )^2}$ from 2 to $\infty $ is $\protect \frac  {1}{\protect \qopname  \relax o{ln}(2)}$. \textbullet An improper integral is defined using a limit, and here the limit does not exist, as the area keeps going up and down by the same amount forever. \protect \newline  \protect \newline  
 {\noindent \protect \bf  Exercise ~2.3.2.1.} \textbullet The curves $y=x^3+x^2-x-1$ and $y=x^3-x^2-x+1$ intersect on the $x$ axis at -1 and 1 and have area 8/3 between them. \textbullet The area inside the unit circle but above the line $y=1/2$ is $\pi /3-\protect \sqrt  {3}/4$. \textbullet Notice graphically that the curves intersect at $x=\pm \pi /4$. The area between curves is $\pi /4-\protect \qopname  \relax o{ln}(2)$. \protect \newline  \protect \newline  
 {\noindent \protect \bf  Exercise ~2.4.1.2.} The exact arc length is $\protect \frac  {2\protect \sqrt  {5}+\protect \qopname  \relax o{ln}\left | 2+\protect \sqrt  {5}\right |}{4}$. \protect \newline  \protect \newline  
 {\noindent \protect \bf  Exercise ~2.4.3.1.} The length of the graph of the natural logarithm from (1,0) to (e,1) is $$\protect \sqrt  {e^2+1}+\protect \frac  {1}{2}\protect \qopname  \relax o{ln}\left | \protect \frac  {\protect \sqrt  {e^2+1}-1}{\protect \sqrt  {e^2+1}+1}\right |-\protect \sqrt  {2}+\protect \frac  {1}{2}\protect \qopname  \relax o{ln}\left | \protect \frac  {\protect \sqrt  {2}+1}{\protect \sqrt  {2}-1}\right | $$ which is roughly 2.003497. Also, notice that the natural exponential function is just the inverse of the natural logarithm; think about what this means regarding arc length! \protect \newline  \protect \newline  
 {\noindent \protect \bf  Exercise ~2.4.4.1.} The two-frusta approximation is $$\protect \frac  {\pi }{8}\left (\protect \sqrt  {5}+3\protect \sqrt  {13} \right )\approx 5.126 $$ The exact value of the surface area is $$\protect \frac  {\pi }{6}\left (5\protect \sqrt  {5}-1 \right )\approx 5.3304 $$ which is just slightly larger, as one would expect.  \protect \newline  \protect \newline  
 {\noindent \protect \bf  Exercise ~2.4.5.3.} Notice that if you turn the pyramid sideways, you can get the 2D side view to be almost exactly the same as we had for the cone! The volume is $V=\protect \frac  {1}{3}r^2h$. \protect \newline  \protect \newline  
 {\noindent \protect \bf  Exercise ~2.4.5.4.} The volume is $V=\protect \frac  {1}{6}abc$. \protect \newline  \protect \newline  
 {\noindent \protect \bf  Exercise ~2.4.5.6.} The circular cross section has equation $x^2+y^2=1$. If you solve for the $y$ coordinate, you'll have a function for the radius of a circular cross section at position $x$. This formula can be integrated to produce the volume $V=\protect \frac  {4}{3}\pi r^3$. \protect \newline  \protect \newline  
 {\noindent \protect \bf  Exercise ~2.4.5.8.} The parabolic bowl has volume $\pi /2$ and occupies exactly fifty percent of the cylinder it sits in! \protect \newline  \protect \newline  
 {\noindent \protect \bf  Exercise ~2.4.5.11.} The volume estimate with a single cylinder is $2\pi $. To get the heights of the six cylindrical shells, you'll need to use the fact that $x^2+y^2=1$ for every point on the boundary of the circle. With six shells, the volume estimate is $\pi \cdot \protect \frac  {\protect \sqrt  {35} + 12 \protect \sqrt  {2} + 15 \protect \sqrt  {3} + 14 \protect \sqrt  {5} + 9 \protect \sqrt  {11}}{108}\approx 1.018\pi $. The first is an overestimate, whereas the second is an underestimate. \protect \newline  \protect \newline  
 {\noindent \protect \bf  Exercise ~2.4.5.12.} The function $g(x)=\protect \sqrt  {1-x^2}$ represents just the QI $y$-coordinate. It needs to be doubled to represent the height of the shell since the each shell extends the same vertical distance into QIII. Once the integral is evaluated, it will return the exact volume $\protect \frac  {4}{3}\pi $. \protect \newline  \protect \newline  
 {\noindent \protect \bf  Exercise ~2.4.6.1.} The volume of the torus is $$V=2\pi ^2Rr^2 $$ and the surface area is $$SA=4\pi ^2Rr. $$ \protect \newline  \protect \newline  
 {\noindent \protect \bf  Exercise ~2.5.1.3.} The sine gumdrop is just a translation $\pi /2$ units to the right of the cosine gumdrop. So, we would expect the center of mass to have the same $y$-coordinate but have an $x$-coordinate that is $\pi /2$ units larger. Indeed, when computed with the moment integrals, we get $\left (\protect \mathaccentV {bar}016{x},\protect \mathaccentV {bar}016{y}\right )=\left (\pi /2,\pi /8\right )$.  \protect \newline  \protect \newline  
 {\noindent \protect \bf  Exercise ~2.5.3.1.} The diagonals have the equations $$ y=\protect \frac  {b}{a}x \protect \text  { and } y=\protect \frac  {b-2c}{a}x+c$$ with intersection point $(a/2,b/2)$, which is also the center of mass of the region. \protect \newline  \protect \newline  
 {\noindent \protect \bf  Exercise ~2.5.4.2.} The coordinates of the vertices are (0,0), (0,$c$), and ($a,b$). Two of the medians are $$y=\protect \frac  {b+c}{a}x \protect \text  { and } y=\protect \frac  {2b-c}{2a}x+\protect \frac  {c}{2}$$ and their intersection point (and center of mass of the triangle) is $\left (\protect \frac  {a}{3},\protect \frac  {b+c}{3}\right )$. \protect \newline  \protect \newline  
 {\noindent \protect \bf  Exercise ~2.5.5.1.} The center of mass is $\left ( 0, \protect \frac  {4r}{3\pi }\right )$. \protect \newline  \protect \newline  
 {\noindent \protect \bf  Exercise ~2.7.1.1.} a.~~ $e^{x}>>x^2$ b.~~ $e^{x}>>x^3$ c.~~$ e^{x}>>x^4$ d.~~The above calculations demonstrate that if you compare $p(x)$ to $e^x$ you will have n iterations of LHR resulting in $\protect \qopname  \relax m{lim}\limits _{x \to \infty }{\protect \frac  {p(x)}{e^x}}=\protect \qopname  \relax m{lim}\limits _{x \to \infty }{\protect \frac  {n!}{e^x}}=0$ Thus $e^x$ has larger growth order than any polynomial.  \protect \newline  \protect \newline  
 {\noindent \protect \bf  Exercise ~2.7.1.2.} $V=9$ \protect \newline  \protect \newline  
 {\noindent \protect \bf  Exercise ~2.7.1.3.} $V=\protect \frac  {\pi }{4} -\protect \frac  {\protect \sqrt  {2}}{6}$ \protect \newline  \protect \newline  
 {\noindent \protect \bf  Exercise ~2.7.1.4.} The area is infinite! \protect \newline  \protect \newline  
 {\noindent \protect \bf  Exercise ~2.7.1.5.} $V=\protect \frac  {4}{3} \pi r^3 $ \protect \newline  \protect \newline  
 {\noindent \protect \bf  Exercise ~2.7.1.6.} $V = \intop _{-r}^{r}{\pi (r^2-x^2) dx} = \protect \frac  {4}{3} \pi r^3 $ \protect \newline  \protect \newline  
 {\noindent \protect \bf  Exercise ~2.7.1.7.} $\protect \mathaccentV {bar}016{x}=\protect \frac  {3}{2}, \protect \mathaccentV {bar}016{y}= \protect \frac  {18}{5} $ \protect \newline  \protect \newline  
 {\noindent \protect \bf  Exercise ~2.7.2.1.} a.) $\protect \qopname  \relax m{lim}\limits _{x\rightarrow \infty }{\protect \frac  {\protect \qopname  \relax o{ln}{x}}{\protect \sqrt  {x}}}=0$ \protect \newline  b.) $\protect \qopname  \relax m{lim}\limits _{x\rightarrow 0^+}{\protect \frac  {\protect \qopname  \relax o{ln}{x}}{\protect \sqrt  {x}}}=-\infty $\protect \newline  d.) $\intop _{0}^{1}{\protect \frac  {\protect \qopname  \relax o{ln}{x}}{\protect \sqrt  {x}}dx}=-4$  \protect \newline  \protect \newline  
 {\noindent \protect \bf  Exercise ~2.7.2.2.}  $4$ \protect \newline  \protect \newline  
 {\noindent \protect \bf  Exercise ~2.7.2.3.} 1. Cylindrical shells takes line segments and revolves them about an axis parallel to those segments in order to find the volume of the solid created by that revolution. Cross sections cuts a 3D object into 2D parallel slices and does not require revolution.\protect \newline  2. You cannot use shells because it is impossible to obtain a tetrahedron via revolution since it has no rotational symmetry. So we use cross sections.\protect \newline  3. $\intop _0^2 {\protect \frac  {1}{2}(1-2x+x^2)dx}=\protect \frac  {1}{6} $  \protect \newline  \protect \newline  
 {\noindent \protect \bf  Exercise ~2.7.2.4.} $\left (\protect \frac  {2}{3},\protect \frac  {1}{3} \right ) $ \protect \newline  \protect \newline  
 {\noindent \protect \bf  Exercise ~3.3.1.4.} \textbullet $\protect \frac  {1}{n+1}$ \textbullet $n+1$ \textbullet $(n+2)(n+1)$ \textbullet $(2n+2)(2n+1)$ \protect \newline  \protect \newline  
 {\noindent \protect \bf  Exercise ~3.4.0.3.} Think about what happens if the common difference $d$ is zero and if the common ratio $r$ is 1. \protect \newline  \protect \newline  
 {\noindent \protect \bf  Exercise ~3.4.1.2.} The common ratio $r$ is what we multiply by to get from term to term. Listing out the terms $a_0, a_0r, a_0r^2, a_0r^3,\protect \cdots  $ shows that $a_0r^n$ is the explicit formula. \protect \newline  \protect \newline  
 {\noindent \protect \bf  Exercise ~3.5.1.5.} In the context of computing a limit to infinity, it is fine to replace $n!$ by $\protect \sqrt  {2\pi n} \left ( \protect \frac  {n}{e} \right )^n$. Setting up limits of ratios and testing growth order with LHR and good old algebra will then verify that the order goes $n^2,e^n,n!,n^n$. \protect \newline  \protect \newline  
 {\noindent \protect \bf  Exercise ~3.6.0.4.} The totals are 12, -6, and 15. \protect \newline  \protect \newline  
 {\noindent \protect \bf  Exercise ~3.6.0.5.} It is easiest to just expand the sums on both sides and see what the terms look like. For example, in the first case the left-hand side is $\left (ca_j+ca_{j+1}+\protect \cdots  +ca_k\right )$, whereas the right-hand side is $c\left (a_j+a_{j+1}+\protect \cdots  +a_k\right )$. These two expressions are equal, because we can factor the $c$ out of the left-hand side to produce the right-hand side. For the last two summations, think about our discussion of fencepost problems above! \protect \newline  \protect \newline  
 {\noindent \protect \bf  Exercise ~3.7.0.4.}  In Gauss's formula, the first term $a_0$ and the common difference $d$ are both 1. The number of terms is $N$. Plugging these into the Arithmetic Series Formula will produce $N(N+1)/2$. \protect \newline  \protect \newline  
 {\noindent \protect \bf  Exercise ~3.7.0.6.} The totals are 500500, 1501500, 214214, and 245. \protect \newline  \protect \newline  
 {\noindent \protect \bf  Exercise ~3.8.0.3.} The common ratio $r=10$. The first term is 1. The number of terms is 6. Putting this all together in the Geometric Series Formula produces $1\cdot \protect \frac  {1-10^6}{1-10}=\protect \frac  {-99999}{-9}=11111.$ \protect \newline  \protect \newline  
 {\noindent \protect \bf  Exercise ~3.8.0.4.} A finite sum of consecutive powers of two, starting at one, is equal to one less than the next power of two. \protect \newline  \protect \newline  
 {\noindent \protect \bf  Exercise ~3.8.0.7.} Sure! If you further factor $A^2-B^2$ via difference of two squares and further factor $A^3+A^2B+AB^2+B^3$ via grouping, you will end up with the same factorizations. \protect \newline  \protect \newline  
 {\noindent \protect \bf  Exercise ~3.8.1.3.} $\protect \$5 \protect \text  { billion }\cdot \protect \frac  {1-0.8^{13}}{1-0.8}\approx \protect \$23.6 \protect \text  { billion}$ \protect \newline  \protect \newline  
 {\noindent \protect \bf  Exercise ~3.8.1.4.} Try to notice how the summations relate to the very next Fibonacci number, the first one not being summed. \protect \newline  \protect \newline  
 {\noindent \protect \bf  Exercise ~3.9.1.3.} In this series, there are $N+1$ terms, the first of which is 1 and the last of which is $2N+1$. Plugging this information into the Arithmetic Series Formula will produce the desired result.  \protect \newline  \protect \newline  
 {\noindent \protect \bf  Exercise ~3.9.1.4.} The partial sum $A_N$ represents the total number of push-ups you've done so far in your push-up routine, up to and including day $N$. \protect \newline  \protect \newline  
 {\noindent \protect \bf  Exercise ~3.9.3.1.} The corresponding partial sums are as follows: \textbullet $A_N=10\left (1-1/2^{N+1}\right )$ \textbullet $A_N=3/4\left (1-1/9^{N+1}\right )$ \textbullet $A_N=\left (10-N\right )\left (N+1\right )/2$ \textbullet $A_N=(N+1)^3$ \textbullet $A_N=1,0,1,0,1,0,\protect \ldots  =\left (1-\left (-1\right )^{N+1}\right )/2$ \textbullet $A_N=1$  \protect \newline  \protect \newline  
 {\noindent \protect \bf  Exercise ~3.10.0.2.} Think about an integral of the form $\intop _{x=0}^{x=\infty }f(x)\mathop {}\protect \tmspace  -\thinmuskip {.1667em}\protect \mathtt  {d}x $. How does one handle that infinity in the bounds? \protect \newline  \protect \newline  
 {\noindent \protect \bf  Exercise ~3.10.1.1.} The sequence is $a_n=\protect \frac  {1}{2^{n+1}}$. Since this is a geometric sequence, the finite geometric series formula can be applied to then find the sequence of partial sums $A_N$. \protect \newline  \protect \newline  
 {\noindent \protect \bf  Exercise ~3.10.1.2.} It ends up one-third of a meter forward from where it started. \protect \newline  \protect \newline  
 {\noindent \protect \bf  Exercise ~3.10.2.4.} Yes, the series is geometric with initial term $\protect \frac  {3^5}{2^{11}}$ and common ratio $3/4$. The infinite series totals to $ \protect \frac  {3^5}{2^9}$. \protect \newline  \protect \newline  
 {\noindent \protect \bf  Exercise ~3.10.2.5.} Think about what the value of $r$ would be for that series. What restrictions did we have on $r$ in the statement of the infinite geometric series formula? \protect \newline  \protect \newline  
 {\noindent \protect \bf  Exercise ~3.10.2.6.} $1+2\protect \frac  {5}{8}+2\left (\protect \frac  {5}{8}\right )^2+2\left (\protect \frac  {5}{8}\right )^3+\protect \cdots  =1+2\protect \frac  {5/8}{1-5/8}=1+2\protect \frac  {5/8}{3/8}=13/3=4.\protect \overline  {3}$ meters. \protect \newline  \protect \newline  
 {\noindent \protect \bf  Exercise ~3.10.2.7.} The partial sums are $A_N=2(N+1)$. The infinite series is the limit of $A_N$ as $N$ goes to infinity, which here is clearly again infinity. Thus, the infinite series diverges. \protect \newline  \protect \newline  
 {\noindent \protect \bf  Exercise ~3.10.2.8.} The partial sums are $$A_N=\protect \frac  {N+1}{N+3}$$ for an infinite sum of 1. \protect \newline  \protect \newline  
 {\noindent \protect \bf  Exercise ~3.10.2.9.} The infinite series $\Sigma _{n=0}^{\infty }a_n$ are \textbullet Divergent \textbullet Divergent \textbullet 3 \textbullet $\protect \frac  {2}{3}$ \textbullet Divergent \textbullet $\protect \frac  {9}{4}$ \textbullet Divergent \textbullet 1 \protect \newline  \protect \newline  
 {\noindent \protect \bf  Exercise ~3.11.0.7.} It is absolutely convergent, since the series of corresponding positive terms is $0.1+0.02+0.004+0.0008+0.00016+\protect \cdots  $ which converges to one-eighth.  \protect \newline  \protect \newline  
 {\noindent \protect \bf  Exercise ~3.12.1.4.} The first one-half would come from the first term itself. But since the total is one, it means the terms $\protect \frac  {1}{4}+\protect \frac  {1}{8}+\protect \frac  {1}{16}+\protect \cdots  $ must themselves total to be the other one-half. Thus, we can try to group terms to form batches that total to one-half, but the second batch uses up all infinitely many remaining terms!  \protect \newline  \protect \newline  
 {\noindent \protect \bf  Exercise ~3.12.1.7.} The first two summands have limits of $\protect \sqrt  {2}$ and 1, respectively. Since these limits are nonzero, the series has no hope of converging and thus diverges. The third summand does approach zero as $n$ goes to infinity, so it gives no information. \protect \newline  \protect \newline  
 {\noindent \protect \bf  Exercise ~3.12.2.2.} \textbullet The summation $\Sigma _{n=1}^\infty \protect \frac  {1}{n^2}$ has no common ratio $r$ and thus is not a geometric series. For example, the first three terms are $1,1/4,$ and $1/9$. Thus, the first two ratios between consecutive terms are $1/4$ and $4/9$, which are not equal. \textbullet The given geometric series has common ratio $r=-1/3$. After taking the absolute value of each term, it becomes the series $18+6+2+\protect \frac  {2}{3}+\protect \frac  {2}{9}+\protect \cdots  $ which still converges as it now has common ratio $r=1/3$. \textbullet It is not possible to build a conditionally convergent geometric series. If we are given a convergent geometric series, then the common ratio $r$ satisfies $|r|<1$. Taking the absolute value of each term in the series might flip the sign on $r$, but it will not change the magnitude. Thus, any convergent geometric series must converge absolutely. \protect \newline  \protect \newline  
 {\noindent \protect \bf  Exercise ~3.12.4.2.} For $p>1$ or $p<1$, one can repeat the corresponding calculations from Example \protect \ref  {IntTest}.\protect \ref  {UsingIntTest}. If $p=1$, the series is the harmonic series, which diverges. \protect \newline  \protect \newline  
 {\noindent \protect \bf  Exercise ~3.12.5.3.} Taking term-by-term absolute values produces the series $1+\protect \frac  {1}{2}+\protect \frac  {1}{4}+\protect \frac  {1}{8}+\protect \cdots  =2$. Since it totals to a finite value, the original series converges absolutely. \protect \newline  \protect \newline  
 {\noindent \protect \bf  Exercise ~3.12.5.4.} \textbullet Since $\protect \qopname  \relax m{lim}_{n\to \infty }\left (-\protect \frac  {1}{2}\right )^n=0$, the No Hope Test gives no information. \textbullet The Integral Test does not apply since the terms are not positive and decreasing. In this case, it is actually even worse than that, as the function $\left (-\protect \frac  {1}{2}\right )^x$ is undefined for all half-integer values of $x$. \textbullet The summand is not of the form $1/n^p$, so the very narrow $p$-Test does not apply. \protect \newline  \protect \newline  
 {\noindent \protect \bf  Exercise ~3.12.5.5.} The first two and last converge by AST. It does not apply to the third. \protect \newline  \protect \newline  
 {\noindent \protect \bf  Exercise ~3.12.5.8.} After three reversals, the bug is within $\protect \frac  {1}{32}$ of its final location. The bug would have to reverse course nine times to be guaranteed by the Alternating Series Error Bound to be within one one-thousandth of its final location.  \protect \newline  \protect \newline  
 {\noindent \protect \bf  Exercise ~3.12.6.3.} The first converges by comparison to $\Sigma \protect \frac  {1}{n^2}$. The second diverges by comparison to $\Sigma \protect \frac  {1}{n^{-1/2}}$. \protect \newline  \protect \newline  
 {\noindent \protect \bf  Exercise ~3.12.6.6.} Use the comparison function $\protect \frac  {1}{n}$ to show the series diverges. \protect \newline  \protect \newline  
 {\noindent \protect \bf  Exercise ~3.12.7.4.} \textbullet Converges, ratio 0. \textbullet No info, ratio 1. \textbullet Converges, ratio 0. \textbullet Diverges, ratio 2. \protect \newline  \protect \newline  
 {\noindent \protect \bf  Exercise ~3.12.8.3.} These series are convergent by DCT against $\protect \frac  {1}{n^2}$, divergent by DCT against $\protect \frac  {1}{n}$, and convergent by DCT against $\protect \frac  {1}{n^2}$. \protect \newline  \protect \newline  
 {\noindent \protect \bf  Exercise ~3.12.9.1.} \textbullet Divergent by NHT or Integral Test. \textbullet Divergent by Integral Test or LCT against $\protect \frac  {1}{n}$. \textbullet Convergent by AST, but only conditionally since the absolute value is the previous summand whose series diverged. \textbullet Absolutely convergent since taking term-by-term absolute value produces a convergent series (which can be shown convergent via LCT with $\protect \frac  {1}{n^3}$). \protect \newline  \protect \newline  
 {\noindent \protect \bf  Exercise ~3.12.10.1.} The black region is one-third of the total square and thus must total to one-third. The infinite series for the black square areas is $\protect \frac  {1}{4}+\protect \frac  {1}{16}+\protect \frac  {1}{32}+\protect \frac  {1}{64}+\protect \cdots  $. NHT and AST give no information here, but all the rest of the tests work to determine convergence! Use $\protect \frac  {1}{n^2}$ as a comparison function for LCT. \protect \newline  \protect \newline  
 {\noindent \protect \bf  Exercise ~3.14.1.1.} a.~~ For $\epsilon = 0.1, n=5,~~~\epsilon = 0.01, n=05,~~~\epsilon = 0.001, n=500$ b.~~ $\protect \frac  {1}{2\epsilon }=N $ c.~~Let $ \epsilon >0$ also, let $N = \protect \frac  {1}{2n}$ and let $n \in \protect \mathbb  {N}$ Assume $n>N$ We wish to show that under these circumstances, the distance from $a_n= \protect \frac  {1}{2n}$ to $L=0$ will be less than $\epsilon $. $\left | a_n-L \right | =\left |\protect \frac  {1}{2n}-0 \right | = \protect \frac  {1}{2n} < \protect \frac  {1}{2N}$ since $N<n$ by our assumptions $ \protect \frac  {1}{2N }=\protect \frac  {1}{2\protect \frac  {1}{2 \epsilon }} = \protect \frac  {1}{\epsilon } = \epsilon $ Thus the terms will be within $\epsilon $ or 0 past the index $\protect \frac  {1}{2 \epsilon }$ , no matter how small $\epsilon $ is chosen. Therefore, $\protect \qopname  \relax m{lim}\limits _{n \to \infty }{\protect \frac  {1}{2n}} = 0$  \protect \newline  \protect \newline  
 {\noindent \protect \bf  Exercise ~3.14.1.2.} $3,375,000$ \protect \newline  \protect \newline  
 {\noindent \protect \bf  Exercise ~3.14.1.3.} a.)~~6 Slices,~b.)~~4.5 Slices,~c.)$\left (\protect \frac  {3}{4} \right )^{20} 8$ Slices,~d.)~~None because as time goes to $\infty $ the number of slices left goes to 0, since $ \left (\protect \frac  {3}{4} \right )^{n} 8$ represents the number of slices left after n minutes and $\protect \qopname  \relax m{lim}\limits _{n \to \infty }{ \left (\protect \frac  {3}{4} \right )^{n} 8} = 0$ slices. \protect \newline  \protect \newline  
 {\noindent \protect \bf  Exercise ~3.14.1.4.} a.) Converges Absolutely by the Ratio Test.\protect \newline  b.) Diverges by the No Hope Test.\protect \newline  c.) Diverges by the No Hope Test.\protect \newline  d.) Converges by the Limit Comparison Test and the p-test.\protect \newline  e.) Diverges by the No Hope Test \protect \newline  f.) Converges Absolutely by the Ratio Test.\protect \newline  g.)Converges Absolutely by the Integral Test or the Ratio Test. \protect \newline  \protect \newline  
 {\noindent \protect \bf  Exercise ~3.14.2.1.} Notice $a_n$ is a geometric sequence with initial term $a_0=2$ and the ratio $r=\protect \frac  {1}{2}$. We can then use the infinite geometric series formula $\Sigma _{n=0}^\infty {a_n}= \protect \frac  {a}{1-r}$ so we have $\Sigma _{n=0}^\infty {2\cdot \left (\protect \frac  {1}{2}\right )^n} = 2 \cdot \protect \frac  {1}{1-\protect \frac  {1}{2}} = 2 \protect \frac  {1}{\protect \frac  {1}{2}} = 2 \cdot 2 = 4.$ Thus, the sequence of partial sums converges to $4$. \protect \newline  \protect \newline  
 {\noindent \protect \bf  Exercise ~3.14.2.2.} a.) $\protect \qopname  \relax m{lim}\limits _{n \to \infty }{a_n} = L \DOTSB \Leftarrow \protect \joinrel  \Rightarrow $ For all $\epsilon >0$ there exists N such that for all $n>N, |a_n-L| < \epsilon $ \protect \newline  b.) $\protect \frac  {1}{3}$\protect \newline  c.) Let $\epsilon >0$ then choose $N=\protect \frac  {1}{9n}-\protect \frac  {1}{3}$ Then choose $n \in \protect \mathbb  {N}$ with $n>N$. We now show that any $a_n$ for such n is no more than $\epsilon $ away from $\protect \frac  {1}{3}$. \protect \newline  $\left | \protect \frac  {n}{3n+1} - \protect \frac  {1}{3} \right |=\left | \protect \frac  {3n}{3(3n+1)} - \protect \frac  {3n+1}{3(3n+1)} \right |=\left | \protect \frac  {-1}{3(3n+1)} \right |= \protect \frac  {1}{3(3n+1)}<\protect \frac  {1}{3(3N+1)}$ \protect \newline  note here we made the denominator smaller by introducing N for n\protect \newline  $\protect \frac  {1}{3(3N+1)}=\protect \frac  {1}{3\left (3\left (\protect \frac  {1}{9 \epsilon }-\protect \frac  {1}{3} \right )+1\right )}=\protect \frac  {1}{3\left (\left (\protect \frac  {1}{3 \epsilon }-1 \right )+1\right )}=\protect \frac  {1}{3\left (\protect \frac  {1}{3 \epsilon } \right )}=\protect \frac  {1}{\protect \frac  {1}{\epsilon }}=\epsilon $  \protect \newline  \protect \newline  
 {\noindent \protect \bf  Exercise ~3.14.2.3.} a.)$\protect \qopname  \relax m{lim}\limits _{n \to \infty }{(-1)^n \protect \frac  {1}{n!}} = 0$ So the no hope test gives no information since it requires the limit to be not equal to zero. \protect \newline  b.)It converges absolutely by the Alternating Series Test.\protect \newline  c.)It converges absolutely by the Ratio Test  \protect \newline  \protect \newline  
 {\noindent \protect \bf  Exercise ~3.14.2.4.} a.)$a_0 = 0, a_1=1, a_2=1+3\cdot 2^2-3\cdot 2+1 = 8, a_3=8+3\cdot 3^2-3\cdot 3+1 = 27, a_4=1+3\cdot 4^2-3\cdot 4+1 = 64 $\protect \newline  b.)$a_n = n^3$\protect \newline  c.)It diverges by the No Hope Test  \protect \newline  \protect \newline  
 {\noindent \protect \bf  Exercise ~4.1.0.4.} Written in sigma notation, the power series is $\protect \qopname  \relax o{sin}(x)=\Sigma _{n=0}^\infty (-1)^{n}\protect \frac  {1}{\left (2n+1\right )!}x^{2n+1}$. \protect \newline  \protect \newline  
 {\noindent \protect \bf  Exercise ~4.1.0.5.} Written in sigma notation, the power series is $e^x=\Sigma _{n=0}^\infty \protect \frac  {1}{n!}x^{n}$. \protect \newline  \protect \newline  
 {\noindent \protect \bf  Exercise ~4.1.0.6.} The power series is $\protect \frac  {1}{1-x}=\Sigma _{n=0}^\infty x^n$. It is a geometric series with initial term 1 and common ratio $x$. \protect \newline  \protect \newline  
 {\noindent \protect \bf  Exercise ~4.1.0.8.} When we try to plug in $x=0$ to find $a_0$, we get $\protect \qopname  \relax o{ln}(0)$ which is not a real number. \protect \newline  \protect \newline  
 {\noindent \protect \bf  Exercise ~4.1.0.9.} The power series centered at one for the natural log is $\protect \qopname  \relax o{ln}(x)=\Sigma _{n=1}^\infty \protect \frac  {(-1)^{n+1}}{n}(x-1)^n$. \protect \newline  \protect \newline  
 {\noindent \protect \bf  Exercise ~4.4.1.3.} If we substitute $x-1$ for $x$ in the power series for sine, we get $\protect \qopname  \relax o{sin}(x-1)=\Sigma _{n=0}^\infty (-1)^{n}\protect \frac  {1}{\left (2n+1\right )!}(x-1)^{2n+1}$. Likewise, substituting $2x$ for $x$ in the power series for sine produces $\protect \qopname  \relax o{sin}(2x)=\Sigma _{n=0}^\infty (-1)^{n}\protect \frac  {1}{\left (2n+1\right )!}(2x)^{2n+1}=\Sigma _{n=0}^\infty (-1)^{n}\protect \frac  {2^{2n+1}}{\left (2n+1\right )!}x^{2n+1}$. \protect \newline  \protect \newline  
 {\noindent \protect \bf  Exercise ~4.4.2.3.} The power series $\protect \frac  {1}{x^2-x-12}=\Sigma _{n=0}^\infty \left (\protect \frac  {-1}{21\cdot (-3)^n}-\protect \frac  {1}{28\cdot 4^n}\right )x^n$ has IOC (-3,3). The power series $\protect \frac  {1}{x}=\Sigma _{n=0}^\infty \protect \frac  {(-1)^n}{5^{n+1}}(x-5)^n$ has IOC (0,10). It turns out these two examples generalize; for rational functions, the IOC will always just be the interval that goes from the center of the series outwards until it bumps into the nearest vertical asymptote! \protect \newline  \protect \newline  
 {\noindent \protect \bf  Exercise ~4.4.4.1.} Antidifferentiate the geometric series to sneak up on $\protect \qopname  \relax o{ln}(1-x)$. \protect \newline  \protect \newline  
 {\noindent \protect \bf  Exercise ~4.4.4.2.} Each method should lead to $$\protect \frac  {1}{(1-x)^2}=1+2x+3x^2+4x^3+5x^4+\protect \cdots  $$ \protect \newline  \protect \newline  
 {\noindent \protect \bf  Exercise ~4.6.0.2.} If $n=1$, we have the following degree one power series centered at $a=4$: $$f(x)=\protect \sqrt  {x}\approx 2+\protect \frac  {1}{4}(x-4)$$ Since $n=1$, we need the second derivative. We compute $\left |f''(x)\right |=\protect \frac  {1}{4x^{3/2}}$, which on the interval [4,4.1] has its maximum at $x=4$. Thus, $M=1/32$, which provides an error bound of $$\protect \frac  {\protect \frac  {1}{32}\cdot |4-4.1|^2}{2!}=\protect \frac  {1}{6400}$$ Thus the error is definitely less than one thousandth, but not necessarily less than one ten thousandth. So for the approximation $$\protect \sqrt  {4.1}\approx 2+\protect \frac  {1}{4}(4.1-4)=2.025$$ we can guarantee three digits past the decimal are correct but not necessarily the fourth. That is, 2.025 is certainly the correct decimal expansion for $\protect \sqrt  {4.1}$ rounded to the thousandths place. However, the digit 0 we implicitly have in the ten-thousandths place may or may not be correct. This process can be repeated for the other $n$ values of two and three. \protect \newline  \protect \newline  
 {\noindent \protect \bf  Exercise ~4.7.2.2.} Substitute $-x$ into the power series for cosine and simplify to demonstrate that $\protect \qopname  \relax o{cos}(-x)=\protect \qopname  \relax o{cos}(x)$, and similarly for sine. \protect \newline  \protect \newline  
 {\noindent \protect \bf  Exercise ~4.8.0.6.} Choices of comparison functions and conclusions are as follows: \textbullet $\protect \frac  {1}{n}$, diverges \textbullet $\protect \frac  {1}{n^2}$, converges \textbullet $\protect \frac  {1}{n}$, diverges.  \protect \newline  \protect \newline  
 {\noindent \protect \bf  Exercise ~4.8.0.7.} If the expression $\protect \frac  {f(x)}{g(x)}$ is indeterminate of the form $\protect \frac  {\infty }{\infty }$, we can trade it out for the expression $\protect \frac  {1/g(x)}{1/f(x)}$, which brings us back to the $\protect \frac  {0}{0}$ case. \protect \newline  \protect \newline  
 {\noindent \protect \bf  Exercise ~4.9.0.7.} The ratio between consecutive terms is $\protect \frac  {1+\protect \sqrt  {5}}{2}$, the Golden Ratio. \protect \newline  \protect \newline  
 {\noindent \protect \bf  Exercise ~4.9.0.8.} The IOC is $\left (\protect \frac  {1-\protect \sqrt  {5}}{2},\protect \frac  {\protect \sqrt  {5}-1}{2}\right )$. This is the interval that proceeds symmetrically left and right from the origin as far as it can until it runs into the nearest vertical asymptote of $\protect \frac  {x}{1-x-x^2}$. \protect \newline  \protect \newline  
 {\noindent \protect \bf  Exercise ~4.10.0.5.} \textbullet $\protect \frac  {1}{2-x}$ \textbullet $\protect \sqrt  {x}$ \textbullet $-\protect \qopname  \relax o{ln}(2-x)$ \textbullet $\protect \frac  {1}{x} $ \textbullet $\protect \frac  {e^{x-1}-1}{x-1}$  \protect \newline  \protect \newline  
 {\noindent \protect \bf  Exercise ~4.13.2.1.} $1.4$ \protect \newline  \protect \newline  
 {\noindent \protect \bf  Exercise ~4.13.2.2.} a.) $\protect \sqrt  {e}$,~~b.) $\protect \sqrt  {2}$,~~c.) $\protect \frac  {20}{99}$,~~d.) $5e^5$ \protect \newline  \protect \newline  
 {\noindent \protect \bf  Exercise ~4.13.2.3.} a.~~ 2 b.~~ $\protect \qopname  \relax m{lim}\limits _{x\rightarrow 0} \protect \frac  {x^2}{1+\protect \frac  {1}{2}x^2} =2$  \protect \newline  \protect \newline  
 {\noindent \protect \bf  Exercise ~4.13.2.4.} a.)~~$y^2-x^2=1 \rightarrow y^2=\pm \protect \sqrt  {1+x^2}$,~~b.)~~$\protect \qopname  \relax m{lim}\limits _{x\rightarrow \infty } \protect \frac  {f(x)}{x}=1$ and $\protect \qopname  \relax m{lim}\limits _{x\rightarrow \infty } \protect \frac  {f(x)}{-x}=-1$~~c.)~~$1+1/2 x^2$ a Parabola with vertex at (0,1) and squished by 1/2. \protect \newline  \protect \newline  
 {\noindent \protect \bf  Exercise ~4.13.2.5.} a.~~ $1-\protect \frac  {1}{2} \protect \qopname  \relax o{ln}{(1+x^2)} $ b.~~ [-1,1]  \protect \newline  \protect \newline  
 {\noindent \protect \bf  Exercise ~6.1.0.2.} This parametric curve is the line $y=\protect \frac  {3}{2}x+1$. \protect \newline  \protect \newline  
 {\noindent \protect \bf  Exercise ~6.1.0.3.} The two curves are the same points in the plane. Both start at the point (1,0) at time $t=0$, but $C_1$ then proceeds counter-clockwise while $C_2$ proceeds clockwise. \protect \newline  \protect \newline  
 {\noindent \protect \bf  Exercise ~6.3.0.4.} The arc length is $$\protect \frac  {6\protect \sqrt  {146}+\protect \qopname  \relax o{ln}\left (\protect \sqrt  {73}+6\protect \sqrt  {2}\right )}{6}\approx 12.55.$$ Also, to handle the absolute value, just find the arc length on the interval [0,2] where you can ignore the absolute value and then apply symmetry. \protect \newline  \protect \newline  
 {\noindent \protect \bf  Exercise ~6.3.0.5.} The arc length is $\protect \sqrt  {2}\left (e^{2\pi }-1\right )$. \protect \newline  \protect \newline  
 {\noindent \protect \bf  Exercise ~6.5.2.7.} Yes, it is in fact a circle with cartesian center $\left (0,1/2\right )$ and radius 1/2. This can be verified by demonstrating the polar equation converts to the cartesian equation $$x^2+\left (y-\protect \frac  {1}{2}\right )^2=\left (\protect \frac  {1}{2}\right )^2. $$ \protect \newline  \protect \newline  
 {\noindent \protect \bf  Exercise ~6.6.0.2.} The derivative is a constant; thus the graph is a straight line! \protect \newline  \protect \newline  
 {\noindent \protect \bf  Exercise ~6.7.0.4.} The area between the curves is $\protect \frac  {\pi }{8}$. \protect \newline  \protect \newline  
 {\noindent \protect \bf  Exercise ~6.7.0.5.} The area inside the inner loop of $r(\theta )=\protect \frac  {1}{2}+\protect \qopname  \relax o{cos}(\theta )$ is $\protect \frac  {\pi }{4}-\protect \frac  {3\protect \sqrt  {3}}{8}$. \protect \newline  \protect \newline  
 {\noindent \protect \bf  Exercise ~6.10.1.1.} a.~~ $r(0) = 4,~~r(\pi /6) = 8/\protect \sqrt  {3},~~r(\pi /4) =8/\protect \sqrt  {2}=4\protect \sqrt  {2},~~r(\pi /3)= 8 $ b.~~ $r=4\protect \qopname  \relax o{sec}{\theta } \DOTSB \protect \Relbar \protect \joinrel \Rightarrow r \protect \qopname  \relax o{cos}{\theta } = 4 \DOTSB \protect \Relbar \protect \joinrel \Rightarrow x=4$ is a vertical line. c.~~ d.~~It is an isosceles triangle with hypotenuse $4\protect \sqrt  {2}$ and sides 4 $A=8$ e.~~$A=8$ They are the same.  \protect \newline  \protect \newline  
 {\noindent \protect \bf  Exercise ~6.10.1.2.} a.)~~ It is a line segment that lies on the line $\protect \frac  {x-1}{4} = t = \protect \frac  {y}{6} \leftrightarrow y = \protect \frac  {3}{2} x - \protect \frac  {3}{2} $ between $(-1,0)$ and $(7,12)$ \protect \newline  b.)~~$\protect \frac  {3}{2}$ \protect \newline  c.)~~ $4\protect \sqrt  {13}$ \protect \newline  \protect \newline  
 {\noindent \protect \bf  Exercise ~6.10.2.1.} $\protect \qopname  \relax o{cosh}{t} = 1+\protect \frac  {t^2}{2!} + \protect \frac  {t^4}{4!}+\protect \frac  {t^6}{6!} + \protect \cdots  $\protect \newline  $\protect \qopname  \relax o{sinh}{t} = t + \protect \frac  {t^3}{3!} + \protect \frac  {t^5}{5!}+ \protect \cdots  $ \protect \newline  a.~~$\protect \frac  {d ~\protect \qopname  \relax o{sinh}(t)}{dt} = 1+\protect \frac  {t^2}{2!}+\protect \frac  {t^4}{4!} + \protect \cdots  =\protect \qopname  \relax o{cosh}(t)$ \protect \newline  b.~~$\protect \frac  {d ~\protect \qopname  \relax o{cosh}(t)}{dt} = t+\protect \frac  {t^3}{3!}+\protect \frac  {t^5}{5!} + \protect \cdots  =\protect \qopname  \relax o{sinh}(t)$ \protect \newline  c.~~$ \protect \qopname  \relax o{cosh}^2{t}- \protect \qopname  \relax o{sinh}^2{t} = \left (1+\protect \frac  {t^2}{2!} + \protect \frac  {t^4}{4!}+\protect \frac  {t^6}{6!} + \protect \cdots  \right )^2 - \left (t + \protect \frac  {t^3}{3!} + \protect \frac  {t^5}{5!}+ \protect \cdots  \right )^2 = 1 $ \protect \newline  d.~~$\protect \frac  {1}{0}$ which is a vertical line \protect \newline  \protect \newline  
 {\noindent \protect \bf  Exercise ~6.10.2.2.}  b.) $\protect \frac  {\pi }{8}$\protect \newline   \protect \newline  \protect \newline  
 {\noindent \protect \bf  Exercise ~7.2.0.4.} Any solution to $\protect \frac  {dy}{dx}=xy+x$ can be written as $y=Ce^{\protect \frac  {x^2}{2}}-1$ for some real number $C$. The second DE with initial condition has the solution $$\protect \frac  {1}{2}e^{-y}\left (\protect \qopname  \relax o{sin}(y)-\protect \qopname  \relax o{cos}(y)\right )=-e^{-x}\left (3+2x+x^2\right )+\protect \frac  {5}{2}. $$ \protect \newline  \protect \newline  
 {\noindent \protect \bf  Exercise ~7.6.1.1.} Linear combinations of hyperbolic sine and hyperbolic cosine functions are the only functions that equal their own second derivatives.  \protect \newline  \protect \newline  
 {\noindent \protect \bf  Exercise ~7.6.2.1.} a.~~ $y=3 +x + (a_0-3)e^x$ b.~~ If $y =3 +x + (a_0-3)e^x$ then $\protect \frac  {dy}{dx} = 1 +(a_0-3)e^x$ but $y-x-2 = 3+x+(a_0-3)e^x -x-2 = 1 + (a_0-3)e^x$ So they match.  \protect \newline  \protect \newline  
 {\noindent \protect \bf  Exercise ~7.6.2.2.} a.)~~ The right-hand side $2y+x$ does not factor into a function of $y$ times a function of $x$ so there can be no separation of variables.\protect \newline  b.)~~$y = -\protect \frac  {1}{4} -\protect \frac  {1}{2}x +Ce^{2x}$ \protect \newline   \protect \newline  \protect \newline  
 {\noindent \protect \bf  Exercise ~8.2.3.2.} \textbullet $\protect \qopname  \relax o{ln}(2)=\protect \qopname  \relax o{ln}(2)+0i$ \textbullet $\protect \qopname  \relax o{ln}(-2)=\protect \qopname  \relax o{ln}(2)+\pi i$ \textbullet $\protect \qopname  \relax o{ln}(1+i)=\protect \qopname  \relax o{ln}\left (\protect \sqrt  {2}\right )+i\protect \frac  {\pi }{4}$ \textbullet $\protect \qopname  \relax o{ln}(3-4i)=\protect \qopname  \relax o{ln}(5)+i\protect \qopname  \relax o{arctan}\left (-\protect \frac  {3}{4}\right )$ \protect \newline  \protect \newline  
 {\noindent \protect \bf  Exercise ~8.2.4.2.} The number $(1+i)^{1+i}$ can be written in complex cartesian form as $$\left (e^{\protect \qopname  \relax o{ln}\left (\protect \sqrt  {2}\right )-\protect \frac  {\pi }{4}}\protect \qopname  \relax o{cos}\left (\protect \qopname  \relax o{ln}\left (\protect \sqrt  {2}\right )+\protect \frac  {\pi }{4} \right )\right )+i\left (e^{\protect \qopname  \relax o{ln}\left (\protect \sqrt  {2}\right )-\protect \frac  {\pi }{4}}\protect \qopname  \relax o{sin}\left (\protect \qopname  \relax o{ln}\left (\protect \sqrt  {2}\right )+\protect \frac  {\pi }{4} \right )\right ). $$ \protect \newline  \protect \newline  
 {\noindent \protect \bf  Exercise ~8.3.0.1.} The PFD over the complex numbers is $$\protect \frac  {4-2x^2}{x^3+4x}=\protect \frac  {1}{x}-\protect \frac  {\protect \frac  {3}{2}}{x+2i}-\protect \frac  {\protect \frac  {3}{2}}{x-2i}. $$ \protect \newline  \protect \newline  
 {\noindent \protect \bf  Exercise ~8.3.0.2.} In taking the limit, the logarithmic term will approach zero. Thus $C=\pi /2$. \protect \newline  \protect \newline  
 {\noindent \protect \bf  Exercise ~8.5.1.1.} $z=e^{i (-\pi /4)}, e^{i(3 \pi /4)} = \protect \frac  {\protect \sqrt  {2}}{2} - i \protect \frac  {\protect \sqrt  {2}}{2},-\protect \frac  {\protect \sqrt  {2}}{2} + i \protect \frac  {\protect \sqrt  {2}}{2}$  \protect \newline  \protect \newline  
 {\noindent \protect \bf  Exercise ~8.5.2.1.} a.)~~$-i$ \protect \newline  b.)~~$e^{-\pi }$ \protect \newline  c.)~~$-1$ \protect \newline  d.)~~$\protect \frac  {\protect \sqrt  {3}}{2} + \protect \frac  {i}{2}, -\protect \frac  {\protect \sqrt  {3}}{2} + \protect \frac  {i}{2}, -i$ \protect \newline  e.)~~$\protect \qopname  \relax o{ln}{7} + i \protect \qopname  \relax o{arctan}{\protect \sqrt  {3}/12}$  \protect \newline  \protect \newline  
 {\noindent \protect \bf  Exercise ~8.5.2.2.} b.~~$r\protect \qopname  \relax o{cos}(\theta )+ri \protect \qopname  \relax o{sin}(\theta ) = re^{i \theta }$ This shows that $r$ and $\theta $ represent a radius and an angle.  \protect \newline  \protect \newline  
