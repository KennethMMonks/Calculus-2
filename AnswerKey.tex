\par 
 {\noindent \protect \bf  Exercise ~1.1.0.2.} The antiderivative is $\protect \frac  {1}{3}\protect \qopname  \relax o{sin}^3(x)+C$. \protect \newline  \protect \newline  
 {\noindent \protect \bf  Exercise ~1.1.1.2.} Since seven is odd, when we pulled out one factor of sine, we ended up with the sixth power of sine remaining. Since six is even, we were able to express it as a power of a perfect square of sine, which in turn let us rewrite as cosines using the Pythagorean identity. \protect \newline  \protect \newline  
 {\noindent \protect \bf  Exercise ~1.1.1.3.} The first antiderivative is $-\protect \frac  {1}{3}\protect \qopname  \relax o{cos}^3(x)+\protect \frac  {2}{5}\protect \qopname  \relax o{cos}^5(x)-\protect \frac  {1}{7}\protect \qopname  \relax o{cos}^7(x)+C$. For the second, rewrite as $(1-\protect \qopname  \relax o{sin}^2(x))^4\protect \qopname  \relax o{cos}(x)$ and proceed by letting $u=\protect \qopname  \relax o{sin}(x)$. \protect \newline  \protect \newline  
 {\noindent \protect \bf  Exercise ~1.1.1.4.} Often when trying to show that two antiderivatives are compatible, it is easiest to verify that their difference is a constant.  \protect \newline  \protect \newline  
 {\noindent \protect \bf  Exercise ~1.1.1.4.} The substitution $u=\protect \qopname  \relax o{sin}(x)$ is much cleaner since the other will involve having to expand a binomial to the fifth power. The antiderivative is $\protect \frac  {1}{12}\protect \qopname  \relax o{sin}^{12}(x)-\protect \frac  {1}{14}\protect \qopname  \relax o{sin}^{14}(x)+C$. \protect \newline  \protect \newline  
 {\noindent \protect \bf  Exercise ~1.1.2.1.} The exponent on sine is zero, which is indeed even. Thus both exponents are even in this case. \protect \newline  \protect \newline  
 {\noindent \protect \bf  Exercise ~1.1.2.3.} When all like terms are combined and the one-eighth is distributed, the result is $\protect \frac  {5}{16}x+\protect \frac  {1}{4}\protect \qopname  \relax o{sin}(2x)-\protect \frac  {1}{48}\protect \qopname  \relax o{sin}^3(2x)+\protect \frac  {3}{64}\protect \qopname  \relax o{sin}(4x)+C$. \protect \newline  \protect \newline  
 {\noindent \protect \bf  Exercise ~1.1.2.4.} The antiderivative to $\protect \qopname  \relax o{cos}^6(x)$ came out to \par $$\protect \frac  {5}{16}x+\protect \frac  {1}{4}\protect \qopname  \relax o{sin}(2x)-\protect \frac  {1}{48}\protect \qopname  \relax o{sin}^3(2x)+\protect \frac  {3}{64}\protect \qopname  \relax o{sin}(4x)+C$$ Before we differentiate, first bash everything back down to an ``$x$'' in the argument using double angle identities. This produces \par $$\protect \frac  {5}{16}x+\protect \frac  {1}{2}\protect \qopname  \relax o{sin}(x)\protect \qopname  \relax o{cos}(x)-\protect \frac  {1}{6}\protect \qopname  \relax o{sin}^3(x)\protect \qopname  \relax o{cos}^3(x)+\protect \frac  {3}{16}\protect \qopname  \relax o{sin}(x)\protect \qopname  \relax o{cos}^3(x)-\protect \frac  {3}{16}\protect \qopname  \relax o{sin}^3(x)\protect \qopname  \relax o{cos}(x)+C$$ Factor out a sine and use the Pythagorean Identity to get everything else in terms of cosine. This produces $$\protect \frac  {5}{16}x+\protect \qopname  \relax o{sin}(x)\left (\protect \frac  {5}{16}\protect \qopname  \relax o{cos}(x)+\protect \frac  {5}{24}\protect \qopname  \relax o{cos}^3(x)+\protect \frac  {1}{6}\protect \qopname  \relax o{cos}^5(x)\right )+C$$ Then we differentiate and obtain $$\protect \frac  {5}{16}+\protect \qopname  \relax o{cos}(x)\left (\protect \frac  {5}{16}\protect \qopname  \relax o{cos}(x)+\protect \frac  {5}{24}\protect \qopname  \relax o{cos}^3(x)+\protect \frac  {1}{6}\protect \qopname  \relax o{cos}^5(x)\right )-\protect \qopname  \relax o{sin}^2(x)\left (\protect \frac  {5}{16}+\protect \frac  {5}{8}\protect \qopname  \relax o{cos}^2(x)+\protect \frac  {5}{6}\protect \qopname  \relax o{cos}^4(x)\right )$$ to which we apply the Pythagorean Identity $\protect \qopname  \relax o{sin}^2(x)=1-\protect \qopname  \relax o{cos}^2(x)$ to produce $$\protect \frac  {5}{16}+\protect \qopname  \relax o{cos}(x)\left (\protect \frac  {5}{16}\protect \qopname  \relax o{cos}(x)+\protect \frac  {5}{24}\protect \qopname  \relax o{cos}^3(x)+\protect \frac  {1}{6}\protect \qopname  \relax o{cos}^5(x)\right )-\left (1-\protect \qopname  \relax o{cos}^2(x)\right )\left (\protect \frac  {5}{16}+\protect \frac  {5}{8}\protect \qopname  \relax o{cos}^2(x)+\protect \frac  {5}{6}\protect \qopname  \relax o{cos}^4(x)\right )$$ This will simplify to $\protect \qopname  \relax o{cos}^6(x)$ once you expand and combine like terms.  \protect \newline  \protect \newline  
 {\noindent \protect \bf  Exercise ~1.1.2.5.} For the first, apply the identity $\protect \qopname  \relax o{sin}^2(3x)=\protect \frac  {1-\protect \qopname  \relax o{cos}(6x)}{2}$ and proceed. For the second, notice that $\protect \qopname  \relax o{sin}^4(x)$ can be rewritten as $\left (\protect \qopname  \relax o{sin}^2(x)\right )^2$, after which the half-angle identity can be applied. \protect \newline  \protect \newline  
 {\noindent \protect \bf  Exercise ~2.1.0.2.} This parametric curve is the line $y=\protect \frac  {3}{2}x+1$. \protect \newline  \protect \newline  
 {\noindent \protect \bf  Exercise ~2.1.0.3.} The two curves are the same points in the plane. Both start at the point (1,0) at time $t=0$, but $C_1$ then proceeds counter-clockwise while $C_2$ proceeds clockwise. \protect \newline  \protect \newline  
 {\noindent \protect \bf  Exercise ~2.3.0.4.} The arc length is $$\protect \frac  {6\protect \sqrt  {146}+\protect \qopname  \relax o{ln}\left (\protect \sqrt  {73}+6\protect \sqrt  {2}\right )}{6}\approx 12.55.$$ Also, to handle the absolute value, just find the arc length on the interval [0,2] where you can ignore the absolute value and then apply symmetry. \protect \newline  \protect \newline  
 {\noindent \protect \bf  Exercise ~2.3.0.5.} The arc length is $\protect \sqrt  {2}\left (e^{2\pi }-1\right )$. \protect \newline  \protect \newline  
 {\noindent \protect \bf  Exercise ~2.5.2.7.} Yes, it is in fact a circle with cartesian center $\left (0,1/2\right )$ and radius 1/2. This can be verified by demonstrating the polar equation converts to the cartesian equation $$x^2+\left (y-\protect \frac  {1}{2}\right )^2=\left (\protect \frac  {1}{2}\right )^2. $$ \protect \newline  \protect \newline  
 {\noindent \protect \bf  Exercise ~2.6.0.2.} The derivative is a constant; thus the graph is a straight line! \protect \newline  \protect \newline  
 {\noindent \protect \bf  Exercise ~2.7.0.4.} The area between the curves is $\protect \frac  {\pi }{8}$. \protect \newline  \protect \newline  
 {\noindent \protect \bf  Exercise ~2.7.0.5.} The area inside the inner loop of $r(\theta )=\protect \frac  {1}{2}+\protect \qopname  \relax o{cos}(\theta )$ is $\protect \frac  {\pi }{4}-\protect \frac  {3\protect \sqrt  {3}}{8}$. \protect \newline  \protect \newline  
 {\noindent \protect \bf  Exercise ~2.10.1.1.} a.~~ $r(0) = 4,~~r(\pi /6) = 8/\protect \sqrt  {3},~~r(\pi /4) =8/\protect \sqrt  {2}=4\protect \sqrt  {2},~~r(\pi /3)= 8 $ b.~~ $r=4\protect \qopname  \relax o{sec}{\theta } \DOTSB \protect \Relbar \protect \joinrel \Rightarrow r \protect \qopname  \relax o{cos}{\theta } = 4 \DOTSB \protect \Relbar \protect \joinrel \Rightarrow x=4$ is a vertical line. c.~~ d.~~It is an isosceles triangle with hypotenuse $4\protect \sqrt  {2}$ and sides 4 $A=8$ e.~~$A=8$ They are the same.  \protect \newline  \protect \newline  
 {\noindent \protect \bf  Exercise ~2.10.1.2.} a.)~~ It is a line segment that lies on the line $\protect \frac  {x-1}{4} = t = \protect \frac  {y}{6} \leftrightarrow y = \protect \frac  {3}{2} x - \protect \frac  {3}{2} $ between $(-1,0)$ and $(7,12)$ \protect \newline  b.)~~$\protect \frac  {3}{2}$ \protect \newline  c.)~~ $4\protect \sqrt  {13}$ \protect \newline  \protect \newline  
 {\noindent \protect \bf  Exercise ~2.10.2.1.} $\protect \qopname  \relax o{cosh}{t} = 1+\protect \frac  {t^2}{2!} + \protect \frac  {t^4}{4!}+\protect \frac  {t^6}{6!} + \protect \cdots  $\protect \newline  $\protect \qopname  \relax o{sinh}{t} = t + \protect \frac  {t^3}{3!} + \protect \frac  {t^5}{5!}+ \protect \cdots  $ \protect \newline  a.~~$\protect \frac  {d ~\protect \qopname  \relax o{sinh}(t)}{dt} = 1+\protect \frac  {t^2}{2!}+\protect \frac  {t^4}{4!} + \protect \cdots  =\protect \qopname  \relax o{cosh}(t)$ \protect \newline  b.~~$\protect \frac  {d ~\protect \qopname  \relax o{cosh}(t)}{dt} = t+\protect \frac  {t^3}{3!}+\protect \frac  {t^5}{5!} + \protect \cdots  =\protect \qopname  \relax o{sinh}(t)$ \protect \newline  c.~~$ \protect \qopname  \relax o{cosh}^2{t}- \protect \qopname  \relax o{sinh}^2{t} = \left (1+\protect \frac  {t^2}{2!} + \protect \frac  {t^4}{4!}+\protect \frac  {t^6}{6!} + \protect \cdots  \right )^2 - \left (t + \protect \frac  {t^3}{3!} + \protect \frac  {t^5}{5!}+ \protect \cdots  \right )^2 = 1 $ \protect \newline  d.~~$\protect \frac  {1}{0}$ which is a vertical line \protect \newline  \protect \newline  
 {\noindent \protect \bf  Exercise ~2.10.2.2.}  b.) $\protect \frac  {\pi }{8}$\protect \newline   \protect \newline  \protect \newline  
 {\noindent \protect \bf  Exercise ~3.2.0.4.} Any solution to $\protect \frac  {dy}{dx}=xy+x$ can be written as $y=Ce^{\protect \frac  {x^2}{2}}-1$ for some real number $C$. The second DE with initial condition has the solution $$\protect \frac  {1}{2}e^{-y}\left (\protect \qopname  \relax o{sin}(y)-\protect \qopname  \relax o{cos}(y)\right )=-e^{-x}\left (3+2x+x^2\right )+\protect \frac  {5}{2}. $$ \protect \newline  \protect \newline  
 {\noindent \protect \bf  Exercise ~3.6.1.1.} Linear combinations of hyperbolic sine and hyperbolic cosine functions are the only functions that equal their own second derivatives.  \protect \newline  \protect \newline  
 {\noindent \protect \bf  Exercise ~3.6.2.1.} a.~~ $y=3 +x + (a_0-3)e^x$ b.~~ If $y =3 +x + (a_0-3)e^x$ then $\protect \frac  {dy}{dx} = 1 +(a_0-3)e^x$ but $y-x-2 = 3+x+(a_0-3)e^x -x-2 = 1 + (a_0-3)e^x$ So they match.  \protect \newline  \protect \newline  
 {\noindent \protect \bf  Exercise ~3.6.2.2.} a.)~~ The right-hand side $2y+x$ does not factor into a function of $y$ times a function of $x$ so there can be no separation of variables.\protect \newline  b.)~~$y = -\protect \frac  {1}{4} -\protect \frac  {1}{2}x +Ce^{2x}$ \protect \newline   \protect \newline  \protect \newline  
 {\noindent \protect \bf  Exercise ~4.2.3.2.} \textbullet $\protect \qopname  \relax o{ln}(2)=\protect \qopname  \relax o{ln}(2)+0i$ \textbullet $\protect \qopname  \relax o{ln}(-2)=\protect \qopname  \relax o{ln}(2)+\pi i$ \textbullet $\protect \qopname  \relax o{ln}(1+i)=\protect \qopname  \relax o{ln}\left (\protect \sqrt  {2}\right )+i\protect \frac  {\pi }{4}$ \textbullet $\protect \qopname  \relax o{ln}(3-4i)=\protect \qopname  \relax o{ln}(5)+i\protect \qopname  \relax o{arctan}\left (-\protect \frac  {3}{4}\right )$ \protect \newline  \protect \newline  
 {\noindent \protect \bf  Exercise ~4.2.4.2.} The number $(1+i)^{1+i}$ can be written in complex cartesian form as $$\left (e^{\protect \qopname  \relax o{ln}\left (\protect \sqrt  {2}\right )-\protect \frac  {\pi }{4}}\protect \qopname  \relax o{cos}\left (\protect \qopname  \relax o{ln}\left (\protect \sqrt  {2}\right )+\protect \frac  {\pi }{4} \right )\right )+i\left (e^{\protect \qopname  \relax o{ln}\left (\protect \sqrt  {2}\right )-\protect \frac  {\pi }{4}}\protect \qopname  \relax o{sin}\left (\protect \qopname  \relax o{ln}\left (\protect \sqrt  {2}\right )+\protect \frac  {\pi }{4} \right )\right ). $$ \protect \newline  \protect \newline  
 {\noindent \protect \bf  Exercise ~4.3.0.1.} The PFD over the complex numbers is $$\protect \frac  {4-2x^2}{x^3+4x}=\protect \frac  {1}{x}-\protect \frac  {\protect \frac  {3}{2}}{x+2i}-\protect \frac  {\protect \frac  {3}{2}}{x-2i}. $$ \protect \newline  \protect \newline  
 {\noindent \protect \bf  Exercise ~4.3.0.2.} In taking the limit, the logarithmic term will approach zero. Thus $C=\pi /2$. \protect \newline  \protect \newline  
 {\noindent \protect \bf  Exercise ~4.5.1.1.} $z=e^{i (-\pi /4)}, e^{i(3 \pi /4)} = \protect \frac  {\protect \sqrt  {2}}{2} - i \protect \frac  {\protect \sqrt  {2}}{2},-\protect \frac  {\protect \sqrt  {2}}{2} + i \protect \frac  {\protect \sqrt  {2}}{2}$  \protect \newline  \protect \newline  
 {\noindent \protect \bf  Exercise ~4.5.2.1.} a.)~~$-i$ \protect \newline  b.)~~$e^{-\pi }$ \protect \newline  c.)~~$-1$ \protect \newline  d.)~~$\protect \frac  {\protect \sqrt  {3}}{2} + \protect \frac  {i}{2}, -\protect \frac  {\protect \sqrt  {3}}{2} + \protect \frac  {i}{2}, -i$ \protect \newline  e.)~~$\protect \qopname  \relax o{ln}{7} + i \protect \qopname  \relax o{arctan}{\protect \sqrt  {3}/12}$  \protect \newline  \protect \newline  
 {\noindent \protect \bf  Exercise ~4.5.2.2.} b.~~$r\protect \qopname  \relax o{cos}(\theta )+ri \protect \qopname  \relax o{sin}(\theta ) = re^{i \theta }$ This shows that $r$ and $\theta $ represent a radius and an angle.  \protect \newline  \protect \newline  
