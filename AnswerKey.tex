\par 
 {\noindent \protect \bf  Exercise ~2.0.0.1.} Saying that $F$ is an antiderivative of $f$ is equivalent to saying the derivative of $F$ is $f$. That is, $F'(x)=f(x)$. The Fundamental Theorem of Calculus states that after antidifferentiating the integrand, one can plug the bounds into the antiderivative and take their difference in order to calculate the integral. Because $F'(x)=f(x)$ by the definition of an antiderivative, a good way to check that your antiderivative $F$ is correct is to take its derivative $F'$. You should get the original function, $f$. \protect \newline  \protect \newline  
 {\noindent \protect \bf  Exercise ~2.1.1.1.} $ \intop f'(g(x)) \cdot g'(x) \mathop {}\protect \tmspace  -\thinmuskip {.1667em}\protect \mathtt  {d}x = \intop \left (f\left (g(x)\right )\right )' \mathop {}\protect \tmspace  -\thinmuskip {.1667em}\protect \mathtt  {d}x= f\left (g(x)\right )+C$ \protect \newline  \protect \newline  
 {\noindent \protect \bf  Exercise ~2.1.1.6.} Use the substitutions $u=x^2+x+8, \protect \qopname  \relax o{ln}(x),$ and $-x^2$. In the last case, the $\mathop {}\protect \tmspace  -\thinmuskip {.1667em}\protect \mathtt  {d}u$ term has nothing to cancel the $x$ with! \protect \newline  \protect \newline  
 {\noindent \protect \bf  Exercise ~2.1.2.2.} To have four intervals in the Riemann sum, $\Delta x$ would be 1 while $\Delta u$ would be 2. Thus, the width of each rectangle is getting doubled, since to convert between $u$ and $x$ we use the formula $u=2x+1$. The ``plus one'' merely slides all the rectangles one unit to the right, but it does not stretch their width at all, so it does not affect their area. Thus, the slope of the graph of $u=2x+1$ is the only thing that mattered regarding our conversion between $x$ and $u$. That is to say, the quantity $\mathop {}\protect \tmspace  -\thinmuskip {.1667em}\protect \mathtt  {d}u/ \mathop {}\protect \tmspace  -\thinmuskip {.1667em}\protect \mathtt  {d}x$ gives us the scaling factor. \protect \newline  \protect \newline  
 {\noindent \protect \bf  Exercise ~2.1.2.3.} The definite integral evaluates to roughly 0.95. The horizontal scaling factor at each $x$-coordinate should correspond to the derivative $\mathop {}\protect \tmspace  -\thinmuskip {.1667em}\protect \mathtt  {d}u /\mathop {}\protect \tmspace  -\thinmuskip {.1667em}\protect \mathtt  {d}x$ at each point.  \protect \newline  \protect \newline  
 {\noindent \protect \bf  Exercise ~2.2.1.3.} By factoring out the quantity $(x+1)^{3/2}$, both answers can be brought into the form $(x+1)^{3/2}\left (\protect \frac  {2}{5}x-\protect \frac  {4}{15}\right )+C$. \protect \newline  \protect \newline  
 {\noindent \protect \bf  Exercise ~2.2.2.2.} Use the substitution $u=1-x^2$. \protect \newline  \protect \newline  
 {\noindent \protect \bf  Exercise ~2.2.2.3.} The antiderivative is $x\protect \qopname  \relax o{ln}(x)-x+C$. \protect \newline  \protect \newline  
 {\noindent \protect \bf  Exercise ~2.2.3.5.} The antiderivative is $\protect \frac  {1}{2}\left (\protect \qopname  \relax o{sec}(x)\protect \qopname  \relax o{tan}(x)+\protect \qopname  \relax o{ln}|\protect \qopname  \relax o{sec}(x)+\protect \qopname  \relax o{tan}(x)|\right )+C$. \protect \newline  \protect \newline  
 {\noindent \protect \bf  Exercise ~2.3.0.1.} Use the substitution $u=\protect \sqrt  {x}$ to transform the first integral into $\intop 2u\protect \qopname  \relax o{cos}(u) \mathop {}\protect \tmspace  -\thinmuskip {.1667em}\protect \mathtt  {d}u$. \protect \newline  \protect \newline  
 {\noindent \protect \bf  Exercise ~2.3.0.2.} Choosing $u=\protect \qopname  \relax o{ln}(x)$ will make the logarithm disappear upon differentiation. The opposite choice will not clean up the log. \protect \newline  \protect \newline  
 {\noindent \protect \bf  Exercise ~2.4.0.2.} The antiderivative is $\protect \frac  {1}{3}\protect \qopname  \relax o{sin}^3(x)+C$. \protect \newline  \protect \newline  
 {\noindent \protect \bf  Exercise ~2.4.1.2.} Since seven is odd, when we pulled out one factor of sine, we ended up with the sixth power of sine remaining. Since six is even, we were able to express it as a power of a perfect square of sine, which in turn let us rewrite as cosines using the Pythagorean identity. \protect \newline  \protect \newline  
 {\noindent \protect \bf  Exercise ~2.4.1.3.} The first antiderivative is $-\protect \frac  {1}{3}\protect \qopname  \relax o{cos}^3(x)+\protect \frac  {2}{5}\protect \qopname  \relax o{cos}^5(x)-\protect \frac  {1}{7}\protect \qopname  \relax o{cos}^7(x)+C$. For the second, rewrite as $(1-\protect \qopname  \relax o{sin}^2(x))^4\protect \qopname  \relax o{cos}(x)$ and proceed by letting $u=\protect \qopname  \relax o{sin}(x)$. \protect \newline  \protect \newline  
 {\noindent \protect \bf  Exercise ~2.4.1.4.} Often when trying to show that two antiderivatives are compatible, it is easiest to verify that their difference is a constant.  \protect \newline  \protect \newline  
 {\noindent \protect \bf  Exercise ~2.4.1.4.} The substitution $u=\protect \qopname  \relax o{sin}(x)$ is much cleaner since the other will involve having to expand a binomial to the fifth power. The antiderivative is $\protect \frac  {1}{12}\protect \qopname  \relax o{sin}^{12}(x)-\protect \frac  {1}{14}\protect \qopname  \relax o{sin}^{14}(x)+C$. \protect \newline  \protect \newline  
 {\noindent \protect \bf  Exercise ~2.4.2.1.} The exponent on sine is zero, which is indeed even. Thus both exponents are even in this case. \protect \newline  \protect \newline  
 {\noindent \protect \bf  Exercise ~2.4.2.3.} When all like terms are combined and the one-eighth is distributed, the result is $\protect \frac  {5}{16}x+\protect \frac  {1}{4}\protect \qopname  \relax o{sin}(2x)-\protect \frac  {1}{48}\protect \qopname  \relax o{sin}^3(2x)+\protect \frac  {3}{64}\protect \qopname  \relax o{sin}(4x)+C$. \protect \newline  \protect \newline  
 {\noindent \protect \bf  Exercise ~2.4.2.4.} The antiderivative to $\protect \qopname  \relax o{cos}^6(x)$ came out to \par $$\protect \frac  {5}{16}x+\protect \frac  {1}{4}\protect \qopname  \relax o{sin}(2x)-\protect \frac  {1}{48}\protect \qopname  \relax o{sin}^3(2x)+\protect \frac  {3}{64}\protect \qopname  \relax o{sin}(4x)+C$$ Before we differentiate, first bash everything back down to an ``$x$'' in the argument using double angle identities. This produces \par $$\protect \frac  {5}{16}x+\protect \frac  {1}{2}\protect \qopname  \relax o{sin}(x)\protect \qopname  \relax o{cos}(x)-\protect \frac  {1}{6}\protect \qopname  \relax o{sin}^3(x)\protect \qopname  \relax o{cos}^3(x)+\protect \frac  {3}{16}\protect \qopname  \relax o{sin}(x)\protect \qopname  \relax o{cos}^3(x)-\protect \frac  {3}{16}\protect \qopname  \relax o{sin}^3(x)\protect \qopname  \relax o{cos}(x)+C$$ Factor out a sine and use the Pythagorean Identity to get everything else in terms of cosine. This produces $$\protect \frac  {5}{16}x+\protect \qopname  \relax o{sin}(x)\left (\protect \frac  {5}{16}\protect \qopname  \relax o{cos}(x)+\protect \frac  {5}{24}\protect \qopname  \relax o{cos}^3(x)+\protect \frac  {1}{6}\protect \qopname  \relax o{cos}^5(x)\right )+C$$ Then we differentiate and obtain $$\protect \frac  {5}{16}+\protect \qopname  \relax o{cos}(x)\left (\protect \frac  {5}{16}\protect \qopname  \relax o{cos}(x)+\protect \frac  {5}{24}\protect \qopname  \relax o{cos}^3(x)+\protect \frac  {1}{6}\protect \qopname  \relax o{cos}^5(x)\right )-\protect \qopname  \relax o{sin}^2(x)\left (\protect \frac  {5}{16}+\protect \frac  {5}{8}\protect \qopname  \relax o{cos}^2(x)+\protect \frac  {5}{6}\protect \qopname  \relax o{cos}^4(x)\right )$$ to which we apply the Pythagorean Identity $\protect \qopname  \relax o{sin}^2(x)=1-\protect \qopname  \relax o{cos}^2(x)$ to produce $$\protect \frac  {5}{16}+\protect \qopname  \relax o{cos}(x)\left (\protect \frac  {5}{16}\protect \qopname  \relax o{cos}(x)+\protect \frac  {5}{24}\protect \qopname  \relax o{cos}^3(x)+\protect \frac  {1}{6}\protect \qopname  \relax o{cos}^5(x)\right )-\left (1-\protect \qopname  \relax o{cos}^2(x)\right )\left (\protect \frac  {5}{16}+\protect \frac  {5}{8}\protect \qopname  \relax o{cos}^2(x)+\protect \frac  {5}{6}\protect \qopname  \relax o{cos}^4(x)\right )$$ This will simplify to $\protect \qopname  \relax o{cos}^6(x)$ once you expand and combine like terms.  \protect \newline  \protect \newline  
 {\noindent \protect \bf  Exercise ~2.4.2.5.} For the first, apply the identity $\protect \qopname  \relax o{sin}^2(3x)=\protect \frac  {1-\protect \qopname  \relax o{cos}(6x)}{2}$ and proceed. For the second, notice that $\protect \qopname  \relax o{sin}^4(x)$ can be rewritten as $\left (\protect \qopname  \relax o{sin}^2(x)\right )^2$, after which the half-angle identity can be applied. \protect \newline  \protect \newline  
 {\noindent \protect \bf  Exercise ~2.5.1.3.} First apply all the product and chain rules to reach the expression $$\protect \frac  {3}{\protect \sqrt  {1-\protect \frac  {x^2}{4}}}+4\protect \sqrt  {1-\protect \frac  {x^2}{4}}+\protect \frac  {-x^2}{\protect \sqrt  {1-\protect \frac  {x^2}{4}}}+\protect \sqrt  {1-\protect \frac  {x^2}{4}}\left (1-\protect \frac  {3}{2}x^2\right )+\protect \frac  {-x}{4\protect \sqrt  {1-\protect \frac  {x^2}{4}}}\left (x-\protect \frac  {x^3}{2}\right ) $$ Put all terms over the common denominator $\protect \sqrt  {4-x^2}$ and combine like terms in the numerator. Notice the numerator becomes $\left (4-x^2\right )^2$ and then reduce for the win! \protect \newline  \protect \newline  
 {\noindent \protect \bf  Exercise ~2.5.1.4.} The antiderivative is $2^{18}\left (\protect \frac  {\left (1-x^2/16\right )^{9/2}}{9}-\protect \frac  {\left (1-x^2/16\right )^{7/2}}{7}\right )+C$ \protect \newline  \protect \newline  
 {\noindent \protect \bf  Exercise ~2.5.2.3.} The antiderivative is $\protect \frac  {x\protect \sqrt  {x^2-4}}{2}-2\protect \qopname  \relax o{ln}|x+\protect \sqrt  {x^2-4}|+C$. \protect \newline  \protect \newline  
 {\noindent \protect \bf  Exercise ~2.6.1.5.} Using properties of logarithms, both answers should be able to be put in the form $\protect \qopname  \relax o{ln}\left | \protect \sqrt  {\protect \frac  {x-1}{x+1}}\right |+C$ \protect \newline  \protect \newline  
 {\noindent \protect \bf  Exercise ~2.6.3.1.} \textbullet The function $\protect \frac  {1}{x^2-9x+20}$ has $\protect \qopname  \relax o{ln}\left |\protect \frac  {x-5}{x-4}\right |+C$ as its antiderivative. \textbullet The factorization $x^4-9=\left (x^2+3\right )\left (x-\protect \sqrt  {3}\right )\left (x+\protect \sqrt  {3}\right )$ will produce the following setup: $$\protect \frac  {1}{x^4-9}=\protect \frac  {Ax+B}{x^2+3}+\protect \frac  {C}{x-\protect \sqrt  {3}}+\protect \frac  {D}{x+\protect \sqrt  {3}} $$ in which you can then solve for the coefficients and antidifferentiate. \textbullet The function $\protect \frac  {x^4}{x^2+1}$ has an irreducible quadratic for a denominator. However, the degree of the numerator is not smaller than the degree of the denominator. Thus, polynomial long division is the only step of PFD that is required in this case. \textbullet The antiderivative of $\protect \frac  {2}{x^5+2x3+x}$ is $$2\protect \qopname  \relax o{ln}|x|-\protect \qopname  \relax o{ln}\left | x^2+1\right |+\protect \frac  {1}{x^2+1}$$ \textbullet The PFD will produce $$ \protect \frac  {x-2}{x^3+x^2+3x-5}=\protect \frac  {-\protect \frac  {1}{8}}{x-1}+\protect \frac  {\protect \frac  {1}{8}x+\protect \frac  {11}{8}}{x^2+2x+5}$$ While the first term is easy to integrate, the second is quite tricky! To hack through it, split it as follows: $$\protect \frac  {\protect \frac  {1}{8}x+\protect \frac  {11}{8}}{x^2+2x+5}=\protect \frac  {\protect \frac  {1}{8}x+\protect \frac  {1}{8}}{x^2+2x+5}+\protect \frac  {\protect \frac  {10}{8}}{x^2+2x+5} $$ The first fraction can then be integrated via $u$-sub, while the second can be done via trig sub after completing the square on the denominator. \protect \newline  \protect \newline  
 {\noindent \protect \bf  Exercise ~2.6.3.2.} For $\protect \frac  {1}{x^4-9x^2}$, keep in mind that $x^2$ is not an irreducible quadratic factor but rather a repeated linear factor. The PFD and integration will produce $$\protect \frac  {1}{9x}+\protect \frac  {1}{54}\protect \qopname  \relax o{ln}\left |\protect \frac  {x-3}{x+3}\right |+C $$  \protect \newline  \protect \newline  
 {\noindent \protect \bf  Exercise ~3.1.0.4.} The limits are 0, $1/\pi $, and -1.  \protect \newline  \protect \newline  
 {\noindent \protect \bf  Exercise ~3.1.1.3.} The limits are 0, -2, and $e$.  \protect \newline  \protect \newline  
 {\noindent \protect \bf  Exercise ~3.1.1.4.} The results are 1, 0, and 1. \protect \newline  \protect \newline  
 {\noindent \protect \bf  Exercise ~3.1.2.2.} Their ratio converges to 3 (both numerically in the table, and analytically as evaluated by LHR). Since this is a nonzero constant, the two functions have the same growth order. \protect \newline  \protect \newline  
 {\noindent \protect \bf  Exercise ~3.1.2.3.} In the first and third, the ratio between $f$ and $g$ seems to grow without bound, so $f$ has larger growth order. In the second, the ratio of $f$ to $g$ seems to always be right around 2. Thus, they have the same growth order. \protect \newline  \protect \newline  
 {\noindent \protect \bf  Exercise ~3.2.1.8.} The integrals evaluate to $2\protect \sqrt  {2},\infty ,\infty ,$ and $\infty $. \protect \newline  \protect \newline  
 {\noindent \protect \bf  Exercise ~3.2.2.4.} \textbullet The area under $xe^{-x^2}$ from zero to $\infty $ is $\protect \frac  {1}{2}$. \textbullet Splitting into two integrals at $x=0$ produces one of area one-half and one of area negative one-half, so the total integral is zero. \textbullet After applying IBP with $u=x$ and $\mathop {}\protect \tmspace  -\thinmuskip {.1667em}\protect \mathtt  {d}v = xe^{-x^2}\mathop {}\protect \tmspace  -\thinmuskip {.1667em}\protect \mathtt  {d}x$, one obtains $\protect \frac  {\protect \sqrt  {\pi }}{2}$ as the area under the curve. \textbullet The area under $\protect \frac  {1}{x\protect \qopname  \relax o{ln}(x)}$ from 2 to $\infty $ is infinite. \textbullet The area under $\protect \frac  {1}{x\left (\protect \qopname  \relax o{ln}(x)\right )^2}$ from 2 to $\infty $ is $\protect \frac  {1}{\protect \qopname  \relax o{ln}(2)}$. \textbullet An improper integral is defined using a limit, and here the limit does not exist, as the area keeps going up and down by the same amount forever. \protect \newline  \protect \newline  
 {\noindent \protect \bf  Exercise ~3.3.2.1.} \textbullet The curves $y=x^3+x^2-x-1$ and $y=x^3-x^2-x+1$ intersect on the $x$ axis at -1 and 1 and have area 8/3 between them. \textbullet The area inside the unit circle but above the line $y=1/2$ is $\pi /3-\protect \sqrt  {3}/4$. \textbullet Notice graphically that the curves intersect at $x=\pm \pi /4$. The area between curves is $\pi /4-\protect \qopname  \relax o{ln}(2)$. \protect \newline  \protect \newline  
 {\noindent \protect \bf  Exercise ~3.4.1.2.} The exact arc length is $\protect \frac  {2\protect \sqrt  {5}+\protect \qopname  \relax o{ln}\left | 2+\protect \sqrt  {5}\right |}{4}$. \protect \newline  \protect \newline  
 {\noindent \protect \bf  Exercise ~3.4.3.1.} The length of the graph of the natural logarithm from (1,0) to (e,1) is $$\protect \sqrt  {e^2+1}+\protect \frac  {1}{2}\protect \qopname  \relax o{ln}\left | \protect \frac  {\protect \sqrt  {e^2+1}-1}{\protect \sqrt  {e^2+1}+1}\right |-\protect \sqrt  {2}+\protect \frac  {1}{2}\protect \qopname  \relax o{ln}\left | \protect \frac  {\protect \sqrt  {2}+1}{\protect \sqrt  {2}-1}\right | $$ which is roughly 2.003497. Also, notice that the natural exponential function is just the inverse of the natural logarithm; think about what this means regarding arc length! \protect \newline  \protect \newline  
 {\noindent \protect \bf  Exercise ~3.4.4.1.} The two-frusta approximation is $$\protect \frac  {\pi }{8}\left (\protect \sqrt  {5}+3\protect \sqrt  {13} \right )\approx 5.126 $$ The exact value of the surface area is $$\protect \frac  {\pi }{6}\left (5\protect \sqrt  {5}-1 \right )\approx 5.3304 $$ which is just slightly larger, as one would expect.  \protect \newline  \protect \newline  
 {\noindent \protect \bf  Exercise ~3.4.6.1.} The volume of the torus is $$V=2\pi ^2Rr^2 $$ and the surface area is $$SA=4\pi ^2Rr $$ \protect \newline  \protect \newline  
 {\noindent \protect \bf  Exercise ~3.5.2.1.} The diagonals have the equations $$ y=\protect \frac  {b}{a}x \protect \text  { and } y=\protect \frac  {b-2c}{a}x+c$$ with intersection point $(a/2,b/2)$, which is also the center of mass of the region. \protect \newline  \protect \newline  
 {\noindent \protect \bf  Exercise ~3.5.3.2.} The coordinates of the vertices are (0,0), (0,$c$), and ($a,b$). Two of the medians are $$y=\protect \frac  {b+c}{a}x \protect \text  { and } y=\protect \frac  {2b-c}{2a}x+\protect \frac  {c}{2}$$ and their intersection point (and center of mass of the triangle) is $\left (\protect \frac  {a}{3},\protect \frac  {b+c}{3}\right )$. \protect \newline  \protect \newline  
 {\noindent \protect \bf  Exercise ~3.5.4.1.} The center of mass is $\left ( 0, \protect \frac  {4r}{3\pi }\right )$. \protect \newline  \protect \newline  
 {\noindent \protect \bf  Exercise ~4.4.1.3.} \textbullet $\protect \frac  {1}{n+1}$ \textbullet $n+1$ \textbullet $(n+2)(n+1)$ \textbullet $(2n+2)(2n+1)$ \protect \newline  \protect \newline  
 {\noindent \protect \bf  Exercise ~4.4.3.1.} In the context of computing a limit to infinity, it is fine to replace $n!$ by $\protect \sqrt  {2\pi n} \left ( \protect \frac  {n}{e} \right )^n$. Setting up limits of ratios and testing growth order with LHR and good old algebra will then verify that the order goes $n^2,e^n,n!,n^n$. \protect \newline  \protect \newline  
 {\noindent \protect \bf  Exercise ~4.5.0.3.} Think about what happens if the common difference $d$ is zero and if the common ratio $r$ is 1. \protect \newline  \protect \newline  
 {\noindent \protect \bf  Exercise ~4.7.0.3.} In most cases, it is easiest to just expand the sums on both sides and see what the terms look like. \protect \newline  \protect \newline  
 {\noindent \protect \bf  Exercise ~4.8.0.5.} The totals are 500500, 1501500, 214214, and 245. \protect \newline  \protect \newline  
 {\noindent \protect \bf  Exercise ~4.9.0.5.} $\protect \$5 \protect \text  { billion }\cdot \protect \frac  {1-0.8^{13}}{1-0.8}\approx \protect \$23.6 \protect \text  { billion}$ \protect \newline  \protect \newline  
 {\noindent \protect \bf  Exercise ~4.9.0.6.} Try to notice how the summations relate to the Fibonacci numbers themselves! \protect \newline  \protect \newline  
 {\noindent \protect \bf  Exercise ~4.11.1.1.} The sequence is $a_n=\protect \frac  {1}{2^{n+1}}$. Since this is a geometric sequence, the finite geometric series formula can be applied to then find the sequence of partial sums $A_N$. \protect \newline  \protect \newline  
 {\noindent \protect \bf  Exercise ~4.11.1.2.} It ends up one-third of a meter forward from where it started. \protect \newline  \protect \newline  
 {\noindent \protect \bf  Exercise ~4.11.1.5.} Yes, the series is geometric with initial term $\protect \frac  {3^5}{2^{11}}$ and common ratio $3/4$. The infinite series totals to $ \protect \frac  {3^5}{2^9}$. \protect \newline  \protect \newline  
 {\noindent \protect \bf  Exercise ~4.11.1.7.} The partial sums are $A_N=2(N+1)$. The infinite series is the limit of $A_N$ as $N$ goes to infinity, which here is clearly again infinity. Thus, the infinite series diverges. \protect \newline  \protect \newline  
 {\noindent \protect \bf  Exercise ~4.11.1.8.} The partial sums are $$A_N=\protect \frac  {N+1}{N+3}$$ for an infinite sum of 1. \protect \newline  \protect \newline  
 {\noindent \protect \bf  Exercise ~5.1.0.4.} Written in sigma notation, the power series is $\protect \qopname  \relax o{sin}(x)=\Sigma _{n=0}^\infty (-1)^{n}\protect \frac  {1}{\left (2n+1\right )!}x^{2n+1}$. \protect \newline  \protect \newline  
 {\noindent \protect \bf  Exercise ~5.1.0.5.} Written in sigma notation, the power series is $e^x=\Sigma _{n=0}^\infty \protect \frac  {1}{n!}x^{n}$. \protect \newline  \protect \newline  
 {\noindent \protect \bf  Exercise ~5.1.0.6.} The power series is $\protect \frac  {1}{1-x}=\Sigma _{n=0}^\infty x^n$. It is a geometric series with initial term 1 and common ratio $x$. \protect \newline  \protect \newline  
 {\noindent \protect \bf  Exercise ~5.1.0.8.} When we try to plug in $x=0$ to find $a_0$, we get $\protect \qopname  \relax o{ln}(0)$ which is not a real number. \protect \newline  \protect \newline  
 {\noindent \protect \bf  Exercise ~5.1.0.9.} The power series centered at one for the natural log is $\protect \qopname  \relax o{ln}(x)=\Sigma _{n=1}^\infty \protect \frac  {(-1)^{n+1}}{n}(x-1)^n$. \protect \newline  \protect \newline  
 {\noindent \protect \bf  Exercise ~5.4.1.1.} If we substitute $x-1$ for $x$ in the power series for sine, we get $\protect \qopname  \relax o{sin}(x-1)=\Sigma _{n=0}^\infty (-1)^{n}\protect \frac  {1}{\left (2n+1\right )!}(x-1)^{2n+1}$. Likewise, substituting $2x$ for $x$ in the power series for sine produces $\protect \qopname  \relax o{sin}(2x)=\Sigma _{n=0}^\infty (-1)^{n}\protect \frac  {1}{\left (2n+1\right )!}(2x)^{2n+1}=\Sigma _{n=0}^\infty (-1)^{n}\protect \frac  {2^{2n+1}}{\left (2n+1\right )!}x^{2n+1}$. \protect \newline  \protect \newline  
 {\noindent \protect \bf  Exercise ~5.4.2.1.} The power series $\protect \frac  {1}{x^2-x-12}=\Sigma _{n=0}^\infty \left (\protect \frac  {-1}{21\cdot (-3)^n}-\protect \frac  {1}{28\cdot 4^n}\right )x^n$ has IOC (-3,3). The power series $\protect \frac  {1}{x}=\Sigma _{n=0}^\infty \protect \frac  {(-1)^n}{5^{n+1}}(x-5)^n$ has IOC (0,10). It turns out these two examples generalize; for rational functions, the IOC will always just be the interval that goes from the center of the series outwards until it bumps into the nearest vertical asymptote! \protect \newline  \protect \newline  
 {\noindent \protect \bf  Exercise ~5.4.4.1.} Antidifferentiate the geometric series to sneak up on $\protect \qopname  \relax o{ln}(1-x)$. \protect \newline  \protect \newline  
 {\noindent \protect \bf  Exercise ~5.4.4.2.} Each method should lead to $$\protect \frac  {1}{(1-x)^2}=1+2x+3x^2+4x^3+5x^4+\protect \cdots  $$ \protect \newline  \protect \newline  
 {\noindent \protect \bf  Exercise ~5.6.0.2.} If $n=1$, we have the following degree one power series centered at $a=4$: $$f(x)=\protect \sqrt  {x}\approx 2+\protect \frac  {1}{4}(x-4)$$ Since $n=1$, we need the second derivative. We compute $\left |f''(x)\right |=\protect \frac  {1}{4x^{3/2}}$, which on the interval [4,4.1] has its maximum at $x=4$. Thus, $M=1/32$, which provides an error bound of $$\protect \frac  {\protect \frac  {1}{32}\cdot |4-4.1|^2}{2!}=\protect \frac  {1}{6400}$$ Thus the error is definitely less than one thousandth, but not necessarily less than one ten thousandth. So for the approximation $$\protect \sqrt  {4.1}\approx 2+\protect \frac  {1}{4}(4.1-4)=2.025$$ we can guarantee three digits past the decimal are correct but not necessarily the fourth. That is, 2.025 is certainly the correct decimal expansion for $\protect \sqrt  {4.1}$ rounded to the thousandths place. However, the digit 0 we implicitly have in the ten-thousandths place may or may not be correct. This process can be repeated for the other $n$ values of two and three. \protect \newline  \protect \newline  
 {\noindent \protect \bf  Exercise ~6.3.0.4.} The arc length is $$\protect \frac  {6\protect \sqrt  {146}+\protect \qopname  \relax o{ln}\left (\protect \sqrt  {73}+6\protect \sqrt  {2}\right )}{6}\approx 12.55$$ \protect \newline  \protect \newline  
 {\noindent \protect \bf  Exercise ~6.3.0.5.} The arc length is $\protect \sqrt  {2}\left (e^{2\pi }-1\right )$. \protect \newline  \protect \newline  
 {\noindent \protect \bf  Exercise ~6.5.2.7.} Yes, it is in fact a circle with cartesian center $\left (0,1/2\right )$ and radius 1/2. This can be verified by demonstrating the polar equation converts to the cartesian equation $$x^2+\left (y-\protect \frac  {1}{2}\right )^2=\left (\protect \frac  {1}{2}\right )^2 $$ \protect \newline  \protect \newline  
 {\noindent \protect \bf  Exercise ~6.6.0.2.} The derivative is a constant; thus the graph is a straight line! \protect \newline  \protect \newline  
 {\noindent \protect \bf  Exercise ~6.7.0.4.} The area between the curves is $\protect \frac  {\pi }{8}$. \protect \newline  \protect \newline  
 {\noindent \protect \bf  Exercise ~6.7.0.5.} The area inside the inner loop of $r(\theta )=\protect \frac  {1}{2}+\protect \qopname  \relax o{cos}(\theta )$ is $\protect \frac  {\pi }{4}-\protect \frac  {3\protect \sqrt  {3}}{8}$. \protect \newline  \protect \newline  
 {\noindent \protect \bf  Exercise ~7.2.0.4.} Any solution to $\protect \frac  {dy}{dx}=xy+x$ can be written as $y=Ce^{\protect \frac  {x^2}{2}}-1$ for some real number $C$. The second DE with initial condition has the solution $$\protect \frac  {1}{2}e^{-y}\left (\protect \qopname  \relax o{sin}(y)-\protect \qopname  \relax o{cos}(y)\right )=-e^{-x}\left (3+2x+x^2\right )+\protect \frac  {5}{2} $$ \protect \newline  \protect \newline  
 {\noindent \protect \bf  Exercise ~8.2.3.1.} \textbullet $\protect \qopname  \relax o{ln}(2)=\protect \qopname  \relax o{ln}(2)+0i$ \textbullet $\protect \qopname  \relax o{ln}(-2)=\protect \qopname  \relax o{ln}(2)+\pi i$ \textbullet $\protect \qopname  \relax o{ln}(i)=0+i\protect \frac  {\pi }{2}$ \textbullet $\protect \qopname  \relax o{ln}(1+i)=\protect \qopname  \relax o{ln}\left (\protect \sqrt  {2}\right )+i\protect \frac  {\pi }{4}$ \textbullet $\protect \qopname  \relax o{ln}(3-4i)=\protect \qopname  \relax o{ln}(5)+i\protect \qopname  \relax o{arctan}\left (-\protect \frac  {3}{4}\right )$ \protect \newline  \protect \newline  
 {\noindent \protect \bf  Exercise ~8.2.4.1.} The number $(1+i)^{1+i}$ can be written in complex cartesian form as $$\left (e^{\protect \qopname  \relax o{ln}\left (\protect \sqrt  {2}\right )-\protect \frac  {\pi }{4}}\protect \qopname  \relax o{cos}\left (\protect \qopname  \relax o{ln}\left (\protect \sqrt  {2}\right )+\protect \frac  {\pi }{4} \right )\right )+i\left (e^{\protect \qopname  \relax o{ln}\left (\protect \sqrt  {2}\right )-\protect \frac  {\pi }{4}}\protect \qopname  \relax o{sin}\left (\protect \qopname  \relax o{ln}\left (\protect \sqrt  {2}\right )+\protect \frac  {\pi }{4} \right )\right ) $$ \protect \newline  \protect \newline  
 {\noindent \protect \bf  Exercise ~8.2.5.1.} The PFD over the complex numbers is $$\protect \frac  {4-2x^2}{x^3+4x}=\protect \frac  {1}{x}-\protect \frac  {\protect \frac  {3}{2}}{x+2i}-\protect \frac  {\protect \frac  {3}{2}}{x-2i} $$ \protect \newline  \protect \newline  
