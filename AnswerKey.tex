\par 
 {\noindent \protect \bf  Exercise ~1.1.1.5.} Using properties of logarithms, both answers should be able to be put in the form $\protect \qopname  \relax o{ln}\left | \protect \sqrt  {\protect \frac  {x-1}{x+1}}\right |+C$ \protect \newline  \protect \newline  
 {\noindent \protect \bf  Exercise ~1.1.3.1.} \textbullet The function $\protect \frac  {1}{x^2-9x+20}$ has $\protect \qopname  \relax o{ln}\left |\protect \frac  {x-5}{x-4}\right |+C$ as its antiderivative. \textbullet The factorization $x^4-9=\left (x^2+3\right )\left (x-\protect \sqrt  {3}\right )\left (x+\protect \sqrt  {3}\right )$ will produce the following setup: $$\protect \frac  {1}{x^4-9}=\protect \frac  {Ax+B}{x^2+3}+\protect \frac  {C}{x-\protect \sqrt  {3}}+\protect \frac  {D}{x+\protect \sqrt  {3}} $$ in which you can then solve for the coefficients and antidifferentiate. \textbullet The function $\protect \frac  {x^4}{x^2+1}$ has an irreducible quadratic for a denominator. However, the degree of the numerator is not smaller than the degree of the denominator. Thus, polynomial long division is the only step of PFD that is required in this case. \textbullet The antiderivative of $\protect \frac  {2}{x^5+2x3+x}$ is $$2\protect \qopname  \relax o{ln}|x|-\protect \qopname  \relax o{ln}\left | x^2+1\right |+\protect \frac  {1}{x^2+1}$$ \textbullet The PFD will produce $$ \protect \frac  {x-2}{x^3+x^2+3x-5}=\protect \frac  {-\protect \frac  {1}{8}}{x-1}+\protect \frac  {\protect \frac  {1}{8}x+\protect \frac  {11}{8}}{x^2+2x+5}$$ While the first term is easy to integrate, the second is quite tricky! To hack through it, split it as follows: $$\protect \frac  {\protect \frac  {1}{8}x+\protect \frac  {11}{8}}{x^2+2x+5}=\protect \frac  {\protect \frac  {1}{8}x+\protect \frac  {1}{8}}{x^2+2x+5}+\protect \frac  {\protect \frac  {10}{8}}{x^2+2x+5} $$ The first fraction can then be integrated via $u$-sub, while the second can be done via trig sub after completing the square on the denominator. \protect \newline  \protect \newline  
 {\noindent \protect \bf  Exercise ~1.1.3.2.} For $\protect \frac  {1}{x^4-9x^2}$, keep in mind that $x^2$ is not an irreducible quadratic factor but rather a repeated linear factor. The PFD and integration will produce $$\protect \frac  {1}{9x}+\protect \frac  {1}{54}\protect \qopname  \relax o{ln}\left |\protect \frac  {x-3}{x+3}\right |+C $$  \protect \newline  \protect \newline  
 {\noindent \protect \bf  Exercise ~2.1.0.2.} This parametric curve is the line $y=\protect \frac  {3}{2}x+1$. \protect \newline  \protect \newline  
 {\noindent \protect \bf  Exercise ~2.1.0.3.} The two curves are the same points in the plane. Both start at the point (1,0) at time $t=0$, but $C_1$ then proceeds counter-clockwise while $C_2$ proceeds clockwise. \protect \newline  \protect \newline  
 {\noindent \protect \bf  Exercise ~2.3.0.4.} The arc length is $$\protect \frac  {6\protect \sqrt  {146}+\protect \qopname  \relax o{ln}\left (\protect \sqrt  {73}+6\protect \sqrt  {2}\right )}{6}\approx 12.55.$$ Also, to handle the absolute value, just find the arc length on the interval [0,2] where you can ignore the absolute value and then apply symmetry. \protect \newline  \protect \newline  
 {\noindent \protect \bf  Exercise ~2.3.0.5.} The arc length is $\protect \sqrt  {2}\left (e^{2\pi }-1\right )$. \protect \newline  \protect \newline  
 {\noindent \protect \bf  Exercise ~2.5.2.7.} Yes, it is in fact a circle with cartesian center $\left (0,1/2\right )$ and radius 1/2. This can be verified by demonstrating the polar equation converts to the cartesian equation $$x^2+\left (y-\protect \frac  {1}{2}\right )^2=\left (\protect \frac  {1}{2}\right )^2. $$ \protect \newline  \protect \newline  
 {\noindent \protect \bf  Exercise ~2.6.0.2.} The derivative is a constant; thus the graph is a straight line! \protect \newline  \protect \newline  
 {\noindent \protect \bf  Exercise ~2.7.0.4.} The area between the curves is $\protect \frac  {\pi }{8}$. \protect \newline  \protect \newline  
 {\noindent \protect \bf  Exercise ~2.7.0.5.} The area inside the inner loop of $r(\theta )=\protect \frac  {1}{2}+\protect \qopname  \relax o{cos}(\theta )$ is $\protect \frac  {\pi }{4}-\protect \frac  {3\protect \sqrt  {3}}{8}$. \protect \newline  \protect \newline  
 {\noindent \protect \bf  Exercise ~2.10.1.1.} a.~~ $r(0) = 4,~~r(\pi /6) = 8/\protect \sqrt  {3},~~r(\pi /4) =8/\protect \sqrt  {2}=4\protect \sqrt  {2},~~r(\pi /3)= 8 $ b.~~ $r=4\protect \qopname  \relax o{sec}{\theta } \DOTSB \protect \Relbar \protect \joinrel \Rightarrow r \protect \qopname  \relax o{cos}{\theta } = 4 \DOTSB \protect \Relbar \protect \joinrel \Rightarrow x=4$ is a vertical line. c.~~ d.~~It is an isosceles triangle with hypotenuse $4\protect \sqrt  {2}$ and sides 4 $A=8$ e.~~$A=8$ They are the same.  \protect \newline  \protect \newline  
 {\noindent \protect \bf  Exercise ~2.10.1.2.} a.)~~ It is a line segment that lies on the line $\protect \frac  {x-1}{4} = t = \protect \frac  {y}{6} \leftrightarrow y = \protect \frac  {3}{2} x - \protect \frac  {3}{2} $ between $(-1,0)$ and $(7,12)$ \protect \newline  b.)~~$\protect \frac  {3}{2}$ \protect \newline  c.)~~ $4\protect \sqrt  {13}$ \protect \newline  \protect \newline  
 {\noindent \protect \bf  Exercise ~2.10.2.1.} $\protect \qopname  \relax o{cosh}{t} = 1+\protect \frac  {t^2}{2!} + \protect \frac  {t^4}{4!}+\protect \frac  {t^6}{6!} + \protect \cdots  $\protect \newline  $\protect \qopname  \relax o{sinh}{t} = t + \protect \frac  {t^3}{3!} + \protect \frac  {t^5}{5!}+ \protect \cdots  $ \protect \newline  a.~~$\protect \frac  {d ~\protect \qopname  \relax o{sinh}(t)}{dt} = 1+\protect \frac  {t^2}{2!}+\protect \frac  {t^4}{4!} + \protect \cdots  =\protect \qopname  \relax o{cosh}(t)$ \protect \newline  b.~~$\protect \frac  {d ~\protect \qopname  \relax o{cosh}(t)}{dt} = t+\protect \frac  {t^3}{3!}+\protect \frac  {t^5}{5!} + \protect \cdots  =\protect \qopname  \relax o{sinh}(t)$ \protect \newline  c.~~$ \protect \qopname  \relax o{cosh}^2{t}- \protect \qopname  \relax o{sinh}^2{t} = \left (1+\protect \frac  {t^2}{2!} + \protect \frac  {t^4}{4!}+\protect \frac  {t^6}{6!} + \protect \cdots  \right )^2 - \left (t + \protect \frac  {t^3}{3!} + \protect \frac  {t^5}{5!}+ \protect \cdots  \right )^2 = 1 $ \protect \newline  d.~~$\protect \frac  {1}{0}$ which is a vertical line \protect \newline  \protect \newline  
 {\noindent \protect \bf  Exercise ~2.10.2.2.}  b.) $\protect \frac  {\pi }{8}$\protect \newline   \protect \newline  \protect \newline  
 {\noindent \protect \bf  Exercise ~3.2.0.4.} Any solution to $\protect \frac  {dy}{dx}=xy+x$ can be written as $y=Ce^{\protect \frac  {x^2}{2}}-1$ for some real number $C$. The second DE with initial condition has the solution $$\protect \frac  {1}{2}e^{-y}\left (\protect \qopname  \relax o{sin}(y)-\protect \qopname  \relax o{cos}(y)\right )=-e^{-x}\left (3+2x+x^2\right )+\protect \frac  {5}{2}. $$ \protect \newline  \protect \newline  
 {\noindent \protect \bf  Exercise ~3.6.1.1.} Linear combinations of hyperbolic sine and hyperbolic cosine functions are the only functions that equal their own second derivatives.  \protect \newline  \protect \newline  
 {\noindent \protect \bf  Exercise ~3.6.2.1.} a.~~ $y=3 +x + (a_0-3)e^x$ b.~~ If $y =3 +x + (a_0-3)e^x$ then $\protect \frac  {dy}{dx} = 1 +(a_0-3)e^x$ but $y-x-2 = 3+x+(a_0-3)e^x -x-2 = 1 + (a_0-3)e^x$ So they match.  \protect \newline  \protect \newline  
 {\noindent \protect \bf  Exercise ~3.6.2.2.} a.)~~ The right-hand side $2y+x$ does not factor into a function of $y$ times a function of $x$ so there can be no separation of variables.\protect \newline  b.)~~$y = -\protect \frac  {1}{4} -\protect \frac  {1}{2}x +Ce^{2x}$ \protect \newline   \protect \newline  \protect \newline  
 {\noindent \protect \bf  Exercise ~4.2.3.2.} \textbullet $\protect \qopname  \relax o{ln}(2)=\protect \qopname  \relax o{ln}(2)+0i$ \textbullet $\protect \qopname  \relax o{ln}(-2)=\protect \qopname  \relax o{ln}(2)+\pi i$ \textbullet $\protect \qopname  \relax o{ln}(1+i)=\protect \qopname  \relax o{ln}\left (\protect \sqrt  {2}\right )+i\protect \frac  {\pi }{4}$ \textbullet $\protect \qopname  \relax o{ln}(3-4i)=\protect \qopname  \relax o{ln}(5)+i\protect \qopname  \relax o{arctan}\left (-\protect \frac  {3}{4}\right )$ \protect \newline  \protect \newline  
 {\noindent \protect \bf  Exercise ~4.2.4.2.} The number $(1+i)^{1+i}$ can be written in complex cartesian form as $$\left (e^{\protect \qopname  \relax o{ln}\left (\protect \sqrt  {2}\right )-\protect \frac  {\pi }{4}}\protect \qopname  \relax o{cos}\left (\protect \qopname  \relax o{ln}\left (\protect \sqrt  {2}\right )+\protect \frac  {\pi }{4} \right )\right )+i\left (e^{\protect \qopname  \relax o{ln}\left (\protect \sqrt  {2}\right )-\protect \frac  {\pi }{4}}\protect \qopname  \relax o{sin}\left (\protect \qopname  \relax o{ln}\left (\protect \sqrt  {2}\right )+\protect \frac  {\pi }{4} \right )\right ). $$ \protect \newline  \protect \newline  
 {\noindent \protect \bf  Exercise ~4.3.0.1.} The PFD over the complex numbers is $$\protect \frac  {4-2x^2}{x^3+4x}=\protect \frac  {1}{x}-\protect \frac  {\protect \frac  {3}{2}}{x+2i}-\protect \frac  {\protect \frac  {3}{2}}{x-2i}. $$ \protect \newline  \protect \newline  
 {\noindent \protect \bf  Exercise ~4.3.0.2.} In taking the limit, the logarithmic term will approach zero. Thus $C=\pi /2$. \protect \newline  \protect \newline  
 {\noindent \protect \bf  Exercise ~4.5.1.1.} $z=e^{i (-\pi /4)}, e^{i(3 \pi /4)} = \protect \frac  {\protect \sqrt  {2}}{2} - i \protect \frac  {\protect \sqrt  {2}}{2},-\protect \frac  {\protect \sqrt  {2}}{2} + i \protect \frac  {\protect \sqrt  {2}}{2}$  \protect \newline  \protect \newline  
 {\noindent \protect \bf  Exercise ~4.5.2.1.} a.)~~$-i$ \protect \newline  b.)~~$e^{-\pi }$ \protect \newline  c.)~~$-1$ \protect \newline  d.)~~$\protect \frac  {\protect \sqrt  {3}}{2} + \protect \frac  {i}{2}, -\protect \frac  {\protect \sqrt  {3}}{2} + \protect \frac  {i}{2}, -i$ \protect \newline  e.)~~$\protect \qopname  \relax o{ln}{7} + i \protect \qopname  \relax o{arctan}{\protect \sqrt  {3}/12}$  \protect \newline  \protect \newline  
 {\noindent \protect \bf  Exercise ~4.5.2.2.} b.~~$r\protect \qopname  \relax o{cos}(\theta )+ri \protect \qopname  \relax o{sin}(\theta ) = re^{i \theta }$ This shows that $r$ and $\theta $ represent a radius and an angle.  \protect \newline  \protect \newline  
