\par 
 {\noindent \protect \bf  Exercise ~1.1.0.4.} The limits are 0, $1/\pi $, and -1.  \protect \newline  \protect \newline  
 {\noindent \protect \bf  Exercise ~1.1.1.3.} The limits are $1$, $-2$, and $e$.  \protect \newline  \protect \newline  
 {\noindent \protect \bf  Exercise ~1.1.1.4.} The results are 1, 0, and 1. \protect \newline  \protect \newline  
 {\noindent \protect \bf  Exercise ~1.1.2.2.} Their ratio converges to 3 (both numerically in the table, and analytically as evaluated by LHR). Since this is a nonzero constant, the two functions have the same growth order. \protect \newline  \protect \newline  
 {\noindent \protect \bf  Exercise ~1.1.2.3.} In the first and third, the ratio between $f$ and $g$ seems to grow without bound, so $f$ has larger growth order. In the second, the ratio of $f$ to $g$ seems to always be right around 2. Thus, they have the same growth order. \protect \newline  \protect \newline  
