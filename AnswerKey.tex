\par 
 {\noindent \protect \bf  Exercise ~0.1.0.4.} The totals are 12, -6, and 15. \protect \newline  \protect \newline  
 {\noindent \protect \bf  Exercise ~0.1.0.5.} It is easiest to just expand the sums on both sides and see what the terms look like. For example, in the first case the left-hand side is $\left (ca_j+ca_{j+1}+\protect \cdots  +ca_k\right )$, whereas the right-hand side is $c\left (a_j+a_{j+1}+\protect \cdots  +a_k\right )$. These two expressions are equal, because we can factor the $c$ out of the left-hand side to produce the right-hand side. For the last two summations, think about our discussion of fencepost problems above! \protect \newline  \protect \newline  
 {\noindent \protect \bf  Exercise ~0.2.0.4.}  In Gauss's formula, the first term $a_0$ and the common difference $d$ are both 1. The number of terms is $N$. Plugging these into the Arithmetic Series Formula will produce $N(N+1)/2$. \protect \newline  \protect \newline  
 {\noindent \protect \bf  Exercise ~0.2.0.6.} The totals are 500500, 1501500, 214214, and 245. \protect \newline  \protect \newline  
 {\noindent \protect \bf  Exercise ~0.3.0.3.} The common ratio $r=10$. The first term is 1. The number of terms is 6. Putting this all together in the Geometric Series Formula produces $1\cdot \protect \frac  {1-10^6}{1-10}=\protect \frac  {-99999}{-9}=11111.$ \protect \newline  \protect \newline  
 {\noindent \protect \bf  Exercise ~0.3.0.4.} A finite sum of consecutive powers of two, starting at one, is equal to one less than the next power of two. \protect \newline  \protect \newline  
 {\noindent \protect \bf  Exercise ~0.3.0.7.} Sure! If you further factor $A^2-B^2$ via difference of two squares and further factor $A^3+A^2B+AB^2+B^3$ via grouping, you will end up with the same factorizations. \protect \newline  \protect \newline  
 {\noindent \protect \bf  Exercise ~0.3.1.3.} $\protect \$5 \protect \text  { billion }\cdot \protect \frac  {1-0.8^{13}}{1-0.8}\approx \protect \$23.6 \protect \text  { billion}$ \protect \newline  \protect \newline  
 {\noindent \protect \bf  Exercise ~0.3.1.4.} Try to notice how the summations relate to the very next Fibonacci number, the first one not being summed. \protect \newline  \protect \newline  
