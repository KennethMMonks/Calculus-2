\par 
 {\noindent \protect \bf  Exercise ~2.0.0.1.} \vskip  \protect \newline  \protect \newline  
 {\noindent \protect \bf  Exercise ~2.1.1.1.} $ \intop f'(g(x)) \cdot g'(x) \mathop {}\protect \tmspace  -\thinmuskip {.1667em}\protect \mathtt  {d}x = \intop \left (f\left (g(x)\right )\right )' \mathop {}\protect \tmspace  -\thinmuskip {.1667em}\protect \mathtt  {d}x= f\left (g(x)\right )+C$ \protect \newline  \protect \newline  
 {\noindent \protect \bf  Exercise ~2.1.1.6.} Use the substitutions $u=x^2+x+8, \protect \qopname  \relax o{ln}(x),$ and $-x^2$. In the last case, the $\mathop {}\protect \tmspace  -\thinmuskip {.1667em}\protect \mathtt  {d}u$ term has nothing to cancel the $x$ with! \protect \newline  \protect \newline  
 {\noindent \protect \bf  Exercise ~2.1.2.2.} To have four intervals in the Riemann sum, $\Delta x$ would be 1 while $\Delta u$ would be 2. Thus, the width of each rectangle is getting doubled, since to convert between $u$ and $x$ we use the formula $u=2x+1$. The ``plus one'' merely slides all the rectangles one unit to the right, but it does not stretch their width at all, so it does not affect their area. Thus, the slope of the graph of $u=2x+1$ is the only thing that mattered regarding our conversion between $x$ and $u$. That is to say, the quantity $\mathop {}\protect \tmspace  -\thinmuskip {.1667em}\protect \mathtt  {d}u/ \mathop {}\protect \tmspace  -\thinmuskip {.1667em}\protect \mathtt  {d}x$ gives us the scaling factor. \protect \newline  \protect \newline  
 {\noindent \protect \bf  Exercise ~2.1.2.3.} The definite integral evaluates to roughly 0.95. The horizontal scaling factor at each $x$-coordinate should correspond to the derivative $\mathop {}\protect \tmspace  -\thinmuskip {.1667em}\protect \mathtt  {d}u /\mathop {}\protect \tmspace  -\thinmuskip {.1667em}\protect \mathtt  {d}x$ at each point.  \protect \newline  \protect \newline  
 {\noindent \protect \bf  Exercise ~2.2.1.3.} By factoring out the quantity $(x+1)^{3/2}$, both answers can be brought into the form $(x+1)^{3/2}\left (\protect \frac  {2}{5}x-\protect \frac  {4}{15}\right )+C$. \protect \newline  \protect \newline  
 {\noindent \protect \bf  Exercise ~2.2.2.2.} Use the substitution $u=1-x^2$. \protect \newline  \protect \newline  
 {\noindent \protect \bf  Exercise ~2.2.2.3.} The antiderivative is $x\protect \qopname  \relax o{ln}(x)-x+C$. \protect \newline  \protect \newline  
 {\noindent \protect \bf  Exercise ~2.2.3.5.} The antiderivative is $\protect \frac  {1}{2}\left (\protect \qopname  \relax o{sec}(x)\protect \qopname  \relax o{tan}(x)+\protect \qopname  \relax o{ln}|\protect \qopname  \relax o{sec}(x)+\protect \qopname  \relax o{tan}(x)|\right )+C$. \protect \newline  \protect \newline  
 {\noindent \protect \bf  Exercise ~2.3.0.1.} Use the substitution $u=\protect \sqrt  {x}$ to transform the first integral into $\intop 2u\protect \qopname  \relax o{cos}(u) \mathop {}\protect \tmspace  -\thinmuskip {.1667em}\protect \mathtt  {d}u$. \protect \newline  \protect \newline  
 {\noindent \protect \bf  Exercise ~2.3.0.2.} Choosing $u=\protect \qopname  \relax o{ln}(x)$ will make the logarithm disappear upon differentiation. The opposite choice will not clean up the log. \protect \newline  \protect \newline  
 {\noindent \protect \bf  Exercise ~2.4.0.2.} The antiderivative is $\protect \frac  {1}{3}\protect \qopname  \relax o{sin}^3(x)+C$. \protect \newline  \protect \newline  
 {\noindent \protect \bf  Exercise ~2.4.1.2.} Since seven is odd, when we pulled out one factor of sine, we ended up with the sixth power of sine remaining. Since six is even, we were able to express it as a power of a perfect square of sine, which in turn let us rewrite as cosines using the Pythagorean identity. \protect \newline  \protect \newline  
 {\noindent \protect \bf  Exercise ~2.4.1.3.} The first antiderivative is $-\protect \frac  {1}{3}\protect \qopname  \relax o{cos}^3(x)+\protect \frac  {2}{5}\protect \qopname  \relax o{cos}^5(x)-\protect \frac  {1}{7}\protect \qopname  \relax o{cos}^7(x)+C$. For the second, rewrite as $(1-\protect \qopname  \relax o{sin}^2(x))^4\protect \qopname  \relax o{cos}(x)$ and proceed by letting $u=\protect \qopname  \relax o{sin}(x)$. \protect \newline  \protect \newline  
 {\noindent \protect \bf  Exercise ~2.4.1.4.} Often when trying to show that two antiderivatives are compatible, it is easiest to verify that their difference is a constant.  \protect \newline  \protect \newline  
 {\noindent \protect \bf  Exercise ~2.4.1.4.} The substitution $u=\protect \qopname  \relax o{sin}(x)$ is much cleaner since the other will involve having to expand a binomial to the fifth power. The antiderivative is $\protect \frac  {1}{12}\protect \qopname  \relax o{sin}^{12}(x)-\protect \frac  {1}{14}\protect \qopname  \relax o{sin}^{14}(x)+C$. \protect \newline  \protect \newline  
 {\noindent \protect \bf  Exercise ~2.4.2.1.} The exponent on sine is zero, which is indeed even. Thus both exponents are even in this case. \protect \newline  \protect \newline  
 {\noindent \protect \bf  Exercise ~2.4.2.3.} When all like terms are combined and the one-eighth is distributed, the result is $\protect \frac  {5}{16}x+\protect \frac  {1}{4}\protect \qopname  \relax o{sin}(2x)-\protect \frac  {1}{48}\protect \qopname  \relax o{sin}^3(2x)+\protect \frac  {3}{64}\protect \qopname  \relax o{sin}(4x)+C$. \protect \newline  \protect \newline  
 {\noindent \protect \bf  Exercise ~2.4.2.4.} The antiderivative to $\protect \qopname  \relax o{cos}^6(x)$ came out to \par $$\protect \frac  {5}{16}x+\protect \frac  {1}{4}\protect \qopname  \relax o{sin}(2x)-\protect \frac  {1}{48}\protect \qopname  \relax o{sin}^3(2x)+\protect \frac  {3}{64}\protect \qopname  \relax o{sin}(4x)+C$$ Before we differentiate, first bash everything back down to an ``$x$'' in the argument using double angle identities. This produces \par $$\protect \frac  {5}{16}x+\protect \frac  {1}{2}\protect \qopname  \relax o{sin}(x)\protect \qopname  \relax o{cos}(x)-\protect \frac  {1}{6}\protect \qopname  \relax o{sin}^3(x)\protect \qopname  \relax o{cos}^3(x)+\protect \frac  {3}{16}\protect \qopname  \relax o{sin}(x)\protect \qopname  \relax o{cos}^3(x)-\protect \frac  {3}{16}\protect \qopname  \relax o{sin}^3(x)\protect \qopname  \relax o{cos}(x)+C$$ Factor out a sine and use the Pythagorean Identity to get everything else in terms of cosine. This produces $$\protect \frac  {5}{16}x+\protect \qopname  \relax o{sin}(x)\left (\protect \frac  {5}{16}\protect \qopname  \relax o{cos}(x)+\protect \frac  {5}{24}\protect \qopname  \relax o{cos}^3(x)+\protect \frac  {1}{6}\protect \qopname  \relax o{cos}^5(x)\right )+C$$ Then we differentiate and obtain $$\protect \frac  {5}{16}+\protect \qopname  \relax o{cos}(x)\left (\protect \frac  {5}{16}\protect \qopname  \relax o{cos}(x)+\protect \frac  {5}{24}\protect \qopname  \relax o{cos}^3(x)+\protect \frac  {1}{6}\protect \qopname  \relax o{cos}^5(x)\right )-\protect \qopname  \relax o{sin}^2(x)\left (\protect \frac  {5}{16}+\protect \frac  {5}{8}\protect \qopname  \relax o{cos}^2(x)+\protect \frac  {5}{6}\protect \qopname  \relax o{cos}^4(x)\right )$$ to which we apply the Pythagorean Identity $\protect \qopname  \relax o{sin}^2(x)=1-\protect \qopname  \relax o{cos}^2(x)$ to produce $$\protect \frac  {5}{16}+\protect \qopname  \relax o{cos}(x)\left (\protect \frac  {5}{16}\protect \qopname  \relax o{cos}(x)+\protect \frac  {5}{24}\protect \qopname  \relax o{cos}^3(x)+\protect \frac  {1}{6}\protect \qopname  \relax o{cos}^5(x)\right )-\left (1-\protect \qopname  \relax o{cos}^2(x)\right )\left (\protect \frac  {5}{16}+\protect \frac  {5}{8}\protect \qopname  \relax o{cos}^2(x)+\protect \frac  {5}{6}\protect \qopname  \relax o{cos}^4(x)\right )$$ This will simplify to $\protect \qopname  \relax o{cos}^6(x)$ once you expand and combine like terms.  \protect \newline  \protect \newline  
 {\noindent \protect \bf  Exercise ~2.4.2.5.} For the first, apply the identity $\protect \qopname  \relax o{sin}^2(3x)=\protect \frac  {1-\protect \qopname  \relax o{cos}(6x)}{2}$ and proceed. For the second, notice that $\protect \qopname  \relax o{sin}^4(x)$ can b