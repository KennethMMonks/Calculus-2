\par 
 {\noindent \protect \bf  Exercise ~2.3.1.4.} \textbullet $\protect \frac  {1}{n+1}$ \textbullet $n+1$ \textbullet $(n+2)(n+1)$ \textbullet $(2n+2)(2n+1)$ \protect \newline  \protect \newline  
 {\noindent \protect \bf  Exercise ~2.4.0.3.} Think about what happens if the common difference $d$ is zero and if the common ratio $r$ is 1. \protect \newline  \protect \newline  
 {\noindent \protect \bf  Exercise ~2.4.1.2.} The common ratio $r$ is what we multiply by to get from term to term. Listing out the terms $a_0, a_0r, a_0r^2, a_0r^3,\protect \cdots  $ shows that $a_0r^n$ is the explicit formula. \protect \newline  \protect \newline  
 {\noindent \protect \bf  Exercise ~2.5.0.7.} It is absolutely convergent, since the series of corresponding positive terms is $0.1+0.02+0.004+0.0008+0.00016+\protect \cdots  $ which converges to one-eighth.  \protect \newline  \protect \newline  
 {\noindent \protect \bf  Exercise ~2.6.1.4.} The first one-half would come from the first term itself. But since the total is one, it means the terms $\protect \frac  {1}{4}+\protect \frac  {1}{8}+\protect \frac  {1}{16}+\protect \cdots  $ must themselves total to be the other one-half. Thus, we can try to group terms to form batches that total to one-half, but the second batch uses up all infinitely many remaining terms!  \protect \newline  \protect \newline  
 {\noindent \protect \bf  Exercise ~2.6.1.7.} The first two summands have limits of $\protect \sqrt  {2}$ and 1, respectively. Since these limits are nonzero, the series has no hope of converging and thus diverges. The third summand does approach zero as $n$ goes to infinity, so it gives no information. \protect \newline  \protect \newline  
 {\noindent \protect \bf  Exercise ~2.6.2.2.} \textbullet The summation $\Sigma _{n=1}^\infty \protect \frac  {1}{n^2}$ has no common ratio $r$ and thus is not a geometric series. For example, the first three terms are $1,1/4,$ and $1/9$. Thus, the first two ratios between consecutive terms are $1/4$ and $4/9$, which are not equal. \textbullet The given geometric series has common ratio $r=-1/3$. After taking the absolute value of each term, it becomes the series $18+6+2+\protect \frac  {2}{3}+\protect \frac  {2}{9}+\protect \cdots  $ which still converges as it now has common ratio $r=1/3$. \textbullet It is not possible to build a conditionally convergent geometric series. If we are given a convergent geometric series, then the common ratio $r$ satisfies $|r|<1$. Taking the absolute value of each term in the series might flip the sign on $r$, but it will not change the magnitude. Thus, any convergent geometric series must converge absolutely. \protect \newline  \protect \newline  
 {\noindent \protect \bf  Exercise ~2.6.4.2.} For $p>1$ or $p<1$, one can repeat the corresponding calculations from Example \protect \ref  {IntTest}.\protect \ref  {UsingIntTest}. If $p=1$, the series is the harmonic series, which diverges. \protect \newline  \protect \newline  
 {\noindent \protect \bf  Exercise ~2.6.5.3.} Taking term-by-term absolute values produces the series $1+\protect \frac  {1}{2}+\protect \frac  {1}{4}+\protect \frac  {1}{8}+\protect \cdots  =2$. Since it totals to a finite value, the original series converges absolutely. \protect \newline  \protect \newline  
 {\noindent \protect \bf  Exercise ~2.6.5.4.} \textbullet Since $\protect \qopname  \relax m{lim}_{n\to \infty }\left (-\protect \frac  {1}{2}\right )^n=0$, the No Hope Test gives no information. \textbullet The Integral Test does not apply since the terms are not positive and decreasing. In this case, it is actually even worse than that, as the function $\left (-\protect \frac  {1}{2}\right )^x$ is undefined for all half-integer values of $x$. \textbullet The summand is not of the form $1/n^p$, so the very narrow $p$-Test does not apply. \protect \newline  \protect \newline  
 {\noindent \protect \bf  Exercise ~2.6.5.5.} The first two and last converge by AST. It does not apply to the third. \protect \newline  \protect \newline  
 {\noindent \protect \bf  Exercise ~2.6.5.8.} After three reversals, the bug is within $\protect \frac  {1}{32}$ of its final location. The bug would have to reverse course nine times to be guaranteed by the Alternating Series Error Bound to be within one one-thousandth of its final location.  \protect \newline  \protect \newline  
 {\noindent \protect \bf  Exercise ~2.6.6.3.} The first converges by comparison to $\Sigma \protect \frac  {1}{n^2}$. The second diverges by comparison to $\Sigma \protect \frac  {1}{n^{-1/2}}$. \protect \newline  \protect \newline  
 {\noindent \protect \bf  Exercise ~2.6.6.6.} Use the comparison function $\protect \frac  {1}{n}$ to show the series diverges. \protect \newline  \protect \newline  
 {\noindent \protect \bf  Exercise ~2.6.7.4.} \textbullet Converges, ratio 0. \textbullet No info, ratio 1. \textbullet Converges, ratio 0. \textbullet Diverges, ratio 2. \protect \newline  \protect \newline  
 {\noindent \protect \bf  Exercise ~2.6.8.3.} These series are convergent by DCT against $\protect \frac  {1}{n^2}$, divergent by DCT against $\protect \frac  {1}{n}$, and convergent by DCT against $\protect \frac  {1}{n^2}$. \protect \newline  \protect \newline  
 {\noindent \protect \bf  Exercise ~2.6.9.1.} \textbullet Divergent by NHT or Integral Test. \textbullet Divergent by Integral Test or LCT against $\protect \frac  {1}{n}$. \textbullet Convergent by AST, but only conditionally since the absolute value is the previous summand whose series diverged. \textbullet Absolutely convergent since taking term-by-term absolute value produces a convergent series (which can be shown convergent via LCT with $\protect \frac  {1}{n^3}$). \protect \newline  \protect \newline  
 {\noindent \protect \bf  Exercise ~2.6.10.1.} The black region is one-third of the total square and thus must total to one-third. The infinite series for the black square areas is $\protect \frac  {1}{4}+\protect \frac  {1}{16}+\protect \frac  {1}{32}+\protect \frac  {1}{64}+\protect \cdots  $. NHT and AST give no information here, but all the rest of the tests work to determine convergence! Use $\protect \frac  {1}{n^2}$ as a comparison function for LCT. \protect \newline  \protect \newline  
 {\noindent \protect \bf  Exercise ~2.8.1.1.} a.~~ For $\epsilon = 0.1, n=5,~~~\epsilon = 0.01, n=05,~~~\epsilon = 0.001, n=500$ b.~~ $\protect \frac  {1}{2\epsilon }=N $ c.~~Let $ \epsilon >0$ also, let $N = \protect \frac  {1}{2n}$ and let $n \in \protect \mathbb  {N}$ Assume $n>N$ We wish to show that under these circumstances, the distance from $a_n= \protect \frac  {1}{2n}$ to $L=0$ will be less than $\epsilon $. $\left | a_n-L \right | =\left |\protect \frac  {1}{2n}-0 \right | = \protect \frac  {1}{2n} < \protect \frac  {1}{2N}$ since $N<n$ by our assumptions $ \protect \frac  {1}{2N }=\protect \frac  {1}{2\protect \frac  {1}{2 \epsilon }} = \protect \frac  {1}{\epsilon } = \epsilon $ Thus the terms will be within $\epsilon $ or 0 past the index $\protect \frac  {1}{2 \epsilon }$ , no matter how small $\epsilon $ is chosen. Therefore, $\protect \qopname  \relax m{lim}\limits _{n \to \infty }{\protect \frac  {1}{2n}} = 0$  \protect \newline  \protect \newline  
 {\noindent \protect \bf  Exercise ~2.8.1.2.} $3,375,000$ \protect \newline  \protect \newline  
 {\noindent \protect \bf  Exercise ~2.8.1.3.} a.)~~6 Slices,~b.)~~4.5 Slices,~c.)$\left (\protect \frac  {3}{4} \right )^{20} 8$ Slices,~d.)~~None because as time goes to $\infty $ the number of slices left goes to 0, since $ \left (\protect \frac  {3}{4} \right )^{n} 8$ represents the number of slices left after n minutes and $\protect \qopname  \relax m{lim}\limits _{n \to \infty }{ \left (\protect \frac  {3}{4} \right )^{n} 8} = 0$ slices. \protect \newline  \protect \newline  
 {\noindent \protect \bf  Exercise ~2.8.1.4.} a.) Converges Absolutely by the Ratio Test.\protect \newline  b.) Diverges by the No Hope Test.\protect \newline  c.) Diverges by the No Hope Test.\protect \newline  d.) Converges by the Limit Comparison Test and the p-test.\protect \newline  e.) Diverges by the No Hope Test \protect \newline  f.) Converges Absolutely by the Ratio Test.\protect \newline  g.)Converges Absolutely by the Integral Test or the Ratio Test. \protect \newline  \protect \newline  
 {\noindent \protect \bf  Exercise ~2.8.2.1.} Notice $a_n$ is a geometric sequence with initial term $a_0=2$ and the ratio $r=\protect \frac  {1}{2}$. We can then use the infinite geometric series formula $\Sigma _{n=0}^\infty {a_n}= \protect \frac  {a}{1-r}$ so we have $\Sigma _{n=0}^\infty {2\cdot \left (\protect \frac  {1}{2}\right )^n} = 2 \cdot \protect \frac  {1}{1-\protect \frac  {1}{2}} = 2 \protect \frac  {1}{\protect \frac  {1}{2}} = 2 \cdot 2 = 4.$ Thus, the sequence of partial sums converges to $4$. \protect \newline  \protect \newline  
 {\noindent \protect \bf  Exercise ~2.8.2.2.} a.) $\protect \qopname  \relax m{lim}\limits _{n \to \infty }{a_n} = L \DOTSB \Leftarrow \protect \joinrel  \Rightarrow $ For all $\epsilon >0$ there exists N such that for all $n>N, |a_n-L| < \epsilon $ \protect \newline  b.) $\protect \frac  {1}{3}$\protect \newline  c.) Let $\epsilon >0$ then choose $N=\protect \frac  {1}{9n}-\protect \frac  {1}{3}$ Then choose $n \in \protect \mathbb  {N}$ with $n>N$. We now show that any $a_n$ for such n is no more than $\epsilon $ away from $\protect \frac  {1}{3}$. \protect \newline  $\left | \protect \frac  {n}{3n+1} - \protect \frac  {1}{3} \right |=\left | \protect \frac  {3n}{3(3n+1)} - \protect \frac  {3n+1}{3(3n+1)} \right |=\left | \protect \frac  {-1}{3(3n+1)} \right |= \protect \frac  {1}{3(3n+1)}<\protect \frac  {1}{3(3N+1)}$ \protect \newline  note here we made the denominator smaller by introducing N for n\protect \newline  $\protect \frac  {1}{3(3N+1)}=\protect \frac  {1}{3\left (3\left (\protect \frac  {1}{9 \epsilon }-\protect \frac  {1}{3} \right )+1\right )}=\protect \frac  {1}{3\left (\left (\protect \frac  {1}{3 \epsilon }-1 \right )+1\right )}=\protect \frac  {1}{3\left (\protect \frac  {1}{3 \epsilon } \right )}=\protect \frac  {1}{\protect \frac  {1}{\epsilon }}=\epsilon $  \protect \newline  \protect \newline  
 {\noindent \protect \bf  Exercise ~2.8.2.3.} a.)$\protect \qopname  \relax m{lim}\limits _{n \to \infty }{(-1)^n \protect \frac  {1}{n!}} = 0$ So the no hope test gives no information since it requires the limit to be not equal to zero. \protect \newline  b.)It converges absolutely by the Alternating Series Test.\protect \newline  c.)It converges absolutely by the Ratio Test  \protect \newline  \protect \newline  
 {\noindent \protect \bf  Exercise ~2.8.2.4.} a.)$a_0 = 0, a_1=1, a_2=1+3\cdot 2^2-3\cdot 2+1 = 8, a_3=8+3\cdot 3^2-3\cdot 3+1 = 27, a_4=1+3\cdot 4^2-3\cdot 4+1 = 64 $\protect \newline  b.)$a_n = n^3$\protect \newline  c.)It diverges by the No Hope Test  \protect \newline  \protect \newline  
