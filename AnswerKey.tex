\par 
 {\noindent \protect \bf  Exercise ~2.0.0.1.} \vskip  \protect \newline  \protect \newline  
 {\noindent \protect \bf  Exercise ~2.1.1.1.} $ \intop f'(g(x)) \cdot g'(x) \mathop {}\protect \tmspace  -\thinmuskip {.1667em}\protect \mathtt  {d}x = \intop \left (f\left (g(x)\right )\right )' \mathop {}\protect \tmspace  -\thinmuskip {.1667em}\protect \mathtt  {d}x= f\left (g(x)\right )+C$ \protect \newline  \protect \newline  
 {\noindent \protect \bf  Exercise ~2.1.1.6.} Use the substitutions $u=x^2+x+8, \protect \qopname  \relax o{ln}(x),$ and $-x^2$. In the last case, the $\mathop {}\protect \tmspace  -\thinmuskip {.1667em}\protect \mathtt  {d}u$ term has nothing to cancel the $x$ with! \protect \newline  \protect \newline  
 {\noindent \protect \bf  Exercise ~2.1.2.2.} To have four intervals in the Riemann sum, $\Delta x$ would be 1 while $\Delta u$ would be 2. Thus, the width of each rectangle is getting doubled, since to convert between $u$ and $x$ we use the formula $u=2x+1$. The ``plus one'' merely slides all the rectangles one unit to the right, but it does not stretch their width at all, so it does not affect their area. Thus, the slope of the graph of $u=2x+1$ is the only thing that mattered regarding our conversion between $x$ and $u$. That is to say, the quantity $\mathop {}\protect \tmspace  -\thinmuskip {.1667em}\protect \mathtt  {d}u/ \mathop {}\protect \tmspace  -\thinmuskip {.1667em}\protect \mathtt  {d}x$ gives us the scaling factor. \protect \newline  \protect \newline  
 {\noindent \protect \bf  Exercise ~2.1.2.3.} The definite integral evaluates to roughly 0.95. The horizontal scaling factor at each $x$-coordinate should correspond to the derivative $\mathop {}\protect \tmspace  -\thinmuskip {.1667em}\protect \mathtt  {d}u /\mathop {}\protect \tmspace  -\thinmuskip {.1667em}\protect \mathtt  {d}x$ at each point.  \protect \newline  \protect \newline  
 {\noindent \protect \bf  Exercise ~2.2.1.3.} By factoring out the quantity $(x+1)^{3/2}$, both answers can be brought into the form $(x+1)^{3/2}\left (\protect \frac  {2}{5}x-\protect \frac  {4}{15}\right )+C$. \protect \newline  \protect \newline  
 {\noindent \protect \bf  Exercise ~2.2.2.2.} Use the substitution $u=1-x^2$. \protect \newline  \protect \newline  
 {\noindent \protect \bf  Exercise ~2.2.2.3.} The antiderivative is $x\protect \qopname  \relax o{ln}(x)-x+C$. \protect \newline  \protect \newline  
 {\noindent \protect \bf  Exercise ~2.2.3.5.} The antiderivative is $\protect \frac  {1}{2}\left (\protect \qopname  \relax o{sec}(x)\protect \qopname  \relax o{tan}(x)+\protect \qopname  \relax o{ln}|\protect \qopname  \relax o{sec}(x)+\protect \qopname  \relax o{tan}(x)|\right )+C$. \protect \newline  \protect \newline  
 {\noindent \protect \bf  Exercise ~2.3.0.1.} Use the substitution $u=\protect \sqrt  {x}$ to transform the first integral into $\intop 2u\protect \qopname  \relax o{cos}(u) \mathop {}\protect \tmspace  -\thinmuskip {.1667em}\protect \mathtt  {d}u$. \protect \newline  \protect \newline  
 {\noindent \protect \bf  Exercise ~2.3.0.2.} Choosing $u=\protect \qopname  \relax o{ln}(x)$ will make the logarithm disappear upon differentiation. The opposite choice will not clean up the log. \protect \newline  \protect \newline  
 {\noindent \protect \bf  Exercise ~2.4.0.2.} The antiderivative is $\protect \frac  {1}{3}\protect \qopname  \relax o{sin}^3(x)+C$. \protect \newline  \protect \newline  
 {\noindent \protect \bf  Exercise ~2.4.1.2.} Since seven is odd, when we pulled out one factor of sine, we ended up with the sixth power of sine remaining. Since six is even, we were able to express it as a power of a perfect square of sine, which in turn let us rewrite as cosines using the Pythagorean identity. \protect \newline  \protect \newline  
 {\noindent \protect \bf  Exercise ~2.4.1.3.} The first antiderivative is $-\protect \frac  {1}{3}\protect \qopname  \relax o{cos}^3(x)+\protect \frac  {2}{5}\protect \qopname  \relax o{cos}^5(x)-\protect \frac  {1}{7}\protect \qopname  \relax o{cos}^7(x)+C$. For the second, rewrite as $(1-\protect \qopname  \relax o{sin}^2(x))^4\protect \qopname  \relax o{cos}(x)$ and proceed by letting $u=\protect \qopname  \relax o{sin}(x)$. \protect \newline  \protect \newline  
 {\noindent \protect \bf  Exercise ~2.4.1.4.} Often when trying to show that two antiderivatives are compatible, it is easiest to verify that their difference is a constant.  \protect \newline  \protect \newline  
 {\noindent \protect \bf  Exercise ~2.4.1.4.} The substitution $u=\protect \qopname  \relax o{sin}(x)$ is much cleaner since the other will involve having to expand a binomial to the fifth power. The antiderivative is $\protect \frac  {1}{12}\protect \qopname  \relax o{sin}^{12}(x)-\protect \frac  {1}{14}\protect \qopname  \relax o{sin}^{14}(x)+C$. \protect \newline  \protect \newline  
 {\noindent \protect \bf  Exercise ~2.4.2.1.} The exponent on sine is zero, which is indeed even. Thus both exponents are even in this case. \protect \newline  \protect \newline  
 {\noindent \protect \bf  Exercise ~2.4.2.3.} When all like terms are combined and the one-eighth is distributed, the result is $\protect \frac  {5}{16}x+\protect \frac  {1}{4}\protect \qopname  \relax o{sin}(2x)-\protect \frac  {1}{48}\protect \qopname  \relax o{sin}^3(2x)+\protect \frac  {3}{64}\protect \qopname  \relax o{sin}(4x)+C$. \protect \newline  \protect \newline  
 {\noindent \protect \bf  Exercise ~2.4.2.4.} The antiderivative to $\protect \qopname  \relax o{cos}^6(x)$ came out to \par $$\protect \frac  {5}{16}x+\protect \frac  {1}{4}\protect \qopname  \relax o{sin}(2x)-\protect \frac  {1}{48}\protect \qopname  \relax o{sin}^3(2x)+\protect \frac  {3}{64}\protect \qopname  \relax o{sin}(4x)+C$$ Before we differentiate, first bash everything back down to an ``$x$'' in the argument using double angle identities. This produces \par $$\protect \frac  {5}{16}x+\protect \frac  {1}{2}\protect \qopname  \relax o{sin}(x)\protect \qopname  \relax o{cos}(x)-\protect \frac  {1}{6}\protect \qopname  \relax o{sin}^3(x)\protect \qopname  \relax o{cos}^3(x)+\protect \frac  {3}{16}\protect \qopname  \relax o{sin}(x)\protect \qopname  \relax o{cos}^3(x)-\protect \frac  {3}{16}\protect \qopname  \relax o{sin}^3(x)\protect \qopname  \relax o{cos}(x)+C$$ Factor out a sine and use the Pythagorean Identity to get everything else in terms of cosine. This produces $$\protect \frac  {5}{16}x+\protect \qopname  \relax o{sin}(x)\left (\protect \frac  {5}{16}\protect \qopname  \relax o{cos}(x)+\protect \frac  {5}{24}\protect \qopname  \relax o{cos}^3(x)+\protect \frac  {1}{6}\protect \qopname  \relax o{cos}^5(x)\right )+C$$ Then we differentiate and obtain $$\protect \frac  {5}{16}+\protect \qopname  \relax o{cos}(x)\left (\protect \frac  {5}{16}\protect \qopname  \relax o{cos}(x)+\protect \frac  {5}{24}\protect \qopname  \relax o{cos}^3(x)+\protect \frac  {1}{6}\protect \qopname  \relax o{cos}^5(x)\right )-\protect \qopname  \relax o{sin}^2(x)\left (\protect \frac  {5}{16}+\protect \frac  {5}{8}\protect \qopname  \relax o{cos}^2(x)+\protect \frac  {5}{6}\protect \qopname  \relax o{cos}^4(x)\right )$$ to which we apply the Pythagorean Identity $\protect \qopname  \relax o{sin}^2(x)=1-\protect \qopname  \relax o{cos}^2(x)$ to produce $$\protect \frac  {5}{16}+\protect \qopname  \relax o{cos}(x)\left (\protect \frac  {5}{16}\protect \qopname  \relax o{cos}(x)+\protect \frac  {5}{24}\protect \qopname  \relax o{cos}^3(x)+\protect \frac  {1}{6}\protect \qopname  \relax o{cos}^5(x)\right )-\left (1-\protect \qopname  \relax o{cos}^2(x)\right )\left (\protect \frac  {5}{16}+\protect \frac  {5}{8}\protect \qopname  \relax o{cos}^2(x)+\protect \frac  {5}{6}\protect \qopname  \relax o{cos}^4(x)\right )$$ This will simplify to $\protect \qopname  \relax o{cos}^6(x)$ once you expand and combine like terms.  \protect \newline  \protect \newline  
 {\noindent \protect \bf  Exercise ~2.4.2.5.} For the first, apply the identity $\protect \qopname  \relax o{sin}^2(3x)=\protect \frac  {1-\protect \qopname  \relax o{cos}(6x)}{2}$ and proceed. For the second, notice that $\protect \qopname  \relax o{sin}^4(x)$ can be rewritten as $\left (\protect \qopname  \relax o{sin}^2(x)\right )^2$, after which the half-angle identity can be applied. \protect \newline  \protect \newline  
 {\noindent \protect \bf  Exercise ~2.5.1.3.} First apply all the product and chain rules to reach the expression $$\protect \frac  {3}{\protect \sqrt  {1-\protect \frac  {x^2}{4}}}+4\protect \sqrt  {1-\protect \frac  {x^2}{4}}+\protect \frac  {-x^2}{\protect \sqrt  {1-\protect \frac  {x^2}{4}}}+\protect \sqrt  {1-\protect \frac  {x^2}{4}}\left (1-\protect \frac  {3}{2}x^2\right )+\protect \frac  {-x}{4\protect \sqrt  {1-\protect \frac  {x^2}{4}}}\left (x-\protect \frac  {x^3}{2}\right ) $$ Put all terms over the common denominator $\protect \sqrt  {4-x^2}$ and combine like terms in the numerator. Notice the numerator becomes $\left (4-x^2\right )^2$ and then reduce for the win! \protect \newline  \protect \newline  
 {\noindent \protect \bf  Exercise ~2.5.1.4.} The antiderivative is $2^{18}\left (\protect \frac  {\left (1-x^2/16\right )^{9/2}}{9}-\protect \frac  {\left (1-x^2/16\right )^{7/2}}{7}\right )+C$ \protect \newline  \protect \newline  
 {\noindent \protect \bf  Exercise ~2.5.2.3.} Exercise \protect \ref  {reappear}.\protect \ref  {seccubed} will be helpful! The antiderivative is $\protect \frac  {x\protect \sqrt  {x^2-4}}{2}-2\protect \qopname  \relax o{ln}|x+\protect \sqrt  {x^2-4}|+C$. \protect \newline  \protect \newline  
 {\noindent \protect \bf  Exercise ~2.5.4.3.} The antiderivative is $-\protect \frac  {1}{5}\protect \frac  {2x+1}{x^2+x-1}+\protect \frac  {4\protect \sqrt  {5}}{25}\protect \qopname  \relax o{ln}\left (\protect \frac  {2x+1+\protect \sqrt  {5}}{2\protect \sqrt  {x^2+x-1}}\right )+C$. Note that one can expand using properties of logarithms and then rename $C$ as $C-\protect \frac  {4\protect \sqrt  {5}}{25}\protect \qopname  \relax o{ln}(2)$ since it is anyhow just an arbitrary constant. Thus, we can slightly clean up the answer to become $-\protect \frac  {1}{5}\protect \frac  {2x+1}{x^2+x-1}+\protect \frac  {4\protect \sqrt  {5}}{25}\protect \qopname  \relax o{ln}\left (2x+1+\protect \sqrt  {5}\right )-\protect \frac  {2\protect \sqrt  {5}}{25}\protect \qopname  \relax o{ln}\left (x^2+x-1\right )+C.$ \protect \newline  \protect \newline  
 {\noindent \protect \bf  Exercise ~2.6.1.5.} Using properties of logarithms, both answers should be able to be put in the form $\protect \qopname  \relax o{ln}\left | \protect \sqrt  {\protect \frac  {x-1}{x+1}}\right |+C$ \protect \newline  \protect \newline  
 {\noindent \protect \bf  Exercise ~2.6.3.1.} \textbullet The function $\protect \frac  {1}{x^2-9x+20}$ has $\protect \qopname  \relax o{ln}\left |\protect \frac  {x-5}{x-4}\right |+C$ as its antiderivative. \textbullet The factorization $x^4-9=\left (x^2+3\right )\left (x-\protect \sqrt  {3}\right )\left (x+\protect \sqrt  {3}\right )$ will produce the following setup: $$\protect \frac  {1}{x^4-9}=\protect \frac  {Ax+B}{x^2+3}+\protect \frac  {C}{x-\protect \sqrt  {3}}+\protect \frac  {D}{x+\protect \sqrt  {3}} $$ in which you can then solve for the coefficients and antidifferentiate. \textbullet The function $\protect \frac  {x^4}{x^2+1}$ has an irreducible quadratic for a denominator. However, the degree of the numerator is not smaller than the degree of the denominator. Thus, polynomial long division is the only step of PFD that is required in this case. \textbullet The antiderivative of $\protect \frac  {2}{x^5+2x3+x}$ is $$2\protect \qopname  \relax o{ln}|x|-\protect \qopname  \relax o{ln}\left | x^2+1\right |+\protect \frac  {1}{x^2+1}$$ \textbullet The PFD will produce $$ \protect \frac  {x-2}{x^3+x^2+3x-5}=\protect \frac  {-\protect \frac  {1}{8}}{x-1}+\protect \frac  {\protect \frac  {1}{8}x+\protect \frac  {11}{8}}{x^2+2x+5}$$ While the first term is easy to integrate, the second is quite tricky! To hack through it, split it as follows: $$\protect \frac  {\protect \frac  {1}{8}x+\protect \frac  {11}{8}}{x^2+2x+5}=\protect \frac  {\protect \frac  {1}{8}x+\protect \frac  {1}{8}}{x^2+2x+5}+\protect \frac  {\protect \frac  {10}{8}}{x^2+2x+5} $$ The first fraction can then be integrated via $u$-sub, while the second can be done via trig sub after completing the square on the denominator. \protect \newline  \protect \newline  
 {\noindent \protect \bf  Exercise ~2.6.3.2.} For $\protect \frac  {1}{x^4-9x^2}$, keep in mind that $x^2$ is not an irreducible quadratic factor but rather a repeated linear factor. The PFD and integration will produce $$\protect \frac  {1}{9x}+\protect \frac  {1}{54}\protect \qopname  \relax o{ln}\left |\protect \frac  {x-3}{x+3}\right |+C $$  \protect \newline  \protect \newline  
 {\noindent \protect \bf  Exercise ~2.8.1.1.} $\protect \qopname  \relax o{ln}\left (1+x\right )+C$ \protect \newline  \protect \newline  
 {\noindent \protect \bf  Exercise ~2.8.1.2.} $x+ -2\protect \sqrt  {x}+2\protect \qopname  \relax o{ln}{|\protect \sqrt  {x}+1|}+C$ \protect \newline  \protect \newline  
 {\noindent \protect \bf  Exercise ~2.8.1.3.} $\protect \qopname  \relax o{ln}\left | \protect \frac  { 2+\protect \sqrt  {3}}{\protect \sqrt  {2} +1} \right |$ \protect \newline  \protect \newline  
 {\noindent \protect \bf  Exercise ~2.8.1.4.} $\protect \frac  {1}{2}\protect \qopname  \relax o{ln}{|x-1|}-\protect \frac  {1}{2}\protect \qopname  \relax o{ln}{|x+1|} +\protect \frac  {1}{x} +C$ \protect \newline  \protect \newline  
 {\noindent \protect \bf  Exercise ~2.8.2.1.} $-\protect \frac  {\protect \qopname  \relax o{cos}^{18}{x}}{18}+ \protect \frac  {\protect \qopname  \relax o{cos}^{16}{x}}{8} - \protect \frac  {\protect \qopname  \relax o{cos}^{14}{x}}{14} + C$ \protect \newline  \protect \newline  
 {\noindent \protect \bf  Exercise ~2.8.2.2.} The antiderivative is $-\protect \frac  {1}{2}\left (\protect \qopname  \relax o{csc}(x)\protect \qopname  \relax o{cot}(x)+\protect \qopname  \relax o{ln}\left |\protect \qopname  \relax o{csc}(x)+\protect \qopname  \relax o{cot}(x)\right |\right )+C$ \protect \newline  \protect \newline  
 {\noindent \protect \bf  Exercise ~2.8.2.3.} $\protect \frac  {1}{8} \protect \qopname  \relax o{ln}\left | \protect \frac  {x-4}{x+4} \right | + C$ \protect \newline  \protect \newline  
 {\noindent \protect \bf  Exercise ~2.8.2.4.} \textbullet $ \protect \frac  {x^3}{x^3-3x^2+4} = 1 + \protect \frac  { -\protect \frac  {1}{9}}{x+1} + \protect \frac  {\protect \frac  {28}{9}}{x-2} + \protect \frac  {\protect \frac  {8}{3}}{(x-2)^2} $ \protect \newline  \textbullet $ \intop {1 + \protect \frac  { -\protect \frac  {1}{9}}{x+1} + \protect \frac  {\protect \frac  {28}{9}}{x-2} + \protect \frac  {\protect \frac  {8}{3}}{(x-2)^2} \mathop {}\protect \tmspace  -\thinmuskip {.1667em}\protect \mathtt  {d}x} = x -\protect \frac  {1}{9} \protect \qopname  \relax o{ln}{|x+1|} + \protect \frac  {28}{9} \protect \qopname  \relax o{ln}{|x-2|} - \protect \frac  {8}{3} \protect \frac  {1}{(x-2)} + C$  \protect \newline  \protect \newline  
 {\noindent \protect \bf  Exercise ~2.8.2.5.} $\protect \frac  {1}{4}\protect \qopname  \relax o{sec}^3{x}\protect \qopname  \relax o{tan}{x} +\protect \frac  {3}{8} \protect \qopname  \relax o{sec}{x}\protect \qopname  \relax o{tan}{x}+\protect \frac  {3}{8} \protect \qopname  \relax o{ln}{|\protect \qopname  \relax o{sec}{x}+\protect \qopname  \relax o{tan}{x}|}+C$ \protect \newline  \protect \newline  
 {\noindent \protect \bf  Exercise ~3.1.0.4.} The limits are 0, $1/\pi $, and -1.  \protect \newline  \protect \newline  
 {\noindent \protect \bf  Exercise ~3.1.1.3.} The limits are $1$, $-2$, and $e$.  \protect \newline  \protect \newline  
 {\noindent \protect \bf  Exercise ~3.1.1.4.} The results are 1, 0, and 1. \protect \newline  \protect \newline  
 {\noindent \protect \bf  Exercise ~3.1.2.2.} Their ratio converges to 3 (both numerically in the table, and analytically as evaluated by LHR). Since this is a nonzero constant, the two functions have the same growth order. \protect \newline  \protect \newline  
 {\noindent \protect \bf  Exercise ~3.1.2.3.} In the first and third, the ratio between $f$ and $g$ seems to grow without bound, so $f$ has larger growth order. In the second, the ratio of $f$ to $g$ seems to always be right around 2. Thus, they have the same growth order. \protect \newline  \protect \newline  
 {\noindent \protect \bf  Exercise ~3.2.1.8.} The integrals evaluate to $2\protect \sqrt  {2},\infty ,\infty ,$ and $\infty $. \protect \newline  \protect \newline  
 {\noindent \protect \bf  Exercise ~3.2.2.4.} \textbullet The area under $xe^{-x^2}$ from zero to $\infty $ is $\protect \frac  {1}{2}$. \textbullet Splitting into two integrals at $x=0$ produces one of area one-half and one of area negative one-half, so the total integral is zero. \textbullet After applying IBP with $u=x$ and $\mathop {}\protect \tmspace  -\thinmuskip {.1667em}\protect \mathtt  {d}v = xe^{-x^2}\mathop {}\protect \tmspace  -\thinmuskip {.1667em}\protect \mathtt  {d}x$, one obtains $\protect \frac  {\protect \sqrt  {\pi }}{2}$ as the area under the curve. \textbullet The area under $\protect \frac  {1}{x\protect \qopname  \relax o{ln}(x)}$ from 2 to $\infty $ is infinite. \textbullet The area under $\protect \frac  {1}{x\left (\protect \qopname  \relax o{ln}(x)\right )^2}$ from 2 to $\infty $ is $\protect \frac  {1}{\protect \qopname  \relax o{ln}(2)}$. \textbullet An improper integral is defined using a limit, and here the limit does not exist, as the area keeps going up and down by the same amount forever. \protect \newline  \protect \newline  
 {\noindent \protect \bf  Exercise ~3.3.2.1.} \textbullet The curves $y=x^3+x^2-x-1$ and $y=x^3-x^2-x+1$ intersect on the $x$ axis at -1 and 1 and have area 8/3 between them. \textbullet The area inside the unit circle but above the line $y=1/2$ is $\pi /3-\protect \sqrt  {3}/4$. \textbullet Notice graphically that the curves intersect at $x=\pm \pi /4$. The area between curves is $\pi /4-\protect \qopname  \relax o{ln}(2)$. \protect \newline  \protect \newline  
 {\noindent \protect \bf  Exercise ~3.4.1.2.} The exact arc length is $\protect \frac  {2\protect \sqrt  {5}+\protect \qopname  \relax o{ln}\left | 2+\protect \sqrt  {5}\right |}{4}$. \protect \newline  \protect \newline  
 {\noindent \protect \bf  Exercise ~3.4.3.1.} The length of the graph of the natural logarithm from (1,0) to (e,1) is $$\protect \sqrt  {e^2+1}+\protect \frac  {1}{2}\protect \qopname  \relax o{ln}\left | \protect \frac  {\protect \sqrt  {e^2+1}-1}{\protect \sqrt  {e^2+1}+1}\right |-\protect \sqrt  {2}+\protect \frac  {1}{2}\protect \qopname  \relax o{ln}\left | \protect \frac  {\protect \sqrt  {2}+1}{\protect \sqrt  {2}-1}\right | $$ which is roughly 2.003497. Also, notice that the natural exponential function is just the inverse of the natural logarithm; think about what this means regarding arc length! \protect \newline  \protect \newline  
 {\noindent \protect \bf  Exercise ~3.4.4.1.} The two-frusta approximation is $$\protect \frac  {\pi }{8}\left (\protect \sqrt  {5}+3\protect \sqrt  {13} \right )\approx 5.126 $$ The exact value of the surface area is $$\protect \frac  {\pi }{6}\left (5\protect \sqrt  {5}-1 \right )\approx 5.3304 $$ which is just slightly larger, as one would expect.  \protect \newline  \protect \newline  
 {\noindent \protect \bf  Exercise ~3.4.5.3.} Notice that if you turn the pyramid sideways, you can get the 2D side view to be almost exactly the same as we had for the cone! The volume is $V=\protect \frac  {1}{3}r^2h$. \protect \newline  \protect \newline  
 {\noindent \protect \bf  Exercise ~3.4.5.4.} The volume is $V=\protect \frac  {1}{6}abc$. \protect \newline  \protect \newline  
 {\noindent \protect \bf  Exercise ~3.4.5.6.} The circular cross section has equation $x^2+y^2=1$. If you solve for the $y$ coordinate, you'll have a function for the radius of a circular cross section at position $x$. This formula can be integrated to produce the volume $V=\protect \frac  {4}{3}\pi r^3$. \protect \newline  \protect \newline  
 {\noindent \protect \bf  Exercise ~3.4.5.8.} The parabolic bowl has volume $\pi /2$ and occupies exactly fifty percent of the cylinder it sits in! \protect \newline  \protect \newline  
 {\noindent \protect \bf  Exercise ~3.4.5.11.} The volume estimate with a single cylinder is $2\pi $. To get the heights of the six cylindrical shells, you'll need to use the fact that $x^2+y^2=1$ for every point on the boundary of the circle. With six shells, the volume estimate is $\pi \cdot \protect \frac  {\protect \sqrt  {35} + 12 \protect \sqrt  {2} + 15 \protect \sqrt  {3} + 14 \protect \sqrt  {5} + 9 \protect \sqrt  {11}}{108}\approx 1.018\pi $. The first is an overestimate, whereas the second is an underestimate. \protect \newline  \protect \newline  
 {\noindent \protect \bf  Exercise ~3.4.5.12.} The function $g(x)=\protect \sqrt  {1-x^2}$ represents just the QI $y$-coordinate. It needs to be doubled to represent the height of the shell since the each shell extends the same vertical distance into QIII. Once the integral is evaluated, it will return the exact volume $\protect \frac  {4}{3}\pi $. \protect \newline  \protect \newline  
 {\noindent \protect \bf  Exercise ~3.4.6.1.} The volume of the torus is $$V=2\pi ^2Rr^2 $$ and the surface area is $$SA=4\pi ^2Rr. $$ \protect \newline  \protect \newline  
 {\noindent \protect \bf  Exercise ~3.5.1.3.} The sine gumdrop is just a translation $\pi /2$ units to the right of the cosine gumdrop. So, we would expect the center of mass to have the same $y$-coordinate but have an $x$-coordinate that is $\pi /2$ units larger. Indeed, when computed with the moment integrals, we get $\left (\protect \mathaccentV {bar}016{x},\protect \mathaccentV {bar}016{y}\right )=\left (\pi /2,\pi /8\right )$.  \protect \newline  \protect \newline  
 {\noindent \protect \bf  Exercise ~3.5.3.1.} The diagonals have the equations $$ y=\protect \frac  {b}{a}x \protect \text  { and } y=\protect \frac  {b-2c}{a}x+c$$ with intersection point $(a/2,b/2)$, which is also the center of mass of the region. \protect \newline  \protect \newline  
 {\noindent \protect \bf  Exercise ~3.5.4.2.} The coordinates of the vertices are (0,0), (0,$c$), and ($a,b$). Two of the medians are $$y=\protect \frac  {b+c}{a}x \protect \text  { and } y=\protect \frac  {2b-c}{2a}x+\protect \frac  {c}{2}$$ and their intersection point (and center of mass of the triangle) is $\left (\protect \frac  {a}{3},\protect \frac  {b+c}{3}\right )$. \protect \newline  \protect \newline  
 {\noindent \protect \bf  Exercise ~3.5.5.1.} The center of mass is $\left ( 0, \protect \frac  {4r}{3\pi }\right )$. \protect \newline  \protect \newline  
 {\noindent \protect \bf  Exercise ~3.7.1.1.} a.~~ $e^{x}>>x^2$ b.~~ $e^{x}>>x^3$ c.~~$ e^{x}>>x^4$ d.~~The above calculations demonstrate that if you compare $p(x)$ to $e^x$ you will have n iterations of LHR resulting in $\protect \qopname  \relax m{lim}\limits _{x \to \infty }{\protect \frac  {p(x)}{e^x}}=\protect \qopname  \relax m{lim}\limits _{x \to \infty }{\protect \frac  {n!}{e^x}}=0$ Thus $e^x$ has larger growth order than any polynomial.  \protect \newline  \protect \newline  
 {\noindent \protect \bf  Exercise ~3.7.1.2.} $V=9$ \protect \newline  \protect \newline  
 {\noindent \protect \bf  Exercise ~3.7.1.3.} $V=\protect \frac  {\pi }{4} -\protect \frac  {\protect \sqrt  {2}}{6}$ \protect \newline  \protect \newline  
 {\noindent \protect \bf  Exercise ~3.7.1.4.} The area is infinite! \protect \newline  \protect \newline  
 {\noindent \protect \bf  Exercise ~3.7.1.5.} $V=\protect \frac  {4}{3} \pi r^3 $ \protect \newline  \protect \newline  
 {\noindent \protect \bf  Exercise ~3.7.1.6.} $V = \intop _{-r}^{r}{\pi (r^2-x^2) dx} = \protect \frac  {4}{3} \pi r^3 $ \protect \newline  \protect \newline  
 {\noindent \protect \bf  Exercise ~3.7.1.7.} $\protect \mathaccentV {bar}016{x}=\protect \frac  {3}{2}, \protect \mathaccentV {bar}016{y}= \protect \frac  {18}{5} $ \protect \newline  \protect \newline  
 {\noindent \protect \bf  Exercise ~3.7.2.1.} a.) $\protect \qopname  \relax m{lim}\limits _{x\rightarrow \infty }{\protect \frac  {\protect \qopname  \relax o{ln}{x}}{\protect \sqrt  {x}}}=0$ \protect \newline  b.) $\protect \qopname  \relax m{lim}\limits _{x\rightarrow 0^+}{\protect \frac  {\protect \qopname  \relax o{ln}{x}}{\protect \sq