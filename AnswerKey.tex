\par 
 {\noindent \protect \bf  Exercise ~1.1.0.2.} Think about an integral of the form $\intop _{x=0}^{x=\infty }f(x)\mathop {}\protect \tmspace  -\thinmuskip {.1667em}\protect \mathtt  {d}x $. How does one handle that infinity in the bounds? \protect \newline  \protect \newline  
 {\noindent \protect \bf  Exercise ~1.1.1.1.} The sequence is $a_n=\protect \frac  {1}{2^{n+1}}$. Since this is a geometric sequence, the finite geometric series formula can be applied to then find the sequence of partial sums $A_N$. \protect \newline  \protect \newline  
 {\noindent \protect \bf  Exercise ~1.1.1.2.} It ends up one-third of a meter forward from where it started. \protect \newline  \protect \newline  
 {\noindent \protect \bf  Exercise ~1.1.2.4.} Yes, the series is geometric with initial term $\protect \frac  {3^5}{2^{11}}$ and common ratio $3/4$. The infinite series totals to $ \protect \frac  {3^5}{2^9}$. \protect \newline  \protect \newline  
 {\noindent \protect \bf  Exercise ~1.1.2.5.} Think about what the value of $r$ would be for that series. What restrictions did we have on $r$ in the statement of the infinite geometric series formula? \protect \newline  \protect \newline  
 {\noindent \protect \bf  Exercise ~1.1.2.6.} $1+2\protect \frac  {5}{8}+2\left (\protect \frac  {5}{8}\right )^2+2\left (\protect \frac  {5}{8}\right )^3+\protect \cdots  =1+2\protect \frac  {5/8}{1-5/8}=1+2\protect \frac  {5/8}{3/8}=13/3=4.\protect \overline  {3}$ meters. \protect \newline  \protect \newline  
 {\noindent \protect \bf  Exercise ~1.1.2.7.} The partial sums are $A_N=2(N+1)$. The infinite series is the limit of $A_N$ as $N$ goes to infinity, which here is clearly again infinity. Thus, the infinite series diverges. \protect \newline  \protect \newline  
 {\noindent \protect \bf  Exercise ~1.1.2.8.} The partial sums are $$A_N=\protect \frac  {N+1}{N+3}$$ for an infinite sum of 1. \protect \newline  \protect \newline  
 {\noindent \protect \bf  Exercise ~1.1.2.9.} The infinite series $\Sigma _{n=0}^{\infty }a_n$ are \textbullet Divergent \textbullet Divergent \textbullet 3 \textbullet $\protect \frac  {2}{3}$ \textbullet Divergent \textbullet $\protect \frac  {9}{4}$ \textbullet Divergent \textbullet 1 \protect \newline  \protect \newline  
