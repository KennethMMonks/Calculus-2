\par 
 {\noindent \protect \bf  Exercise ~2.3.1.4.} \textbullet $\protect \frac  {1}{n+1}$ \textbullet $n+1$ \textbullet $(n+2)(n+1)$ \textbullet $(2n+2)(2n+1)$ \protect \newline  \protect \newline  
 {\noindent \protect \bf  Exercise ~2.4.0.3.} Think about what happens if the common difference $d$ is zero and if the common ratio $r$ is 1. \protect \newline  \protect \newline  
 {\noindent \protect \bf  Exercise ~2.4.1.2.} The common ratio $r$ is what we multiply by to get from term to term. Listing out the terms $a_0, a_0r, a_0r^2, a_0r^3,\protect \cdots  $ shows that $a_0r^n$ is the explicit formula. \protect \newline  \protect \newline  
 {\noindent \protect \bf  Exercise ~2.5.1.5.} In the context of computing a limit to infinity, it is fine to replace $n!$ by $\protect \sqrt  {2\pi n} \left ( \protect \frac  {n}{e} \right )^n$. Setting up limits of ratios and testing growth order with LHR and good old algebra will then verify that the order goes $n^2,e^n,n!,n^n$. \protect \newline  \protect \newline  
