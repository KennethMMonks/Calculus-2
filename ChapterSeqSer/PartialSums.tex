\section{The Sequence of Partial Sums}

\subsection{Adding Terms in a Sequence: Integration for Sequences}\label{AddingDong}
Given a sequence $a_n$, we build a new sequence $A_N$ called the \emph{sequence of partial sums} by keeping a running total of all terms in $a_n$ from $0$ to $N$.  We state this definition more formally.

\begin{definition}{Sequence of \series{Partial Sums}}
Let $a_n$ be a sequence.  Define $A_N$, the \emph{sequence of partial sums} of $a_n$ to be $$A_N=\sum_{n=0}^Na_n=a_0+a_1+a_2+\cdots+a_N. $$
\end{definition}
When studying a sequence and its partial sums, it can be helpful to organize your data in a table.

\begin{example}{From a Sequence to  Partial Sums}\label{OddSum}

Consider the sequence of odd natural numbers $a_n=2n+1$.  We compute a few partial sums and see if we can notice a pattern.

\begin{center}
\begin{tabular}{|c|c|c|c|} \hline
$n$ & $a_n$ & $A_n$ & Total \\ \hline
0 & 1 & 1 &1 \\
1 & 3 & 1+3 &4 \\
2 & 5 & 1+3+5 &9 \\
3 & 7 & 1+3+5+7 &16 \\
4 & 9 & 1+3+5+7+9 &25 \\
5 & 11 &1+3+5+7+9+11 &36 \\ \hline
\end{tabular}
\end{center}

We notice the column of totals contains all perfect squares.  In particular, the number in row $n$ is always exactly $\left(n+1\right)^2$.  Thus, the pattern suggests that
$$A_N=\sum_{n=0}^N\left(2n+1\right) = \left(N+1\right)^2 $$
\end{example}

\begin{exercise}{Computing a Partial Sum with the Arithmetic Series Formula \Coffeecup \Coffeecup}
Notice the sum $A_N$ above is in fact an arithmetic series!  Use the arithmetic series formula to evaluate $$A_N=\sum_{n=0}^N\left(2n+1\right)$$ and confirm it matches our conjectured formula from the table.
\solushun{
The first term is $1$ and the last term is $2N+1$. There are $N+1$ terms. Plugging these values into the Arithmetic Series Formula, we get
$$A_N=\sum_{n=0}^N\left(2n+1\right)=(N+1)\frac{1+2N+1}{2}=(N+1)\frac{2N+2}{2}=(N+1)\frac{2(N+1)}{2}=(N+1)^2.$$
}{.5in}
\AnswerKeyEntry{In this series, there are $N+1$ terms, the first of which is 1 and the last of which is $2N+1$.  Plugging this information into the Arithmetic Series Formula will produce the desired result.  }
\end{exercise}
Notice in Example \ref{AppleCations}.\ref{Critteria}, $a_n$ represents the number of critters in generation $n$, whereas $A_N$ represents the total number of descendants up to and including generation $N$.  This is a good way to think of the relationship between a sequence $a_n$ and the corresponding partial sums $A_N$; the quantities $A_N$ keep running totals of all $a_n$ we have encountered up to and including index $N$.
\begin{exercise}{Add-Ups \Coffeecup}
Suppose you start a push-up routine on day 0, where you do $a_0$ push-ups.  On day 1, you do $a_1$ push-ups. On day 2, you do $a_2$ push-ups, and so on.  In this context, what does the sequence of partial sums $A_N$ represent? \solushun{The partial sum $A_N$ represents the total number of push-ups you've done so far, up to and including day $N$.\\}{.2in} 
\AnswerKeyEntry{The partial sum $A_N$ represents the total number of push-ups you've done so far in your push-up routine, up to and including day $N$.}
\end{exercise}
\subsection{Discrete Derivatives: Derivatives for Sequences}

Given a sequence of partial sums, we can uncover the sequence from which it came.  The difference of two consecutive partial sums will be a single term in the sequence, since \begin{align*}
A_{N}-A_{N-1}&=\sum_{n=0}^Na_n-\sum_{n=0}^{N-1}a_n\\ &=\left( a_0+a_1+a_2+\cdots+a_N\right)-\left( a_0+a_1+a_2+\cdots+a_{N-1}\right) \\
&=a_N.
\end{align*}
\begin{example}{From Partial Sums to a Sequence} 
Let's try to undo Example \ref{AddingDong}.\ref{OddSum}.  Suppose we start with $A_N=(N+1)^2$.  We draw a table to see what terms $a_n$ would have been added together to obtain those totals. 
\begin{center}
\begin{tabular}{|c|c|c|c|} \hline
$N$ & $A_N$ & $a_N$ & Difference \\ \hline
0 & 1 & $1-0$ &1 \\
1 & 4 & $4-1$ &3 \\
2 & 9 & $9-4$ &5 \\
3 & 16 & $16-9$ &7 \\
4 & 25 & $25-16$ &9 \\
5 & 36 & $36-25$ &11 \\ \hline
\end{tabular}
\end{center}

We see that sure enough, the last column is the sequence of odd numbers and is always one more than twice $N$.  Thus, we have that $a_N=2N+1$. 

\end{example}

\begin{exercise}{Taking a Difference of Partial Sums \Coffeecup}
Using the formula $A_N=(N+1)^2$, try taking the difference $A_{N}-A_{N-1}$ and verify you get the same $a_N$.  That is, simplify the right-hand side of the following expression: $$ A_{N}-A_{N-1}=(N+1)^2-\left((N-1)+1\right)^2 $$
\solushun{
\begin{align*}
    A_{N}-A_{N-1}&=(N+1)^2-\left((N-1)+1\right)^2\\
    &=(N^2+2N+1)-(N-1+1)^2\\
    &=N^2+2N+1-N^2\\
    &=2N+1
\end{align*}
}{0in}
\end{exercise}


Note that we have two different indices, as we are taking the convention that $n$ indexes the sequence $a_n$ and $N$ indexes the partial sums $A_N$.  Thus, depending on which we start with, it looks like we have the ``wrong'' index for the other ($A_n$ vs $A_N$ or $a_n$ vs $a_N$).  This is nothing to worry about, as the sequence is really just the mapping from the natural numbers to the reals.  This is similar to how $f(x)=x^2$ and $f(t)=t^2$ are the same function on the reals, but just listed with different independent variables.

\begin{example}{Revisitng Our Critters }
In Example \ref{AppleCations}.\ref{Critteria}, we had computed the formula for a sequence of partial sums, even though at the time we didn't call it that.  In particular, given the sequence $$a_n=3^n$$
we found a closed formula for the sequence of partial sums as $$A_N=\sum_{n=0}^Na_n=\frac{1}{2}\left(3^{N+1}-1\right)$$ by using the Geometric Series Formula.  Here we demonstrate that the difference of consecutive partial sums will reproduce the original summand.
\begin{align*}
A_N-A_{N-1}&=\frac{1}{2}\left(3^{N+1}-1\right)-\frac{1}{2}\left(3^{\left(N-1\right)+1}-1\right) \\
&=\frac{1}{2}\left(3^{N+1}-1-3^{N}+1\right) \\
&=\frac{1}{2}\left(3^{N+1}-3^{N}\right) \\
&=\frac{3^N}{2}\left(3^{1}-1\right)\\
&=3^N
\end{align*}
Sure enough, $a_n=3^n$ was our original summand!
\end{example}

\subsection{Taking a Sequence to Partial Sum and Back Again}

Here we summarize a bit of what happened above. 
\begin{itemize}
\item Given a sequence $a_n$, we can define an associated sequences of partial sums $A_N=\sum_{n=0}^Na_n$.
\item If $a_n$ is an arithmetic or geometric sequence, we can find a formula for $A_N$ using the Arithmetic Series Formula or Geometric Series Formula.  If it $a_n$ is not an arithmetic or geometric sequence, then writing out a table of partial sums and looking for a pattern can be a good strategy.
\item Given $A_N$, we can recover $a_n$ by taking differences of consecutive partial sums $A_{N}-A_{N-1}$.
\end{itemize}

Notice this is very similar to what happened in Calculus I or even in the first part of this course.  In those sections, you could start with a function $f(x)$ and find its antiderivative $F(x)$.  If you then took the derivative of this $F(x)$, you would end up with the original $f(x)$. 

\begin{exercise}{Converting Back and Forth \Coffeecup \Coffeecup \Coffeecup}

For each of the following sequences $a_n$, compute the corresponding sequence of partial sums $A_N$.  Once you have $A_N$, then compute the difference of consecutive partial sums $A_{N}-A_{N-1}$ and verify that the original sequence comes back!

\begin{itemize}
\item $a_n=\frac{5}{2^n}$
\solushun{
The corresponding series is $$\sum_{n=0}^N\frac{5}{2^n}=5+\frac{5}{2}+\frac{5}{4}+\cdots+\frac{5}{2^N}.$$ This series has a common ratio of $\frac{1}{2}$, so we can use the Geometric Series Formula to find the partial sum:
$$A_N=5\frac{1-\left(\frac{1}{2}\right)^{N+1}}{1-\frac{1}{2}}=5\frac{1-\left(\frac{1}{2}\right)^{N+1}}{\frac{1}{2}}=10 \left(1-\left(\frac{1}{2}\right)^{N+1}\right).$$

Then, we take the difference of consecutive partial sums:

\begin{align*}
    A_{N}-A_{N-1}&=10 \left(1-\left(\frac{1}{2}\right)^{N+1}\right)-10 \left(1-\left(\frac{1}{2}\right)^{N}\right)\\
    &=10\left[1-\left(\frac{1}{2}\right)^{N+1}-\left(1-\left(\frac{1}{2}\right)^{N}\right)\right]\\
    &=10\left[1-\left(\frac{1}{2}\right)^{N+1}-1+\left(\frac{1}{2}\right)^{N}\right]\\
    &=10\frac{1}{2^N}\left[-\left(\frac{1}{2}\right)+1\right]\\
    &=10\frac{1}{2^N}\left[\frac{1}{2}\right]\\
    &=\frac{5}{2^N}.
\end{align*}
This is the last term of our original sum.\\
}{1in}
\item $a_n=\frac{2}{3^{2n+1}}$
\solushun{The corresponding series is
$$\sum_{n=0}^N \frac{2}{3^{2n+1}}= \frac{2}{3}+\frac{2}{3^{3}}+\frac{2}{3^{5}}+\cdots+\frac{2}{3^{2N+1}}.$$
Then, the common ratio is $\frac{1}{3^{2}}$. We can use the Geometric Series Formula to get 
$$A_N=\frac{2}{3}\frac{1-\frac{1}{9}^{N+1}}{1-\frac{1}{9}}=\frac{2}{3}\cdot\frac{9}{8}\left(1-\frac{1}{9^{N+1}}\right)=\frac{3}{4}\left(1-\frac{1}{9^{N+1}}\right).$$
Then we can take the difference of consecutive partial sums:
\begin{align*}
    A_N-A_{N-1}=&\frac{3}{4}\left(1-\frac{1}{9^{N+1}}\right)-\frac{3}{4}\left(1-\frac{1}{9^{(N-1)+1}}\right)\\
    &=\frac{3}{4}\left[\left(1-\frac{1}{9^{N+1}}\right)-\left(1-\frac{1}{9^{N}}\right)\right]\\
    &=\frac{3}{4}\left[\left(1-\frac{1}{9^{N+1}}\right)-1+\frac{1}{9^{N}}\right]\\
    &=\frac{3}{4}\left[\left(-\frac{1}{9^{N+1}}\right)+\frac{1}{9^{N}}\right]\\
    &=\frac{3}{4}\cdot\frac{1}{9^N}\left[\left(-\frac{1}{9}\right)+1\right]\\
    &=\frac{3}{4}\cdot\frac{1}{3^{2N}}\cdot\frac{8}{9}\\
    &=\frac{3}{4}\cdot\frac{1}{3^{2N}}\cdot\frac{8}{3^2}\\
    &=\frac{3\cdot1\cdot 8}{4\cdot3^{2N}\cdot3^2}\\
    &=\frac{2}{3^{2N+1}}.\\
\end{align*}
This agrees with the last term of our partial sum.\\
}{1in}
\item $a_n=5-n$
\solushun{
The corresponding sum is $$\sum_{n=0}^N 5-n=5+4+3+2+1+0+(-1)+\cdots+5-N.$$ This is an arithmetic series of $N+1$ terms with difference of $-n$, last term $5-N$, and first term $5$. Plugging this into the Arithmetic Series Formula, we get
$$A_N=(N+1)\frac{5+(5-N)}{2}=\frac{(N+1)(10-N)}{2}.$$
To check it, we take the difference of consecutive partial sums,
\begin{align*}
    A_N-A_{N-1}&=\frac{(N+1)(10-N)}{2}-\frac{((N-1)+1)(10-(N-1))}{2}\\
    &=\frac{(N+1)(10-N)}{2}-\frac{(N)(10-N+1)}{2}\\
    &=\frac{(N+1)(10-N)}{2}-\frac{(N)(11-N)}{2}\\
    &=\frac{9N-N^2+10-N}{2}-\frac{11N-N^2}{2}\\
    &=\frac{-2N+10}{2}\\
    &=5-N.
\end{align*}
This agrees with the last term of our series.
}{1in}
\item $a_n=3n^2+3n+1$
\solushun{
The corresponding sum is $$\sum_{n=0}^N 3n^2+3n+1=1+7+19+37+\cdots+(3N^2+3N+1).$$ There isn't an obvious pattern, but if we look at the sequence of partial sums, we notice a pattern:
\begin{center}
    \begin{tabular}{l|l}
        N=0 &  A_N=1\\
        N=1 & A_N=1+7=8\\
        N=2 & A_N=1+7+19=27\\
        N=3 & A_N=1+7+19+37=64\\
        \cdots & \cdots\\
        N & A_N=(N+1)^3.
    \end{tabular}
\end{center}
Taking the difference of consecutive partial sums, we get
\begin{align*}
    A_N-A_{N-1}&=(N+1)^3-((N-1)+1)^3\\
    &=N^3+3N^2+3N+1-N^3\\
    &=3N^2+3N+1
\end{align*}
}{1in}
\item $a_n=\left(-1\right)^n$
\solushun{
The corresponding sum is $$\sum_{n=0}^N(-1)^n=1-1+1-1+\cdots+(-1)^N.$$
\begin{center}\begin{tabular}{l|l}
    N=0 & 1 \\
    N=1 & 0 \\
    N=2 & 1 \\
    N=3 & 0 
\end{tabular}\end{center}
Getting an expression for this is a little tricky, but some trial and error suggests $$A_N=\frac{1-(-1)^{N+1}}{2}.$$

Taking the difference of consecutive partial sums, we get
\begin{align*}
    A_N-A{N-1}&=\frac{1-(-1)^{N+1}}{2}-\frac{1-(-1)^{(N-1)+1}}{2}\\
    &=\frac{1-(-1)^{N+1}}{2}-\frac{1-(-1)^{N}}{2}\\
    &=\frac{1-(-1)^{N+1}-1+(-1)^{N}}{2}\\
    &=\frac{-(-1)^{N+1}+(-1)^{N}}{2}\\
    &=(-1)^N\frac{-(-1)+1}{2}\\
    &=(-1)^N.
\end{align*}
This agrees with the final term of our intial series.\\
}{1in}
\item $a_n=\begin{cases}
1, & \text{if $n=0$;} \\
0, & \text{otherwise.}
\end{cases}$
\solushun{The corresponding sum is $$1+0+0+\cdots+0$$. Thus, the partial sum $A_N=1$. Comparing consecutive partial sums, we get $$A_N-A_{N-1}=1-1=0,$$ which is indeed the last term in the series.\\}{1in}
\end{itemize}

% *************************
% Check the answer key entry for 3n^2+3n+1. It was N^3, but I think it's supposed to be (N+1)^3. The solushun above checks out.
% *************************

\AnswerKeyEntry{The corresponding partial sums are as follows: \textbullet $A_N=10\left(1-1/2^{N+1}\right)$ \textbullet $A_N=3/4\left(1-1/9^{N+1}\right)$ \textbullet $A_N=\left(10-N\right)\left(N+1\right)/2$ \textbullet $A_N=(N+1)^3$ \textbullet $A_N=1,0,1,0,1,0,\ldots=\left(1-\left(-1\right)^{N+1}\right)/2$ \textbullet $A_N=1$ }
\end{exercise}

\begin{comment}

\subsubsection*{Sequence vs Partial Sum Telephone!}
We have two forms, a sequence and a sequence of partial sums.  We also have a way to go back and forth between the two forms.  This calls for a game of... TELEPHONE!

Break into groups of four and play telephone with one of the following pages.  If you are handed a sequence $a_n$, find the sequence of partial sums $A_N$, fold over the original $a_n$, and pass it along.  If you are handed a sequence of partial sums $A_N$, find the sequence it came from $a_n$, fold over the original $A_N$, and pass it along. 

\newpage

$$A_N=5\cdot\frac{1-\frac{1}{2^{(N+1)}}}{1-\frac{1}{2}} $$

\hrulefill

\vspace{.5in}

\begin{center}
\fbox{$a_n=$ \hspace{3in}}
\end{center}

\vspace{.5in}

\hrulefill

\vspace{.5in}

\begin{center}
\fbox{$A_N=$ \hspace{3in}}
\end{center}

\vspace{.5in}

\hrulefill

\vspace{.5in}

\begin{center}
\fbox{$a_n=$ \hspace{3in}}
\end{center}

\vspace{.5in}

\hrulefill

\vspace{.5in}

\begin{center}
\fbox{$A_N=$ \hspace{3in}}
\end{center}

\vspace{.5in}

\newpage

$$a_n=2n+3 $$

\hrulefill

\vspace{.5in}

\begin{center}
\fbox{$A_N=$ \hspace{3in}}
\end{center}

\vspace{.5in}

\hrulefill

\vspace{.5in}

\begin{center}
\fbox{$a_n=$ \hspace{3in}}
\end{center}

\vspace{.5in}

\hrulefill

\vspace{.5in}

\begin{center}
\fbox{$A_N=$ \hspace{3in}}
\end{center}

\vspace{.5in}

\hrulefill

\vspace{.5in}

\begin{center}
\fbox{$a_n=$ \hspace{3in}}
\end{center}

\vspace{.5in}

\newpage

$$a_n=2/3^n $$

\hrulefill

\vspace{.5in}

\begin{center}
\fbox{$A_N=$ \hspace{3in}}
\end{center}

\vspace{.5in}

\hrulefill

\vspace{.5in}

\begin{center}
\fbox{$a_n=$ \hspace{3in}}
\end{center}

\vspace{.5in}

\hrulefill

\vspace{.5in}

\begin{center}
\fbox{$A_N=$ \hspace{3in}}
\end{center}

\vspace{.5in}

\hrulefill

\vspace{.5in}

\begin{center}
\fbox{$a_n=$ \hspace{3in}}
\end{center}

\vspace{.5in}

\newpage
$$A_N=1 $$

\hrulefill

\vspace{.5in}

\begin{center}
\fbox{$a_n=$ \hspace{3in}}
\end{center}

\vspace{.5in}

\hrulefill

\vspace{.5in}

\begin{center}
\fbox{$A_N=$ \hspace{3in}}
\end{center}

\vspace{.5in}

\hrulefill

\vspace{.5in}

\begin{center}
\fbox{$a_n=$ \hspace{3in}}
\end{center}

\vspace{.5in}

\hrulefill

\vspace{.5in}

\begin{center}
\fbox{$A_N=$ \hspace{3in}}
\end{center}

\vspace{.5in}

\newpage

\end{comment}
Often the study of the real numbers and related objects is called \emph{continuous} mathematics while the study of the natural numbers and related objects is called \emph{discrete} mathematics.  In this course, we encounter many interesting parallels between the two pursuits!

\begin{exercise}{Discrete/Continuous Analogy \Coffeecup \Coffeecup \Coffeecup \Coffeecup}
\begin{itemize}
\item In what ways is taking the partial sums of a sequence similar to taking the integral of a function over the real numbers?  ({\bf Hint:} Plot your sequence and draw rectangles with height $a_n$ and width one.)
\solushun{
An integral sums up the area under a curve representing a function $f(x)$. Each point on the curve is the value of $f(x)$, so the area right under each point is just the value of $f(x)$. The integral sums up all these infinite values. Similarly, the partial sum adds up all the values of each term of the generating series $a_n$ at each point. If we draw a series of rectangles of height $a_n$ and width 1, the area of the rectangles is equal to the partial sum.\\
}{1in}
\item In what ways is taking the difference of consecutive terms in a sequence ($a_{n}-a_{n-1}$) similar to taking the derivative of a function over the real numbers?  ({\bf Hint: } Plot your sequence and think of how you could obtain $a_{n}-a_{n-1}$ as the slope of a secant line.)
\solushun{
When we take the derivative of a continuous function $f(x)$, we are finding out the slope between two infinitely close points along $f(x)$. Since $f$ is continuous, the concept of consecutive doesn't really make sense, but that what we're sort of aiming for with the derivative. In a sequence, we actually can look at the difference, or slope, between two points that are as close as they can get, which actually is consecutive. So the difference between consecutive terms $a_n$ and $a_{n-1}$ is essentially the instantaneous rate of change, or the derivative, of a sequence.\\
}{1in}
\item Suppose you start with a sequence $a_n$.  You add up terms to create the sequence of partial sums $A_N$.  You then take the difference of consecutive terms in the partial sums and find that $A_{N}-A_{N-1}=a_N$.  What theorem of calculus is this analogous to and why?  
\solushun{This is analogous to the fundamental theorem of calculus. If the difference of consecutive terms of a sequence is analogous to taking the derivative (as the previous exercise demonstrates), and the partial sum is analogous to the integral (as the first exercise of this set showed), then taking the difference of two consecutive partial sums is analogous to taking the derivative of the integral. In the case of partial sums, it gives back the original series term $a_n$, which is analogous to the original function that the integral was generated from. That is, $a_n$ is analogous to $f$ in $\frac{\dif}{\dif x}\left(\int_a^x f(t)\dif t\right)=f(x).$\\}{1in}
\end{itemize}
\end{exercise}