

\section{Integrating Products of Powers of Sine and Cosine}

In this section, we give an algorithm to find an \sine{\cosine{antiderivative}} of the form $$\int \sin^n(x)\cos^m(x) \dif x$$
for $n,m\in \mathbb{N}$.

\begin{exercise}{Knowledge is Power \Coffeecup}
There are two exponents in the integrand above. 
\begin{itemize}
\item What symbol above is the exponent of sine?  
    \solushun{$n$\\}{}
\item What symbol above is the exponent of cosine?  
    \solushun{$m$\\}{}
\end{itemize}
\end{exercise}

Note that some \antider{sine-cosine} integrals can be done by techniques you have already learned.  For example, $n$ or $m$ is equal to 1, ordinary $u$-substitution will work just fine!

\begin{exercise}{$u$-sub with Sines and Cosines \Coffeecup \Coffeecup}
Evaluate the following integral using the substitution $u=\sin(x)$:
$$\int \sin^2(x)\cos(x)\dif x $$
\solushun{Let $u=\sin(x)$, $\dif u = \cos(x)$. Then:
\begin{align*}
\int \sin^2(x)\cos(x)\dif x &= \int u^2\dif u\\
&=\frac{1}{3}u^3+C\\
&=\frac{1}{3}\sin^3(x)+C
\end{align*}}{2in}
\AnswerKeyEntry{The antiderivative is $\frac{1}{3}\sin^3(x)+C$.}
\end{exercise}

There are two types of integrals containing \integ{powers of sine and cosine}. The first type is the case where we have at least one odd exponent; the second type is where both exponents are even.  We show an overview of how to handle each case in the following awesome flow chart:

 

%\begin{figure}
	%\centering
	\begin{tikzpicture}[
      >=latex',
      auto
    ]
    
    \tikzstyle{box} = [rectangle, rounded corners, minimum width=4.75cm, minimum height=1cm, text centered, text width=4.25cm, draw=black, fill=gray!15, drop shadow]
    
    \tikzstyle{arrow} = [thick,>=stealth,arrowhead=5mm,->]
    
    \node [box] (given) {
    	Given\\ $\int \sin^n (x) \cdot \cos^m (x) \mathtt{d}x$. \\
        Is at least one of $n$ or $m$ odd?
        };
	\node [box]  (nmOdd) [node distance=2cm and -1cm,below left=of given] {Is $n$ odd or is $m$ odd?};
    \node [box]  (nOdd) [node distance=2cm and -1.75cm,below left=of nmOdd] {Break off one power of sine so you have: \\ 
    $\int \sin^{n-1}(x)\cos^m(x)\sin(x)\mathtt{d}x $ \\};
    \node [box]  (bothOdd) [node distance=7cm,below=of nmOdd] {Your choice! Use one of the following substitutions:
    $\cos^2(x)=1-\sin^2(x)$\\
    Or! \\
    $\sin^2(x)=1-\cos^2(x)$
    };
    
    \node [box]  (nOddSub) [node distance=1.5cm,below=of nOdd] {Use the substitution: \\
    $\sin^2(x)=1-\cos^2(x)$};
    \node [box]  (mOdd) [node distance=2cm and -1.75cm,below right=of nmOdd] {Break off one power of cosine so you have: \\ 
    $\int \cos^{m-1}(x)\sin^n(x)\cos(x)\mathtt{d}x $ \\};
    \node [box]  (mOddSub) [node distance=1.5cm,below=of mOdd] {Use the substitution: \\
    $\cos^2(x)=1-\sin^2(x)$};
    
    \node [box] (nmEven) [node distance=2cm and -1cm,below right=of given] {Proceed with half angle identities.};
    \node [box] (nmEven2) [node distance=2cm and -3cm,below right=of nmEven] {
    Cosine half angle identity: \\
    $\cos^2(x)=\frac{1+\cos(2x)}{2} $ \\
    \vspace{10pt}
    Sine half angle identity: \\
    $ \sin^2(x)=\frac{1-\cos(2x)}{2}$};
    
    
    
    \draw [->, >=open triangle 60, ultra thick,line width= 3pt, shorten >=2pt] (given) -- ($(given.south)+(0,-1)$) -| (nmOdd) node[above,pos=0.25] {Yes} ;
    \draw [->, >=open triangle 60, ultra thick,line width= 3pt, shorten >=2pt] (given) -- ($(given.south)+(0,-1)$) -| (nmEven) node[above,pos=0.25] {No} ;
    
    \draw [->, >=open triangle 60, ultra thick,line width= 3pt, shorten >=2pt] (nmOdd) -- ($(nmOdd.south)+(0,-.75)$) -| (nOdd) node[above,pos=0.25] {$n$ is odd} ;
    \draw [->, >=open triangle 60, ultra thick,line width= 3pt, shorten >=2pt] (nmOdd) -- ($(nmOdd.south)+(0,-.75)$) -| (mOdd) node[above,pos=0.25] {$m$ is odd} ;
    \draw [->, >=open triangle 60, ultra thick,line width= 3pt, shorten >=2pt] (nmOdd) -- (bothOdd) node[above, fill=white, pos=0.25] {Both odd} ;
    
    \draw [->, >=open triangle 60, ultra thick,line width= 3pt, shorten >=2pt] (nOdd) -- (nOddSub) ;
    \draw [->, >=open triangle 60, ultra thick,line width= 3pt, shorten >=2pt] (mOdd) -- (mOddSub) ;
    
    
    \draw [->, >=open triangle 60, ultra thick,line width= 3pt, shorten >=2pt] (nmEven) -- ($(nmEven.south)+(0,-.75)$) -| (nmEven2) ;
	
	\end{tikzpicture}
%\end{figure}

\subsection{At Least One Odd Power}\label{OneOdd}
Recall the Pythagorean identity for sine and cosine (written in two useful forms here): 
\FormulaBox{Pythagorean Theorem Slightly Rewritten}{\begin{tabular}{c|c} \hline
 $   \cos^2(x)=1-\sin^2(x) $ &  $   \sin^2(x)=1-\cos^2(x) $ \\ \hline
\end{tabular}
}

If at least one exponent is odd, we pull one of those functions out for the ``$\dif u$" and perform $u$-sub.  We then use the Pythagorean trig identity to rewrite sine and cosine in terms of each other as needed.  

\begin{example}{Odd Power Case}\label{OddMatters}
 Here we compute the integral $$\int \sin^7(x) \cos^2(x) \dif x$$.  In this case, we proceed using the substitution $u=\cos(x)$, so $\dif x=\frac{1}{-\sin(x)}\dif u$.
 
\begin{align*} \int \sin^7(x) \cos^2(x) \dif x &=\int \sin^6(x) \cos^2(x) \sin(x) \dif x \\
&= \int \left(\sin^2(x)\right)^3 \cos^2(x) \sin(x) \frac{1}{-\sin(x)}\dif u \\
&= \int \left(1-\cos^2(x)\right)^3 \cos^2(x) (-1) \dif u \\
&= -\int \left(1-u^2\right)^3 u^2 \dif u \\
&= -\int \left(1-3u^2+3u^4-u^6\right) u^2 \dif u \\
&= -\int \left(u^2-3u^4+3u^6-u^8\right) \dif u \\
&= -\left(\frac{1}{3}u^3-\frac{3}{5}u^5+\frac{3}{7}u^7-\frac{1}{9}u^9\right)+C \\
&= -\frac{1}{3}\cos^3(x)+\frac{3}{5}\cos^5(x)-\frac{3}{7}\cos^7(x)+\frac{1}{9}\cos^9(x)+C \\
\end{align*}

\end{example}

\begin{exercise}{Why Odd Mattered \Coffeecup}

In Example \ref{OneOdd}.\ref{OddMatters}, the exponent of sine (in this case, the number 7) being odd really mattered.  If that 7 were replaced by an even number instead, why would this approach have failed?  Answer in a few short sentences below.
\solushun{If the exponent of sine had been even, then we couldn't have used the Pythagorean identity to express it in terms of cosine, and still had an extra sine to use with the $u$-sub, which would have prevented us from expressing all the parts in terms of a single expression.\\}{1in}
\AnswerKeyEntry{Since seven is odd, when we pulled out one factor of sine, we ended up with the sixth power of sine remaining.  Since six is even, we were able to express it as a power of a perfect square of sine, which in turn let us rewrite as cosines using the Pythagorean identity.}
\end{exercise}

\begin{exercise}{Try a Few with Odd Exponents \Coffeecup \Coffeecup}

\begin{itemize}
\item Find an antiderivative for the function $\sin^5(x) \cos^2(x)$.
\solushun{\begin{align*}
\int \sin^5(x) \cos^2(x)\dif x &= \int \sin^4(x) \cos^2(x)\sin(x)\dif x\\
&= \int \left(\sin^2(x)\right)^2 \cos^2(x)\sin(x)\dif x\\
&= \int \left(1-\cos^2(x)\right)^2 \cos^2(x)\sin(x)\dif x
\end{align*}
Let $u=\cos(x)$ so $\dif u = -\sin(x)$ and $\dif x = \frac{1}{-\sin(x)}\dif u$
\begin{align*}
\int \left(1-\cos^2(x)\right)^2 \cos^2(x)\sin(x)\dif x&=\int \left(1-u^2\right)^2 u^2\sin(x)\frac{1}{-\sin(x)}\dif u\\
&=\int \left(1-u^2\right)^2 u^2(-1)\dif u\\
&=-\int \left(1-2u^2+u^4\right)u^2\dif u\\
&=-\int u^2-2u^4+u^6\dif u\\
&=-\left(\frac{1}{3}u^3-\frac{2}{5}u^5+\frac{1}{7}u^7+C\right)\\
&=-\frac{1}{3}\cos^3(x)+\frac{2}{5}\cos^5(x)-\frac{1}{7}\cos^7(x)+C
\end{align*}}{2.5in}

\item Evaluate $\int \cos^9(x) \dif x$. ({\bf Hint:} Pascal's Triangle will be extremely helpful!)
\solushun{\begin{align*}
\int \cos^9(x) \dif x&=\int\cos^8(x)\cos(x)\dif x\\
&=\int(1-\sin^2(x))^4\cos(x)\dif x\\
&\text{Let $u=\sin(x)$}\\
&=\int(1-u^2)^4\dif u\\
&=\int1-4u^2+6u^2-4u^6+u^8\dif u\\
&=u-\frac{4}{3}u^3+\frac{6}{3}u^3-\frac{4}{7}u^7+\frac{1}{9}u^9+C\\
&=\sin(x)-\frac{4}{3}\sin^3(x)+\frac{6}{3}\sin^3(x)-\frac{4}{7}\sin^7(x)+\frac{1}{9}\sin^9(x)+C\\
\end{align*}}{2.5in}

\end{itemize}
\AnswerKeyEntry{The first antiderivative is $-\frac{1}{3}\cos^3(x)+\frac{2}{5}\cos^5(x)-\frac{1}{7}\cos^7(x)+C$.  For the second, rewrite as $(1-\sin^2(x))^4\cos(x)$ and proceed by letting $u=\sin(x)$.}
\end{exercise}

\begin{exercise}{Two Different Options \Coffeecup \Coffeecup}
\begin{itemize}

\item  Consider $\int \cos(x)\sin^3(x) \dif x$.
\begin{itemize}
\item  Compute this integral using $u=\cos(x)$.
\solushun{\begin{align*}
\int \cos(x)\sin^3(x)\dif x&=\int\cos(x)\left(1-\cos^2(x)\right)\sin(x)\dif x\\
&=-\int u\left(1-u^2\right)\dif u\\
&=-\int u-u^3\dif u\\
&=-\left(\frac{1}{2}u^2-\frac{1}{4}u^4\right)+C\\
&=-\frac{1}{2}\cos^2(x)+\frac{1}{4}\cos^4(x)+C\\
\end{align*}}{2in}

\item  Compute this integral using $u=\sin(x)$.
\solushun{\begin{align*}
\int \cos(x)\sin^3(x)\dif x&=\int u^3\dif u\\
&=\frac{1}{4}\sin^4(x)+C
\end{align*}}{2in}

\item Your two answers will appear very different!  Show that they are in fact compatible.
\solushun{\begin{align*}
-\frac{1}{2}\cos^2(x)+\frac{1}{4}\cos^4(x)+C&=-\frac{1}{2}(1-\sin^2(x))+\frac{1}{4}(1-\sin^2(x))^2+C\\
&=-\frac{1}{2}+\frac{1}{2}\sin^2(x)+\frac{1}{4}(1-2\sin^2(x)+\sin^4(x))+C\\
&=-\frac{1}{2}+\frac{1}{2}\sin^2(x)+\frac{1}{4}-\frac{1}{2}\sin^2(x)+\frac{1}{4}\sin^4(x)+C\\
&=-\frac{1}{4}+\frac{1}{4}\sin^4(x)+C\\
&\text{$-\frac{1}{4}$ is absorbed into $C$ leaving}\\
&=\frac{1}{4}\sin^4(x)+C\\
\end{align*}}{1in}

\AnswerKeyEntry{Often when trying to show that two antiderivatives are compatible, it is easiest to verify that their difference is a constant. }
\end{itemize}

\item Consider $\int \cos^3(x)\sin^{11}(x) \dif x$.
\begin{itemize}
\item  Can you compute this integral using $u=\cos(x)$?  Explain.
\solushun{Taking $u=\cos(x)$, we can express the integral as $\int\cos^3(x)\left(1-\cos^2(x)\right)^5\cos(x)\dif x=\int u^3\left(1-u^2\right)^5\dif u$\\}{1in}

\item  Can you compute this integral using $u=\sin(x)$?  Explain.
\solushun{We can express the integral as $\left(1-sin^2(x)\right)\sin^11(x)\cos(x)\dif x = \int\left(1-u^2\right)\left(u^{11}\right)\dif x\\$}{1in}

\item Which of the two above substitutions will be easier to use?  Carry out the integration, using the easier of the two.
\solushun{The second option, $u=\sin(x)$ will be easier because it avoids expading a binomial to the 5th power.
Let $u=\sin(x)$
\begin{align*}
\int \cos^3(x)\sin^{11}(x) \dif x &= \left(1-u^2\right)u^{11}\dif u\\
&=\int u^{11}-u^{13}\dif u \\
&=\frac{1}{12}u^{12}-\frac{1}{14}u^{14}+C\\
&=\frac{1}{12}\sin^{12}(x)-\frac{1}{14}\sin^{14}(x)+C
\end{align*}}{2.5in}

\AnswerKeyEntry{The substitution $u=\sin(x)$ is much cleaner since the other will involve having to expand a binomial to the fifth power.  The antiderivative is $\frac{1}{12}\sin^{12}(x)-\frac{1}{14}\sin^{14}(x)+C$.}
\end{itemize}
\end{itemize}

\end{exercise}

\subsection{Both Even Powers}

Recall the \trigidentities{Half-Angle Identities}!
\FormulaBox{Half-Angle Identities}{
\begin{tabular}{c|c} \hline
 $     \cos^2(x)=\frac{1+\cos(2x)}{2}  
   $ &  $    \sin^2(x)=\frac{1-\cos(2x)}{2} $ \\ \hline
\end{tabular}}

If the \halfangle{powers of sine and cosine} are both even, we use the half-angle identities for both sine and cosine.  This can get quite messy, but it works!

\begin{exercise}{Just Cosines without Sine \Coffeecup}
 Consider the following integral: $$\int \cos^6(x)\dif x $$ Here the exponent on cosine is the even number 6.  What is the exponent of sine in that integrand?  Is that an even number? 
 \solushun{The exponent on sine is zero, which is indeed even.  Thus both exponents are even in this case.\\}{.2in}
 \AnswerKeyEntry{The exponent on sine is zero, which is indeed even.  Thus both exponents are even in this case.}
 \end{exercise}
 
\begin{example}{Carrying Out Antidifferentiation with the Half-Angle Identities}
We now show how the half-angle identities help antidifferentiate the sixth power of cosine. 

\begin{align*} \int  \cos^6(x) \dif x &=\int \left(\cos^2(x)\right)^3  \dif x \\
&=\int \left(\frac{1+\cos(2x)}{2}\right)^3  \dif x \\ 
&=\frac{1}{8}\int 1+3\cos(2x)+3\cos^2(2x)+\cos^3(2x)  \dif x \\
&=\frac{1}{8}\left(\int 1\dif x+\int 3\cos(2x)\dif x+\int 3\cos^2(2x)\dif x+\int \cos^3(2x) \dif x\right)
\end{align*}

Notice that we now have four integrals.  The first is easy, the second is a $u$-substitution, and the third is another even power of cosine (where we again use the half-angle identity).  Finally, the fourth is an odd power of cosine, so we can use the technique from Section \ref{OneOdd}.  

\end{example}

\begin{exercise}{Finishing the Example \Coffeecup \Coffeecup}
Carry out each of these processes to compute the four integrals:
\begin{itemize}
\item $\int 1\dif x$
\solushun{$$\int 1\dif x=x+C$$}{.2in}
\item $\int 3\cos(2x)\dif x$
\solushun{
Let $u=2x$. Then,
$$\int3\cos2x\dif x = \frac{3}{2}\int\cos u\dif u =\frac{3}{2}\sin2x+C$$}{1in}
\item $\int 3\cos^2(2x)\dif x$
\solushun{$$\int 3\cos^2(2x)\dif x=3\int\frac{1+\cos4x}{2}\dif x=3\left(\frac{1}{2}x +\frac{1}{8}\sin4x+C\right)=\frac{3}{2}x +\frac{3}{8}\sin4x+C$$}{1.5in}
\item $\int \cos^3(2x) \dif x$
\solushun{Let $u=\sin2x$ and $\dif u = 2\cos2x\dif x$. Then, 
\begin{align*}\int \cos^3(2x) \dif x&=\int \cos2x\left(1-\sin^22x\right)\dif x\\&=\frac{1}{2}\int1-u^2\dif u\\&=\frac{1}{2}u-\frac{1}{6}u^3+C\\&=\frac{1}{2}\sin2x-\frac{1}{6}\sin^32x+C\end{align*}}{1.5in}
\end{itemize}

Add your antiderivatives together and combine like terms to produce your final answer for the integral!  Oh and remember that one-eighth.

$$ \int  \cos^6(x) \dif x = \hspace{4in} $$ 
\solushun{$$
\frac{1}{8}\left(x+\frac{3}{2}\sin2x+\frac{3}{2}x+\frac{3}{8}\sin4x+\frac{1}{2}\sin2x-\frac{1}{6}\sin^32x+C\right)$$ $$=\frac{1}{8}\left(\frac{5}{2}x+2\sin2x+\frac{3}{8}\sin4x-\frac{1}{6}\sin^32x+C\right)$$
$$=\frac{5}{16}x+\frac{1}{4}\sin(2x)-\frac{1}{48}\sin^3(2x)+\frac{3}{64}\sin(4x)+C
$$}{.5in}

\AnswerKeyEntry{When all like terms are combined and the one-eighth is distributed, the result is $\frac{5}{16}x+\frac{1}{4}\sin(2x)-\frac{1}{48}\sin^3(2x)+\frac{3}{64}\sin(4x)+C$.}
\end{exercise}

\begin{exercise}{Checking the Previous Example \Coffeecup \Coffeecup \Coffeecup}
Differentiate your answer and verify you get the original integrand back.
\solushun{$$\frac{\dif}{\dif x}\left(\frac{5}{16}x+\frac{1}{4}\sin(2x)-\frac{1}{48}\sin^3(2x)+\frac{3}{64}\sin(4x)+C\right)$$ Before we differentiate, first bash everything back down to an ``$x$''
in the argument using double angle identities.  This produces

$$\frac{5}{16}x+\frac{1}{2}\sin(x)\cos(x)-\frac{1}{6}\sin^3(x)\cos^3(x)+\frac{3}{16}\sin(x)\cos^3(x)-\frac{3}{16}\sin^3(x)\cos(x)+C$$
Factor out a sine and use the Pythagorean Identity to get everything else in terms of cosine.  This produces $$\frac{5}{16}x+\sin(x)\left(\frac{5}{16}\cos(x)+\frac{5}{24}\cos^3(x)+\frac{1}{6}\cos^5(x)\right)+C$$  
Then we differentiate and obtain $$\frac{5}{16}+\cos(x)\left(\frac{5}{16}\cos(x)+\frac{5}{24}\cos^3(x)+\frac{1}{6}\cos^5(x)\right)-\sin^2(x)\left(\frac{5}{16}+\frac{5}{8}\cos^2(x)+\frac{5}{6}\cos^4(x)\right)$$ to which we apply the Pythagorean Identity $\sin^2(x)=1-\cos^2(x)$ to produce $$\frac{5}{16}+\cos(x)\left(\frac{5}{16}\cos(x)+\frac{5}{24}\cos^3(x)+\frac{1}{6}\cos^5(x)\right)-\left(1-\cos^2(x)\right)\left(\frac{5}{16}+\frac{5}{8}\cos^2(x)+\frac{5}{6}\cos^4(x)\right)$$
This will simplify to $\cos^6(x)$ once you expand and combine like terms.\\}{2in}
 
\AnswerKeyEntry{The antiderivative to $\cos^6(x)$ came out to

$$\frac{5}{16}x+\frac{1}{4}\sin(2x)-\frac{1}{48}\sin^3(2x)+\frac{3}{64}\sin(4x)+C$$ Before we differentiate, first bash everything back down to an ``$x$''
in the argument using double angle identities.  This produces

$$\frac{5}{16}x+\frac{1}{2}\sin(x)\cos(x)-\frac{1}{6}\sin^3(x)\cos^3(x)+\frac{3}{16}\sin(x)\cos^3(x)-\frac{3}{16}\sin^3(x)\cos(x)+C$$
Factor out a sine and use the Pythagorean Identity to get everything else in terms of cosine.  This produces $$\frac{5}{16}x+\sin(x)\left(\frac{5}{16}\cos(x)+\frac{5}{24}\cos^3(x)+\frac{1}{6}\cos^5(x)\right)+C$$  
Then we differentiate and obtain $$\frac{5}{16}+\cos(x)\left(\frac{5}{16}\cos(x)+\frac{5}{24}\cos^3(x)+\frac{1}{6}\cos^5(x)\right)-\sin^2(x)\left(\frac{5}{16}+\frac{5}{8}\cos^2(x)+\frac{5}{6}\cos^4(x)\right)$$ to which we apply the Pythagorean Identity $\sin^2(x)=1-\cos^2(x)$ to produce $$\frac{5}{16}+\cos(x)\left(\frac{5}{16}\cos(x)+\frac{5}{24}\cos^3(x)+\frac{1}{6}\cos^5(x)\right)-\left(1-\cos^2(x)\right)\left(\frac{5}{16}+\frac{5}{8}\cos^2(x)+\frac{5}{6}\cos^4(x)\right)$$
This will simplify to $\cos^6(x)$ once you expand and combine like terms.
}
\end{exercise}

\begin{exercise}{Practice with the Even Case \Coffeecup \Coffeecup}
\begin{itemize}

\item  Find an antiderivative for the function $\sin^2(3x)$.
\solushun{\begin{align*}
\int\sin^2(3x)&=\int\frac{1-\cos(6x)}{2}\dif x\\
&=\int\frac{1}{2}-\frac{1}{2}\cos(6x)\dif x\\
\text{Let $u=6x, \dif u=6\dif x$}\\
&=\frac{1}{2}x-\frac{1}{2}\cdot\frac{1}{6}\int\cos u\dif u\\
&=\frac{1}{2}x-\frac{1}{12}\sin u+C\\
&=\frac{1}{2}x-\frac{1}{12}\sin 6x+C
\end{align*}}{1in}
\item  Find an antiderivative for the function $\sin^4(x)$.  
\solushun{\begin{align*}
\int\sin^4(x)\dif x&=\int\left(\sin^2(x)\right)^2\dif x\\
&=\int\left(\frac{1-\cos(2x)}{2}\right)^2\dif x\\
&=\int\frac{1-2\cos(2x)+cos^2(2x)}{4}\dif x\\
&=\int\frac{1}{4}-\frac{1}{2}\cos(2x)+\frac{1}{4}\cdot\frac{1+\cos(4x)}{2}\dif x \\
&=\int\frac{1}{4}-\frac{1}{2}\cos(2x)+\frac{1}{8}+\frac{1}{8}\cos(4x)\dif x \\
&=\int\frac{3}{8}-\frac{1}{2}\cos(2x)+\frac{1}{8}\cos(4x)\dif x \\
&=\frac{3}{8}x-\frac{1}{4}\sin(2x)+\frac{1}{32}\sin(4x) +C\\
\end{align*}}{2in}

\item  Find an antiderivative for the function $\sin^2(x)\cos^2(x)$.
\solushun{\begin{align*}
\int\sin^2(x)\cos^2(x)\dif x&=\int\sin^2(x)\left(1-\sin^2(x)\right)\dif x \\
&=\int\sin^2(x)-\sin^4(x)\dif x \\
\text{From previous examples: }\\
\int\sin^2(x)\dif x &= \frac{1}{2}x-\frac{1}{2}\sin x+C\\
\int\sin^4(x)\dif x &= \frac{3}{8}x-\frac{1}{4}\sin(2x)+\frac{1}{32}\sin(4x) +C\\
\int\sin^2(x)-\sin^4(x)\dif x &=\frac{1}{2}x-\frac{1}{2}\sin x-\left(\frac{3}{8}x-\frac{1}{4}\sin(2x)+\frac{1}{32}\sin(4x)\right)+C\\
&=\frac{1}{2}x-\frac{1}{2}\sin x-\frac{3}{8}x+\frac{1}{4}\sin(2x)-\frac{1}{32}\sin(4x)+C\\
&=\frac{1}{8}x-\frac{1}{2}\sin x+\frac{1}{4}\sin(2x)-\frac{1}{32}\sin(4x)+C\\
\end{align*}}{2in}

\AnswerKeyEntry{For the first, apply the identity $\sin^2(3x)=\frac{1-\cos(6x)}{2}$ and proceed.  For the second, notice that $\sin^4(x)$ can be rewritten as $\left(\sin^2(x)\right)^2$, after which the half-angle identity can be applied.}
\end{itemize}
\end{exercise}
