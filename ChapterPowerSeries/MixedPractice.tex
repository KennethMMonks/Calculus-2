\section{Mixed Practice}
Here is a wide spread of practice problems using power series to help absorb all of the concepts and techniques!

\begin{exercise}{Nothing New, Just Practice \Coffeecup \Coffeecup}
\begin{itemize}

\item \begin{itemize} \item Find a degree two power series for the function $ \sqrt[3]{x} $ centered at 8. 
 
\vspace*{2in}

\item Use your approximation from part a) to estimate the value of $\sqrt[3]{8.05}$.

\vspace*{2in}

\item Use Taylor's Error Theorem to give a bound on how bad the error could be in your estimation of the value of $\sqrt[3]{8.005}$.  Type the exact value into a calculator or CAS and confirm that you have obtained the desired accuracy.

\vspace*{2in}

\end{itemize}

\item Evaluate each of the following infinite series to a closed form.  Explain your reasoning. \begin{itemize}
\item  $ \overset{\infty}{\underset{n=0}{\sum}} \frac{3^n}{n!2^{n+1}}$

\vspace*{1.5in}

\item $ \overset{\infty}{\underset{n=3}{\sum}} \frac{(-1)^n}{n} $

\vspace*{1.5in}

\item $ \overset{\infty}{\underset{n=0}{\sum}} \frac{(-1)^{n}\pi^{2n}}{(2n)!2^{2n}} $

\vspace*{1.5in}

\end{itemize}
\begin{comment}

\item  \begin{itemize} \item List by hand how many ways you could make change for 35 cents using nickels, dimes, and quarters.

\vspace*{2in}

\item Show how you would use geometric series as generating functions to reach this same conclusion.  Verify your answers match.

\vspace*{3in}
\end{itemize}

\end{comment}
\item Find a rational number that approximates the square root of $e$ accurate to three decimal places.  ({\bf Hint:} Think of the function $e^{x}$ evaluated at $x={1/2}$.  Then use Taylor's Error Theorem to guarantee sufficient accuracy.)
\vspace*{3in}

\item Find power series and Interval of Convergence for the following functions:

\begin{itemize} \item $f(x)=\frac{1}{x}$ centered at -3

\vspace*{1in}

\item $f(x)=\frac{1}{1-x}$ centered at -3

\vspace*{1in}

\item $f(x)=\frac{1}{e^x}$ centered at 0

\vspace*{1in}

\item $f(x)=\frac{1-x-x^2}{1-x}$ centered at 0

\vspace*{1in}

\item $f(x)=2^x$ centered at 1

\vspace*{1in}

\item $f(x)=x^2+x+1 $ centered at 5

\vspace*{1in}

\end{itemize}

\item  Find a degree three power series centered at zero for the function $\frac{1}{4-x^2}$ in five different ways:
 \begin{itemize}
\item Brute force.

\vspace*{2in}

\item Via geometric series with a substitution for $x$.  

\vspace*{2in}

\item Via a multiplication of the series for $\frac{1}{2-x}$ with 
$\frac{1}{2+x}$.

\vspace*{2in}

\item Via a sum of the series that result in a partial fraction decomposition of $\frac{1}{4-x^2}$.

\vspace*{2in}

\item Via long division, dividing the numerator 1 by the denominator $4-x^2$.

\vspace*{2in}

\end{itemize}
\end{itemize}
\end{exercise}
