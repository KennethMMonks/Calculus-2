\section{Partition \partition{Counting via Power Series}}

It turns out that power series are one of the best tools for counting that which is hard to count.  Ready to play? 

\begin{definition}{Partition}
A {\bf partition} of a positive natural number $n$ is an expression of $n$ as a sum of positive integers.  The terms in that sum are called {\bf parts}.
\end{definition} 

For example, 1+1+2+3 is a partition of 7 with parts 1,1,2, and 3.  We do not count different orderings of the same numbers as different partitions.  So 1+1+2+3 and 1+2+1+3 would be counted as the same partition.

It often becomes of interest to know how many partitions a number has.  For example, the number 4 has five partitions because:

$$ 4=3+1=2+2=2+1+1=1+1+1+1$$

Note that writing 4 as a sum of just the one number 4 itself counts as a valid partition.

$$ 5=4+1=3+2=3+1+1=2+2+1=2+1+1+1=1+1+1+1+1$$

shows that 5 has seven partitions.

In case you aren't tickled pink by the inherent mathematical challenge of counting such a thing, know that partitions come up in innumerable (ok, numerable but large) applications outside of mathematics, as you are often faced with decomposing a quantity into smaller quantities.  For example, they come up when simulating nuclear fission; when an atom is smashed, the nucleus of protons and neutrons is broken into a set of smaller clusters of subatomic particles. The sum of the particles in the set of clusters must equal the original size of the nucleus. As such, the number of partitions of the original number of protons counts all the possible ways to smash the atom.

Thus, we seek to find how many partitions of a natural number $n$ are there?  If we were to program a computer or graphing calculator to find such a quantity, how could we do it?  It turns out that one answer lies with power series!

Consider the following function: $$f(x)=\frac{1}{1-x}\frac{1}{1-x^2}\frac{1}{1-x^3}\frac{1}{1-x^4}\frac{1}{1-x^5}\frac{1}{1-x^6}\cdots$$

We claim that in expanded form,  $$ f(x)=a_0+a_1x+a_2x^2+a_3x^3+a_4x^4+\cdots$$
then $a_n$ is the number of partitions of the integer $n$.  Work through the following exercise to see why!
\begin{exercise}{Partition \partition{Generating Function} \Coffeecup \Coffeecup}

\begin{itemize}

\item Begin by using the geometric series to replace each of the factors in the above infinite product with a power series.  Write this expression below.

\vspace*{2in}

\item Begin to expand out the series by multiplying out the factors above.  Simply proceed one degree at a time.  Get all the coefficients out to at least degree 5.  How do the terms that arise in this product correspond to the partitions we listed for 5 above?  (This will be a bit computationally intensive, but it will be worth it.)

\vspace*{4in}

\item Use the above to find the number of partitions of 20.  You may use a CAS to do algebra or differentiation for you, but indicate below what instructions you gave the software.

\vspace*{1in}
\end{itemize}
\end{exercise}

\newpage

It turns out the beauty of this method is that it is highly robust!  It is easy to modify if we want to find the partitions using only particular positive integers instead of having any part size at our disposal.  

\begin{example}{Only Certain Part Sizes}
Find all partitions of 10 using only parts of size 1, 4, and 5.  Here we would have 
\begin{align*}
10&=5+5 \\ 
&=5+4+1 \\ 
&=5+1+1+1+1+1 \\
&=4+4+1+1 \\ 
&=4+1+1+1+1+1+1 \\ 
&=1+1+1+1+1+1+1+1+1+1
\end{align*}
which is six partitions of 10 using only parts of size 1,4, and 5.
\end{example}

\begin{exercise}{Restricted Partition Generating Functions \Coffeecup \Coffeecup \Coffeecup}

\begin{enumerate}

\item Expand the following function out to get a power series of degree ten. Explain how the number of partitions with restricted part size above corresponds to the degree 10 coefficient. $$f(x)=\frac{1}{1-x}\frac{1}{1-x^4}\frac{1}{1-x^5}$$

\vspace*{3in}

\item Suppose we wanted to figure out how many ways we can make change for a dollar using pennies, nickels, dimes, and quarters.  How could we accomplish this?  Find the number and explain your solution.  Again you may use mathematica, WolframAlpha, or another computer algebra system to multiply polynomials for you.

\vspace*{2in}

\end{enumerate}
\end{exercise}