
\section{Mixed Practice with IOC} 
Recall our general framework for power series.  To find a power series centered at $x=a$ for a function $f(x)$, we write down an equation of the form $$ f(x)=a_0+a_1(x-a)+a_2(x-a)^2+a_2(x-a)^3+a_2(x-a)^4+\cdots$$
and then repeatedly plug in $x=a$ and differentiate in order to solve for the coefficients, one at a time.

Though this method will work to find the coefficients as long as the function and its derivatives exist at $x=a$, there remains the question: for what $x$-values will the infinite series on the right actually converge to the corresponding value of $f(x)$?  In this activity, we investigate this both numerically (just using a table of values) and theoretically (using the ratio test).  We call the set of all $x$-values for which a given series converges the \emph{interval of convergence}.

Here we work through this framework in a variety of examples.

\begin{exercise}{Natural Logarithm \Coffeecup \Coffeecup \Coffeecup}
Start with the function $f(x)=\ln(x)$ and do the following:

\begin{itemize}
\item Set up a power series centered at $a=2$ for $\ln(x)$.  Solve for the degree 2, degree 3, degree 4, degree 5, and degree 6 power series approximations for $\ln(x)$ centered at $2$.  Accomplish this by just repeatedly plugging in $x=2$ and differentiating both sides.  Call these functions $P_2(x), P_3(x), P_4(x),$ and  $P_5(x),$ respectively.
\item Make a small table of values where you list out the values of each of your $P_i$ evaluated at $x=1/2$.  Compare this to the true value of $\ln(1/2)$.  
\item Make a small table of values where you list out the values of each of your $P_i$ evaluated at $x=2$.  Compare this to the true value of $\ln(2)$.
\item Perform the Ratio Test on your series expansion for $\ln(x)$.  For what $x$ would you have a ratio less than one?
\item How do your results of the ratio test compare to the numerical evidence you found above?
\item Notice that there will be two $x$-values that cause the series to have ratio exactly equal to one when Ratio Test is applied.  Since the Ratio Test gives no info in that case, try a different test for each of those two series.  
\item At last, state the power series you came up with and the Interval of Convergence.
\end{itemize}
\end{exercise}

\begin{exercise}{Arctangent \Coffeecup \Coffeecup \Coffeecup}
Start with the function $f(x)=\arctan(x)$ and do the following:

\begin{itemize}
\item Set up a power series centered at $x=0$ for $\arctan(x)$.  Solve for the degree 3, degree 5, degree 7, and degree 9 power series approximations.  Accomplish this by just repeatedly plugging in $x=0$ and differentiating both sides.  Call these functions $P_3(x), P_5(x), P_7(x)$, and $P_9(x),$ respectively.  You may want to use a computer algebra system to help with the messy derivatives that will arise!
\item Make a small table of values where you list out the values of each of your $P_i$ evaluated at $x=1/2$.  Compare this to the true value of $\arctan(1/2)$.  
\item Make a small table of values where you list out the values of each of your $P_i$ evaluated at $x=2$.  Compare this to the true value of $\arctan(2)$.
\item Perform the Ratio Test on your series expansion for $\arctan(x)$.  For what $x$ would you have a ratio less than one?
\item How do your results of the ratio test compare to the numerical evidence you found above?
\item Notice that there will be two $x$-values that cause the series to have ratio exactly equal to one when Ratio Test is applied.  Since the Ratio Test gives no info in that case, try a different test for each of those two series.  
\item At last, state the power series you came up with and the Interval of Convergence.
\end{itemize}
\end{exercise}

\begin{exercise}{Exponential \Coffeecup \Coffeecup \Coffeecup}
Start with the function $f(x)=e^x$ and do the following:

\begin{itemize}
\item Set up a power series centered at $x=0$ for $e^x$.  Solve for the degree 2, degree 3, degree 4, and degree 5 power series approximations.  Accomplish this by just repeatedly plugging in $x=0$ and differentiating both sides.  Call these functions $P_2(x), P_3(x), P_4(x),$ and $P_5(x),$ respectively.
\item Make a small table of values where you list out the values of each of your $P_i$ evaluated at $x=1/2$.  Compare this to the true value of $e^{1/2}$.  
\item Make a small table of values where you list out the values of each of your $P_i$ evaluated at $x=2$.  Compare this to the true value of $e^2$.
\item Perform the Ratio Test on your series expansion for $e^x$.  For what $x$ would you have a ratio less than one?
\item How do your results of the ratio test compare to the numerical evidence you found above?  
\item At last, state the power series you came up with and the Interval of Convergence.
\end{itemize}
\end{exercise}

\begin{exercise}{Cosine \Coffeecup \Coffeecup \Coffeecup}
Start with the function $f(x)=\cos(x)$ and do the following:
\begin{itemize}
\item Set up a power series centered at $x=\pi$ for $\cos(x)$.  Solve for the degree 2, degree 4, degree 6, and degree 8 power series approximations.  Accomplish this by just repeatedly plugging in $x=\pi$ and differentiating both sides.  Call these functions $P_2(x), P_4(x), P_6(x),$ and $P_8(x),$ respectively.
\item Make a small table of values where you list out the values of each of your $P_i$ evaluated at $x=\pi/2$.  Compare this to the true value of $\cos\left(\pi/2\right)$.  
\item Make a small table of values where you list out the values of each of your $P_i$ evaluated at $x=2$.  Compare this to the true value of $\cos(2)$.
\item Perform the Ratio Test on your series expansion for $\cos(x)$.  For what $x$ would you have a ratio less than one?
\item How do your results of the ratio test compare to the numerical evidence you found above?  
\item At last, state the power series you came up with and the Interval of Convergence.
\end{itemize}
\end{exercise}

\begin{exercise}{Square Root \Coffeecup \Coffeecup \Coffeecup}
Start with the function $f(x)=\sqrt{x}$ and do the following:

\begin{itemize}

\item Explain why you can't do a power series centered at zero for $\sqrt{x}$.  ({\bf Hint:} Try it.)
\item Instead, set up a power series centered at $x=1$ for $\sqrt{x}$.  Solve for the degree 2, degree 3, degree 4, degree 5, and degree 6 power series approximations for $\sqrt{x}$ centered at $x=1$.  Accomplish this by just repeatedly plugging in $x=1$ and differentiating both sides.  Call these functions $P_2(x), P_3(x), P_4(x),$ and $P_5(x),$ respectively.
\item Make a small table of values where you list out the values of each of your $P_i$ evaluated at $x=1/2$.  Compare this to the true value of $\sqrt{1/2}$.  
\item Make a small table of values where you list out the values of each of your $P_i$ evaluated at $x=2$.  Compare this to the true value of $\sqrt{2}$.
\item Perform the Ratio Test on your series expansion for $\sqrt{x}$.  For what $x$ would you have a ratio less than one?
\item How do your results of the Ratio Test compare to the numerical evidence you found above?
\item Notice that there will be two $x$-values that cause the series to have ratio exactly equal to one when Ratio Test is applied.  Since the Ratio Test gives no info in that case, try a different test for each of those two series.
\item At last, state the power series you came up with and the Interval of Convergence.
\end{itemize}
\end{exercise}

\begin{exercise}{Reciprocal \Coffeecup \Coffeecup \Coffeecup}
Start with the function $f(x)=\frac{1}{1+x}$ and do the following:

\begin{itemize}
\item Set up a power series centered at $x=0$ for $\frac{1}{1+x}$.  Solve for the degree 2, degree 3, degree 4, and degree 5 power series approximations.  Accomplish this by just repeatedly plugging in $x=0$ and differentiating both sides.  Call these functions $P_2(x), P_3(x), P_4(x)$, and $P_5(x),$ respectively.
\item Make a small table of values where you list out the values of each of your $P_i$ evaluated at $x=1/2$.  Compare this to the true value of $\frac{1}{1+\frac{1}{2}}$.  
\item Make a small table of values where you list out the values of each of your $P_i$ evaluated at $x=2$.  Compare this to the true value of $\frac{1}{1+2}$.
\item Perform the Ratio Test on your series expansion for $\frac{1}{1+x}$.  For what $x$ would you have a ratio less than one?
\item How do your results of the ratio test compare to the numerical evidence you found above?
\item Notice that there will be two $x$-values that cause the series to have ratio exactly equal to one when Ratio Test is applied.  Since the Ratio Test gives no info in that case, try a different test for each of those two series.  
\item At last, state the power series you came up with and the Interval of Convergence.
\end{itemize}
\end{exercise}
