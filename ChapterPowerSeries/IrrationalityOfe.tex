
\section{Rational and Irrational Numbers}

One of the oldest questions in mathematics is the following:  

$$ \emph{ \text{ Which real numbers are rational, and which are irrational? } }
$$

\subsection{Rational Numbers}

Recall that the set of rational numbers is the set of all numbers that can be expressed as the ratio of two integers.  More formally: 

$$ {\mathbb{Q}}= \Bigg \lbrace \frac{a}{b} : a, b \in { \mathbb{Z} }, b \not = 0 \Bigg \rbrace $$

So to show a number is rational is usually not terribly hard... we simply need to find some integers $a$ and $b$ whose ratio is your number.
\begin{exercise}{Definition of Rational Numbers \Coffeecup}
Show that the real number 2.1 is rational.  What is your $a$ and what is your $b$?
\vspace*{1in}
\end{exercise}

If the decimal expansion is repeating and not terminating, it is slightly more tricky, as we need an infinite geometric series.  
\begin{exercise}{Repeating Decimals are Rational \Coffeecup \Coffeecup}
\begin{enumerate}
\item Use a geometric series to show that .0131313131313... is rational:
\vspace*{2in}
\item The infinite geometric series formula in fact proves that every number with a repeating decimal expansion is rational.  Explain why this is the case:
\vspace*{2in}
\end{enumerate}
\end{exercise}

\subsection{Irrational Numbers}

What is \emph{much} harder is to show that a number is irrational.  To accomplish this, we must show that it is not possible to express a number as a ratio of integers for \emph{any} choice of integers.  Since there are infinitely many, we obviously cannot run through all choices of $a$ and $b$ to verify that none of them work.  Rather,  an irrationality argument typically follows a pattern of logic known as ``proof by contradiction'' (\emph{Reductio ad absurdum}).  The basic idea is to assume the opposite of what you are trying to prove and deduce an absurd conclusion, thus implying your working assumption was false.  In this case, 
to show that a number $r \in \mathbb{R}$ is irrational via proof by contradiction:

\begin{enumerate}
\item Assume $r$ is rational.
\item Thus there exist some integers $a$ and $b$ with $r=\frac{a}{b}$.
\item Use the equation $r=\frac{a}{b}$ and known properties of the number $r$ to deduce a statement we know is false. 
\item Conclude that our assumption of $r$ being rational must have been false, so $r$ is in fact irrational.
\end{enumerate}

\subsection{The Square Root of Two}
Here is the classic proof that the square root of two is irrational.  It relies on the simple fact that a number is even if and only if it can be written as a multiple of two.  

\begin{exercise}{Irrationality of $\sqrt{2}$ \Coffeecup \Coffeecup \Coffeecup}
Fill in the blanks in the following proof:

\vspace*{.1in}

{\bf Proof: }  Assume that $\sqrt{2}$ is rational.  Then $$\sqrt{2}=\frac{a}{b}$$ for some $a,b \in \mathbb{Z}$.  We may assume that $a$ and $b$ are relatively prime.  That is, if $a$ and $b$ had a common factor greater than one, we could cancel it out of the fraction, so we may as well assume $a$ and $b$ were chosen to have no common factors greater than one.

By squaring both sides and clearing denominators in the above equation, we get $$ \underline{\hspace{2in}}$$

Thus $a^2$ is even, since it is two times an integer.  But since $a^2$ is even, $a$ must be even as well.  Therefore we can write $a=2m$ for some number $m \in \underline{\hspace{.4in}}$.  Plugging this into the above equation produces the following: $$2b^2=(2m)^2 \implies 2b^2 = 4m^2 \implies b^2 = \underline{\hspace{.4in}} $$  Thus $b^2$ is even so $\underline{\hspace{.4in}}$ is also even.

However, $a$ and $b$ were taken to be relatively prime!  If $a$ and $b$ are both even than the initial fraction $\frac{a}{b}$ was not a reduced fraction as assumed since we could cancel a two out of the top and bottom.  Thus we have reached a contradiction, so our initial assumption must have been false.

We conclude the square root of two is irrational!
{\bf QED}
\end{exercise}

There is a lot of interesting history behind this result.  The square root of two first came up as a result of the Pythagorean Theorem being used to measure the diagonal of a square.  The exact history of this result is sketchy!  Different authors tell different versions of the story.

\begin{exercise}{A Brief Literature Search \Coffeecup \Coffeecup}
Regardless of which account is correct, history indicates that the result is at least how old?  Who are two other Greeks who may have discovered the result?
Cite your sources below.
\vspace{1in}
\end{exercise}

\subsection{Euler's Constant $e$}

Significantly harder than proving the irrationality of the square root of two is proving that the number $e$ is irrational.  The reason $e$ is fundamentally harder to deal with is because $e$ turns out to be {\bf transcendental } whereas $\sqrt{2}$ is not.  This means that $\sqrt{2}$ is a root of a polynomial equation with integer coefficients, whereas $e$ is not the root of any such polynomial. 

\begin{exercise}{The Square Root of 2 is not Transcendental  \Coffeecup}
Justify the above claim by finding a polynomial with integer coefficients that has $\sqrt{2}$ as a root.

\vspace*{1in}
\end{exercise}

Notice that in our proof of the irrationality of $\sqrt{2}$, our very first step was exactly the condition that the number is a root of a polynomial with integer coefficients.  Since we have no such polynomial for $e$, we must start our proof based on something else.  It turns out this ``something'' is the infinite series expansion for $e!$ $\leftarrow$ {\small{\emph{excitement, not factorial}}}  Let us step through this together.  

\begin{exercise}{Proof That $e$ is Irrational \Coffeecup \Coffeecup \Coffeecup}

Fill in the missing parts of the argument below:

\vspace*{.2in}

{\bf Proof:} First let's write $e$ as an infinite series.  To do this, recall the power series for the exponential function:  $$ e^x = 1+x+\frac{x^2}{2!}+\frac{x^3}{3!}+\frac{x^4}{4!}+ \cdots $$ Set $x=1$ to get an infinite series for $e$: $$ e=e^1= \underline{\hspace{3in}}$$

Now we proceed by the classic proof technique: proof by \underline{\hspace{1in}}. Accordingly, we assume $e$ is in fact rational and then show that it leads to an absurd statement.

Thus, assume $e$ is rational.  Then there exist some $a,b \in \mathbb{Z}$ such that

$$ e = \underline{\hspace{1in}} $$

For any $n$ we can multiply both sides of the above equation by $\underline{\hspace{1in}}$ to obtain $$ n!be=n!a  $$  Notice that the right-hand side is an integer because $n!$ and $a$ both are.  Thus the left-hand side must also be an integer.  Notice however, the left-hand-side can be decomposed as follows:

$$ n!be=bn!\left( 1+\frac{1}{1!}+\frac{1}{2!}+\cdots +\frac{1}{n!}\right) + bn!\left(\frac{1}{(n+1)!} +\frac{1}{(n+2)!}+\frac{1}{(n+3)!}+\cdots\right)$$ The term $bn!\left( 1+\frac{1}{1!}+\frac{1}{2!}+\cdots +\frac{1}{n!}\right)$ is an integer because 

\vspace{.1in}

\underline{\hspace{2.5in}}.

\vspace{.1in}

We now proceed to show that the second term is not an integer for sufficiently large $n$.  This will produce a contradiction since the left-hand side was an integer for all $n$.  In particular, we will show that  $bn!\left(\frac{1}{(n+1)!} +\frac{1}{(n+2)!}+\frac{1}{(n+3)!}+\cdots\right)$ $=b\left( \frac{1}{n+1}+ \frac{1}{(n+1)(n+2)}+\frac{1}{(n+1)(n+2)(n+3)}+\cdots\right)$ is between $\frac{b}{n+1}$ and 
$\frac{b}{n}$, which for $n>b$ will be nonintegral.  Proceeding, we have:

\begin{align}
\frac{1}{n+1} &< \frac{1}{n+1}+ \frac{1}{(n+1)(n+2)}+\frac{1}{(n+1)(n+2)(n+3)}+\cdots \\
 &< \frac{1}{n+1}+ \frac{1}{(n+1)(n+1)}+\frac{1}{(n+1)(n+1)(n+1)}+\cdots \\
 &= \frac{1}{n+1}+ \frac{1}{(n+1)^2}+\frac{1}{(n+1)^3}+\cdots \\
 &= \frac{\frac{1}{n+1}}{1-\frac{1}{n+1}} \\
 &= \frac{1}{n}
\end{align} The above steps are justified as follows.  The inequality on line (1) is true because \underline{\hspace{2in}}.  To get from line (1) to line (2) we notice that \underline{\hspace{3in}}.  To get from line (2) to line (3) is simply algebra.  To get from line (3) to line (4) is an infinite geometric series with common ratio \underline{\hspace{1in}} and initial term \underline{\hspace{1in}}.  To get from line (4) to line (5) is again just algebra (check this step).

Multiplying all sides by $b$, we get $$ \frac{b}{n+1} < b \left( \frac{1}{n+1}+ \frac{1}{(n+1)(n+2)}+\frac{1}{(n+1)(n+2)(n+3)}+\cdots\right) < \frac{b}{n} $$ as desired, which completes the proof. 

\hspace{4in} {\bf QED}
\end{exercise}
\vspace{.2in}

This proof is actually not the original proof this fact; Euler first proved it using continued fractions.  The proof above is due to Fourier.  It was again proved in 1873 by Charles Hermite.  It appeared in his landmark paper \emph{ Sur la fonction exponentielle}.  Hermite's method worked not only for $e$ but also lead to the proof that $\pi$ is irrational.  This proof however is quite a bit harder than the proof for $e$, and will be saved for later in your mathematical career, along with the proofs that $\pi$ and $e$ are transcendental!