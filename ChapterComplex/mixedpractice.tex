\section{Mixed Practice}
\subsection{Warm Ups}
These are good problems for reinforcing the vocabulary and foundational concepts of this chapter.

\begin{exercise}{\Coffeecup \Coffeecup}

\begin{enumerate}[label=\alph*.)] 

\item Find all square roots of $-i$. Write your answers in complex cartesian form.
\solushun{If $z^2 = e^{i (-\pi/2)}$ and $z=r e^{i\theta}$ then $r^2 e^{i2\theta} = e^{i(-\pi/2}$ so $r=1$ and $2\theta = \frac{-\pi}{2} + 2\pi k$ Thus $\theta = \frac{-\pi}{4} + \pi k = \frac{-\pi}{4}, \frac{3\pi}{4}$ So $z=e^{i (\pi/4)}, e^{i(3 \pi/4)} = \frac{\sqrt{2}}{2} - i \frac{\sqrt{2}}{2},-\frac{\sqrt{2}}{2} + i \frac{\sqrt{2}}{2}$
\\ }{0in}
\item Take your answers and square them to verify their square is $-i$ as you claim above.
\solushun{$\left(\frac{\sqrt{2}}{2} - i \frac{\sqrt{2}}{2}\right)^2 = \frac{2}{4} -  2i\frac{ 2}{4} + i^2 \frac{2}{4} = \frac{1}{2} -\frac{1}{2} -i = -i$ and $\left(-\frac{\sqrt{2}}{2} + i \frac{\sqrt{2}}{2}\right)^2 = \frac{2}{4} -  2i\frac{ 2}{4} + i^2 \frac{2}{4} = \frac{1}{2} -\frac{1}{2} -i = -i$
\\ }{0in}
\end{enumerate}
\AnswerKeyEntry{$z=e^{i (\pi/4)}, e^{i(3 \pi/4)} = \frac{\sqrt{2}}{2} - i \frac{\sqrt{2}}{2},-\frac{\sqrt{2}}{2} + i \frac{\sqrt{2}}{2}$
}

\end{exercise}


\subsection{Sample Test Problems}

\begin{exercise}{\Coffeecup \Coffeecup \Coffeecup}

Compute the following complex numbers in standard $a+bi$ form for $a,b \in \mathbb{R}$.  List all values if there are multiple answers.
\begin{enumerate}[label=\alph*.)] 
\item $i^{2015}$
\solushun{$i^{2015} = i^{2012+3} = i^{2012}i^3 = 1 \cdot i^3 = -i$
\\ }{0in}

\item $i^{(2i)}$
\solushun{$i^{(2i)} = e^{\ln{i^{(2i)}}} = e^{2i \ln{i}} $ Note that $ i = \cos(\pi/2) + i \sin(\pi/2) = e^{i \pi/2}$ So $e^{2i \ln{i}} = e^{2i \ln{e^{i \pi/2}}}=e^{2i \cdot i \pi/2}= e^{-\pi}$
\\ }{0in}

\item $1+i\pi-\frac{\pi^2}{2!}-\frac{i\pi^3}{3!}+\frac{\pi^4}{4!}+\frac{i\pi^5}{5!}-\frac{\pi^6}{6!}-\frac{i\pi^7}{7!}+\frac{\pi^8}{8!}+\cdots $
\solushun{$1+i\pi-\frac{\pi^2}{2!}-\frac{i\pi^3}{3!}+\frac{\pi^4}{4!}+\frac{i\pi^5}{5!}-\frac{\pi^6}{6!}-\frac{i\pi^7}{7!}+\frac{\pi^8}{8!}+\cdots =1+i\pi+\frac{(i\pi)^2}{2!}+\frac{(i\pi)^3}{3!}+\frac{(i\pi)^4}{4!}+\frac{(i\pi)^5}{5!}-\frac{(i\pi)^6}{6!}-\frac{(i\pi)^7}{7!}+\frac{(i\pi)^8}{8!}+\cdots  = e^{i \pi} = \cos(\pi) + i \sin(\pi) = -1$
\\ }{0in}

\item $\sqrt[3]{i}$
\solushun{$\left(r(e^{i\theta}) \right)^3=r^3 e^{i 3\theta}= i = e^{i \pi/2}$ \\
$\Rightarrow r = 1$ and $3 \theta = \frac{\pi}{2} + 2 \pi k$ \\
$ \Rightarrow \theta = \frac{\pi}{6} + \frac{2 \pi}{3}k \Rightarrow \frac{\pi}{6}, \frac{\pi}{6} + \frac{2 \pi}{3}, \frac{\pi}{6} + \frac{4 \pi}{3} $\\
$\Rightarrow \sqrt[3]{i} = e^{i \pi/6}, e^{i 5\pi/6}, e^{i 3\pi/2}= \frac{\sqrt{3}}{2} + \frac{i}{2}, -\frac{\sqrt{3}}{2} + \frac{i}{2}, -i$
\\ }{0in}

\item $\ln\left(4\sqrt{3}+i\right)$
\solushun{First convert $4\sqrt{3}+i$ to polar using a triangle where $4\sqrt{3}$ is the horizontal side and $1$ is the vertical side. Then $r = \sqrt{(4\sqrt{3})^2 + 1^2} = \sqrt{16 \cdot 3 +1} = \sqrt{49}=7$ also, $\theta = \arctan\left(\frac{1}{4\sqrt{3}} \right)=\arctan\left(\frac{\sqrt{3}}{12}\right)$ Thus $ln(4\sqrt{3}+i) = \ln{\left(7e^{i \arctan{\sqrt{3}/12}}\right)}=\ln{7} + i \arctan{\sqrt{3}/12}$
\\ }{0in}
\end{enumerate}

\AnswerKeyEntry{a.)~~$-i$ \newline
b.)~~$e^{-\pi}$ \newline
c.)~~$-1$ \newline
d.)~~$\frac{\sqrt{3}}{2} + \frac{i}{2}, -\frac{\sqrt{3}}{2} + \frac{i}{2}, -i$ \newline
e.)~~$\ln{7} + i \arctan{\sqrt{3}/12}$
}

\end{exercise}

\begin{exercise}{\Coffeecup \Coffeecup}

\begin{enumerate}[label=\alph*.)] 

\item State and prove Euler's Identity using power series.
\solushun{$\cos(\theta)+i \sin(\theta) = \left( 1-\frac{\theta^2}{2!} + \frac{\theta^4}{4!} -\frac{\theta^6}{6!} \cdots \right) +i \left( \theta-\frac{\theta^3}{3!} + \frac{\theta^5}{5!} -\frac{\theta^7}{7!} \cdots \right)= \left( 1+i\theta-\frac{\theta^2}{2!} -i\frac{\theta^3}{3!}+ \frac{\theta^4}{4!} + i\frac{\theta^5}{5!}-\frac{\theta^6}{6!} \cdots \right)=\left( 1+i\theta+\frac{(i\theta)^2}{2!} +\frac{(i\theta)^3}{3!}+ \frac{(i\theta)^4}{4!} + \frac{(i\theta)^5}{5!}+\frac{(i\theta)^6}{6!} \cdots \right)= e^{i \theta}$
\\ }{0in}
\item Multiply both sides of Euler's Identity by $r$.  Explain how this formula relates to our conversion between polar and cartesian coordinates.
\solushun{$r\cos(\theta)+ri \sin(\theta) = re^{i \theta}$  This shows that if $x+iy$ is the \begin{bf}{horizontal + i vertical}\end{bf} representation of a complex number, then $x= r\cos(\theta)$ and $y=ri \sin(\theta) $ so that $r$ and $\theta$ represent a radius and an angle.
\\ }{0in}
\item Prove the sine double-angle identity using Euler's Identity.
\solushun{Prove $ \sin{(2\theta)} = 2\cos(\theta)\sin(\theta)$\\
Start with $\cos(2\theta) + i \sin(2\theta) = e^{i (2\theta)}  = (e^{i \theta})^2 = (\cos(\theta) + i \sin(\theta))^2 = \cos^2(\theta) + 2 i \sin(\theta) \cos(\theta) + i^2 \sin^2(\theta) = (\cos^2(\theta) - \sin^2(\theta)) +i (2\sin(\theta)\cos(\theta))$ So $\cos(2 \theta) = \cos^2(\theta) - \sin^2(\theta)$ and $\sin(2 \theta) = 2 \sin(\theta)\cos(\theta)$
\\ }{0in}
\end{enumerate}
\AnswerKeyEntry{b.~~$r\cos(\theta)+ri \sin(\theta) = re^{i \theta}$  This shows that $r$ and $\theta$ represent a radius and an angle.
}

\end{exercise}
